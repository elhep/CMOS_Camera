% TeX encoding = utf8
% TeX spellcheck = en 

% Piotr Zdunek - Master Thesis 
\documentclass{pzdunek_master_thesis}
\usepackage[english]{babel}
%\usepackage{polski}
\usepackage{graphicx}
\usepackage{caption}
\usepackage{subcaption}
\usepackage{amsfonts}
\usepackage{amsmath}
\usepackage{amsthm}
\usepackage{wrapfig}
\usepackage{listings}
\lstset{language=C}
\usepackage{moreverb}
\usepackage{booktabs}
\usepackage{eurosym}
\usepackage{indentfirst}
\usepackage{pdfpages}
\usepackage[utf8]{inputenc}
\usepackage{enumitem}
\usepackage{multirow}
\usepackage{rotating}
\usepackage{float}
\usepackage{tocloft}
\usepackage{color}
\usepackage{mathtools}
\usepackage{setspace}\setstretch{1.2}
\usepackage{tabularx}
\usepackage{tgtermes}
\usepackage[titletoc]{appendix}
\usepackage{booktabs}
\usepackage{enumitem}
\setlist[1]{itemsep=-5pt}
\newcommand{\ra}[1]{\renewcommand{\arraystretch}{#1}}
\usepackage{color, colortbl}
\usepackage{array}
\usepackage{ragged2e}
\newcolumntype{P}[1]{>{\RaggedRight\hspace{0pt}}p{#1}}
\usepackage{footnote}
\usepackage{ctable}
%no borders around hyperlinks
\usepackage{hyperref}
\hypersetup{%
    pdfborder = {0 0 0}
}

\usepackage[nonumberlist,nopostdot]{glossaries}
\deftranslation{Glossary}{Glossary}
\newcommand{\dictentry}[2]{%
  \newglossaryentry{#1}{name=#1,description={#2}}%
  \glslink{#1}{}%
  \glsgroupskip
}%
\makeglossaries
\cleardoublepage

%\author{\Huge\href{p.zdunek@stud.elka.pw.edu.pl}{\textbf{Piotr Zdunek}}}
\nralbumu{229417}
\title{High speed multichannel camera with 10 GbE}

%\tytulang{The aim of this Master Thesis was to design a high performance }
  
%\kierunek{Microsystems and electronics systems}

%\titlepl{Szybka wielokanałowa kamera z interfejsem 10 GbE}

%\abs_text{Poniższa praca przedstawia opis projektu oprogramowania na kamerę. }

%\keywords{Camera, ZYNQ, FPGA, Scientific, Framework}

% miesi?c i rok
\date{Warsaw, 2015}

% koniec definicji

\begin{document}
%\def\tablename{Tabela}%
\maketitle
%\tableofcontents
%\newpage
%\listoffigures
%\listoftables
%\clearpage

\dictentry{BGA}{Ball Grid Array}
\dictentry{CMOS}{Complementary Metal-Oxide Semiconductor} 
\dictentry{PCIe}{ Peripherial Component Interconnect Express }
\dictentry{GTP}{Gigabit Transceiver Port}
\dictentry{AC}{Alternating Current}
\dictentry{DC}{Direct Current}
\dictentry{FPGA}{Field Programmable Gate Array}
\dictentry{IC}{Integrated Circuit}
\dictentry{JTAG}{Joint Test Action Group}
\dictentry{DDR}{Double Data Rate}
\dictentry{PCB}{Printed Circuit Board}
\dictentry{PERG}{Photonics and Web Engineering Group}
\dictentry{SFP+} {Small form-factor pluggable transceiver}
\dictentry{Gbps} {Gigabits per second}
\dictentry{GFLOPs} {Giga Floating Point Operations Per Second}
\dictentry{SPI} {Serial Peripherial Interconnect}
\dictentry{I2C} {Inter Integrated Circuit}
\dictentry{ppm} {points per milion}
\dictentry{LVDS} {Low Voltage Differential Signaling}
\dictentry{GB} {Gigabyte}
\dictentry{EDA} {Electronic Design Automation}
\dictentry{OHWR} {Open Hardware Repository}
\dictentry{TTM} {Time To Market}
\dictentry{SoC} {System-On-Chip}
\dictentry{PC}	{Personal Computer}
\dictentry{GPU}	{Graphics Processing Unit}
\dictentry{GPGPU}	{General Purpose Graphics Processing Unit}
\glsnogroupskiptrue
\clearpage
\printglossary[style=index]
\clearpage

%----------------------------------------------------------------------------------------------------------
%Introduction
\chapter{Introduction}
\label{chapt:introduction}

\chapter{Introduction}



\section{Motivation and Objectives}


\section{Requirements}
\subsection{Statement of work}
\subsection{Literature review}
\subsection{Market review} 
\section{Thesis statement} %Statement here.


%----------------------------------------------------------------------------------------------------------
%GENEZA, CEL, ZAŁOŻENIA
\chapter{Genesis, goals and assumptions}
\label{chapt:genesis}
\chapter{Genesis}

In this master thesis an implementation of Precision-Time-Protocol Timestamping Unit in FPGA fabric for scientific
camera systems is presented. The project was completed at Photonics and Web Engineering Group at the Institute of 
Electronics Systems which has a significant contribution to X-ray measurement research (TODO publikacje).  Having a scientific cooperation with
another Polish university, there was a need to develop hardware and firmware for novel extremely high-speed,
multichannel, X-ray silicon based camera. This project is undergoing a patent application, and for this reason the 
detailed description of the project cannot be included in this thesis. 

Specifically, a time synchronisation system providing an accurate UTC time was required in order to correctly control
the exposure time between the systems' channels.  This master thesis focuses on that aspect of the project.   


\section{Problem statement}
Providing an accurate timestamping for modern scientific grade camera system is a \textbf{complicated engineering
problem}. The designed hardware for the camera system used Xilinx Zynq SoC\cite{XIL:ZYNQ} which has built
in timestamping capability in the Media Access Controller (MAC). Nevertheless, the timestamping register is not available for
to be read by the operating system and programmable logic \cite[16.4.2]{XIL:ZYNQ_TRM} and the provided functionality of timestamping from
Xilinx is limited and provides low accuracy \cite[16.2.7]{XIL:ZYNQ_TRM} and significant jitter \cite{XIL:PTP_TESTS}. 
Xilinx User Guide Number 585 - Technical Rerence Manual explicitly mentions the fact that the Timestamping Unit can be 
implemented in hardware (programmable logic) in order to achieve better accuracy. This has not been done before and 
this thesis provides the solution to the mentioned problem. 

\section{Solution}
The solution for the problem is to design a Timestamping Unit (TSU) in digital system in FPGA fabric for the Zynq SoC
and use the MAC's built in PTP filtering capability to use this IP Core as a replacement for the internal built in TSU.
What is more, an Ethernet driver modification is required to exchange the TSU and an external oscillator has to be added
to the system in order to precisely run the counters in the TSU. 

\section{Statement of Originality}

This solution provides a way to perform PTP based time synchronisation using Zynq SoC. There are
other methods which provide time synchronisation of different precision such as:
\begin{itemize}
    \item GPS
    \item NTP - precision of up to
    \item PTP (by standard) - sub-milisecond precision 
    \item White Rabbit - sub-nanosecond precision 
\end{itemize}

Nevertheless, the solution provided in this master thesis is \textbf{original}. Standard PTP in the
Zynq SoC does not function properly and in order to be able to use PTP on Zynq with high precision and low jitter,
TSU needs to be implemented in digital fabric.  



%----------------------------------------------------------------------------------------------------------
%KONCEPCJA
\chapter{Concept of realisation}
\label{chapt:concept}

\chapter{Concept}\label{CH3}

%\section{X-band sensor description and interface analysis}
\section{X-band silicon sensor serial data acquisition}
\section{Sensor control}
\section{Firmware}
\section{Communication interfaces}

%\subsection{Deserialisation using CMOS CMV4000}
%\subsection{Programmable sequencer}
%
%\section{Firmware implementation}
%\subsection{Operating system architecture}
%\subsection{Software digital system control}
%
%\section{Synchronisation}
%\subsection{IEEE1588}
%\subsection{External trigger}
%
%\section{Interfaces}
%\subsection{1 GbE Ethernet}
%\subsection{Serial ATA}
%
%\section{Hardware tests}

%\section{Development platform}
%    \subsection{Xilinx Zynq SoC}
%    \subsection{Development board ZC706}
%
%\section{Design of framework camera subsystems}
%    \subsection{Data transfer}
%        \subsubsection{1 Gigabit Ethernet}
%
%        The 1 Gigabit Ethernet interface (1000BASE-X\cite{WWW:ETH_1000BASEX}, or 802.3z) is a network data transfer
%        interface. Xilinx Zynq SoC provides two possible ways of adding a support of 1 GbE, in programmable logic or using
%        build-in hardware MAC\footnote{Media Access Control\cite{WWW:MAC}}. 
%
%        Table \ref{tab:1gbe_pros_cons} presents a comparison between the two possible implementations.  
%
%
%        \begin{table}
%            \centering
%            \caption{Pros and cons of different 1 GbE implementations}
%            %\begin{tabular}{|p{3cm}||p{5cm}|p{5cm}|}
%            \begin{tabularx}{\textwidth}{|l|X|X|}
%                \hline
%                & \textbf{1 GbE with PHY}  & \textbf{1 GbE in programmable logic}  \\ 
%                \hline
%                \hline
%                \textbf{Pros}    & 
%                \vspace{0.1cm}
%                \begin{itemize}
%                    \item more reliable  
%                    \item shorter development time
%                \end{itemize}
%
%                & 
%
%                \vspace{0.1cm}
%                \begin{itemize}
%                    \item lower cost 
%                    \item more versatile 
%                \end{itemize}
%                \\  
%                \hline
%                \textbf{Cons}    &  
%
%
%                \vspace{0.1cm}
%                \begin{itemize}
%                    \item high cost (PHY chip)
%                    \item limited future upgrades 
%                \end{itemize}
%
%                & 
%                \vspace{0.1cm}
%                \begin{itemize}
%                    \item longer development time 
%                    \item less reliable  
%                \end{itemize}
%
%                \\  
%                \hline
%            \end{tabularx}
%            \label{tab:1gbe_pros_cons}
%        \end{table}
%
%        The most important aspect of the development of 1 GbE is the usefulness in scientific camera systems. From the two
%        solutions the implementation of 1 GbE using programmable logic is a better solution, because it's more versatile.  
%
%%
%%\textbf{Pros:}
%%\begin{itemize}
%%    \item No need to use PHY chip in the camera hardware (lower cost)
%%    \item Possibility to upgrade the IP Core to the 10 Gbit version in future
%%    \item Full support by Zynq SoC ~\cite{XIL:PCS_PMA}
%%    \item PTP support  
%%\end{itemize}
%%
%%
%%\textbf{Cons:}
%%\begin{itemize}
%%    \item More complicated development
%%    \item Performance is dependent on the PCS/PMA Linux driver quality
%%    \item Not supported by FreeRTOS and Baremetal code on Zynq SoC 
%%    \item Necessity of using 1 GbE SFP transceiver (additional cost)
%%\end{itemize}
%%
%%As for the camera framework, the pros far outweighs the cons. The increased complexity allows for the future improvement
%%of the whole system.   
%%
%    \subsection{Operating system architecture}
%    \subsection{Video data acquisition from the sensor}
%    \subsection{Software based digital system control}
%    \subsection{Multicamera synchronisation}
%
%In this chapter, the concept of a camera design framework is presented. Based on the requirement analysis made in Chapter 2.
%Firstly, main camera functions are listed and defined. In this chapter, the \emph{key project decisions} are made.
%For an embedded system design, some irreversible decisions have to be made at the beginning of the design, such as:
%choosing an architecture of the Main Processing Unit, sensor type, operating system etc. These reflect on the future fulfilment of the 
%requirements. 
%
%%Due to complexity of the project many members of PERG\footnote{Photonics and Web Engineering Group} were involved in the
%%process of development of the camera framework. It is clearly stated in this and following sections which parts were
%%developed by whom.  
%
%%Section~\ref{ch2:methodology} contains general information about the design methodology for the embedded system
%%engineering design, which aims to present a view on the way the project was being worked on.  
%
%\section{Design idea}\label{ch3:ida}
%
%The purpose of this project is to design a framework that allows for testing and verification of any scientific or
%task-specialised camera system. The project was designed with a bottom-up methodology in mind, taking into account how
%a typical camera design process would look like. To provide versatility of the solutions there are many features options 
%for the designer to choose from depending on the needs. 
%
%The basic idea is to have a common architecture where main design blocks can be changed or adjusted. The main component of the camera
%framework is the Main Processing Unit (MPU) which consists of programmable logic and a general purpose processor. 
%Any given sensor can be connected to the MPU either directly to the programmable logic or using a dedicated analog
%front-end (AFE), as in the case of CCD sensors. The data coming from the sensor can be processed in the programmable
%logic or in the general purpose processor. Storage can be made on a non-volatile medium or transmitted directly to the PC or
%any other device. The operating system of the camera can be changed depending on the needs and any additional sensor or
%actuator control can be implemented. 
%
%Having the basic blocks working minimizes the amount of work needed for the developer to make a prototype work. The
%The figure~\ref{FIG:BASIC_BLOCK_DIAG} presents the basic block diagram. Each of these components is defined in this 
%section and some main project decisions are made which have to be done at an early stage of development of any embedded system. 
%\begin{figure}[h!]
%    \centering
%    \includegraphics[width=12cm]{img/concept/block_diagram_basic.png}
%    \caption{Basic diagram of Camera Framework}
%    \label{FIG:BASIC_BLOCK_DIAG}
%\end{figure} 
%
%
%\section{Main project decisions}~\label{ch3:decisions}
%
%After the analysis of requirements key project decisions have been made. The main decisions were to define and
%characterise a Main Processing Unit, sensor and the interfaces that were to be used in the project.    
%
%\subsection{Main processing unit}
%
%An MPU\footnote{Main Processing Unit} for a camera consists of:
%
%\begin{itemize}
%    \item Camera interface 
%    \item High speed interface for data transmission 
%    \item Can run Embedded Linux and an RTOS
%\end{itemize}
%
%The requirement analysis (\ref{ch2:req_analysis}) suggests that an FPGA can be used as an MPU. 
%This is due to the fact that FPGAs allow for connecting any desired sensor to the camera. This might not always be done directly, 
%since in certain cases 
%(such as a CCD sensor) a specialised AFE and readout electronics have to be designed. Nevertheless, FPGAs are 
%versatile devices for this kind of application. They can also be used for video processing. The architecture allows for
%efficient video processing, which can process the video data in real-time. One of the real-world application of FPGAs in
%camera systems are ADAS\footnote{Advanced driver assistance systems} where they are used for radar and image processing
%\cite{PAPER:UAV_CAMERA,PAPER:DYN_RECONF_CAMERA,PAPER:ADAS}.
%
%The requirement of versatility of camera framework in terms of supporting different sensors (hardware requirement) and
%possibility to run a high level operating system (software requirement) leaves the following development paths for
%choosing the Main Processing Unit:
%
%\begin{enumerate}[nolistsep]
%    \item use two ICs: an  FPGA and an Application Processing Unit
%    \item use just an FPGA with soft processor done in logic   
%    \item use an SoC with FPGA and APU on one silicon die 
%\end{enumerate}
%
%The table~\ref{tab:ic_comp} presents the pros and cons of each scenario with examples of possible devices that can be used.
%
%% \begin{table}
%%   \centering
%%   \caption{Comparision of possible scenarios of choosing an MPU}
%%   \begin{tabular}{l|c|c|c|}
%%     \hline
%%     \textbf{Characteristic} & FPGA + MPU & FPGA with softprocessor & SoC (FPGA + APU) \\ 
%%     \textbf{Pros} & 
%%     Design is easily dividable. 
%%     Discrete MPU can have the highest clock speed. 
%%     An FPGA can act as glue logic for a sensor interface.
%%         & 
%%     Single IC solution. 
%%     Softprocessor is capable of running an OS like Linux or FreeRTOS\@.
%%     Softprocessor has direct access to FPGA internal bus (AHB/AXI).  
%%         &  
%%     APU is capable of running an OS like Linux or FreeRTOS\@.
%%     Hardened processor has higher clock speed than a softprocessor and comparable to a discrete MPU. \\
%%     APU has direct access to FPGA internal bus (AHB/AXI).  
%%     \hline
%%     \textbf{Cons} & 
%%     Troublesome interfacing between FPGA and MPU. 
%%     & 
%%      Softprocessor can run only on low frequencies~\ref{data:microblaze}.
%%      Interrupt-latency is higher than in SoC~\ref{art:softprocessor_rtos}.  
%%     &
%%      Cost is higher than FPGA-only solution. 
%%      \\ 
%%     \hline
%%   \end{tabular}
%%   \label{tab:ic_comp}
%% \end{table}
%% 
%
%\begin{table}
%    %    \centering
%    \caption{Comparison of possible scenarios of choosing an MPU}
%    \begin{tabularx}{\textwidth}{|c|X|X|}
%        \hline
%         &  \textbf{Pros:}   &  \textbf{Cons:} \\
%        \hline
%        \hline
%        \textbf{FPGA+AP\footnote{Application Processor}} & 
%        \vspace{0.1cm}
%        \begin{itemize}
%            \item  Discrete MPU can have the highest clock speed. 
%            \item  An FPGA can act as glue logic for a sensor interface.
%        \end{itemize}
%        &
%        \vspace{0.1cm}
%        \begin{itemize}
%            \item Troublesome interfacing between FPGA and MPU. 
%            \item Limited MPUs on the market with high-speed interfaces for data transmission.
%        \end{itemize}
%        \\  
%        \hline
%        \textbf{FPGA+SP} & 
%        \vspace{0.3cm}
%        \begin{itemize}
%            \item Single IC solution. 
%            \item Softprocessor is capable of running an OS like Linux or FreeRTOS\@.
%            \item Softprocessor has direct access to FPGA internal bus (AHB/AXI).  
%        \end{itemize}
%        &
%        \vspace{0.1cm}
%        \begin{itemize}
%            \item Softprocessor can run only at low frequency\cite{data:microblaze}.
%            \item Interrupt-latency is higher than in SoC\cite{art:softproc_rtos}.  
%        \end{itemize}
%        \\
%        \hline
%        \textbf{SoC} & 
%        \vspace{0.3cm}
%        \begin{itemize}
%            \item Integrated APU is capable of running an OS like Linux or FreeRTOS\@.
%            \item Hardened processor has higher clock speed than a softprocessor and comparable to a discrete MPU. 
%            \item APU has direct access to FPGA internal bus (AHB/AXI).  
%        \end{itemize}
%        & 
%        \vspace{0.1cm}
%        \begin{itemize}
%            \item Cost is higher than FPGA-only solution. 
%            \item Immaturity of vendor tools and silicon  
%        \end{itemize}
%        \\
%        \hline
%
%    \end{tabularx}
%    \label{tab:ic_comp}
%\end{table}
%
%
%Having analysed possible solutions an SoC as MPU was chosen. At the time of designing this project there were few
%solutions available on the market:
%
%\begin{itemize}
%    \item Xilinx Zynq SoC
%    \item Altera Cyclone V SoC / Arria V SoC
%\end{itemize}
%
%The table 1~\ref{tab:soc_comp} in Altera WP1202 Application note~\cite{UG:WP1202} presents the comparison of both
%devices.
%
%
%\begin{table}
%    \centering
%    \caption{Zynq SoC and Altera Cyclone V comparison}
%    %\begin{tabular}{|p{3cm}||p{5cm}|p{5cm}|}
%    \begin{tabularx}{\textwidth}{|l|X|X|}
%        \hline
%        \hline
%        & Xilinx Zynq 7000 & Altera SoC  \\ 
%        \textbf{Processor}    & Cortex A9 MPCore        & Cortex A9 MPCore      \\  
%        \hline
%        \textbf{No. of cores}    &  2                 & 1 or 2            \\  
%        \hline
%        \textbf{Processor Max Frequency}    &   1.0 GHz        &    1.05 GHz         \\  
%        \hline
%        \textbf{L1 Cache}    &       Data: 32 KB Instruction: 32 KB          &    Data: 32 KB Instruction: 32 KB          \\  
%        \hline
%        \textbf{L2 Cache}    &       Unified: 512 KB           &     Unified: 512 KB with ECC       \\  
%        \hline
%        \textbf{Memory Management unit}    &   Yes         &     Yes        \\  
%        \hline
%        \textbf{Floating point unit}    &   Yes              &     Yes        \\  
%        \hline
%        \textbf{Acceleration Coherency Port (ACP)}    &        Yes          &       Yes      \\  
%        \hline
%        \textbf{Interrupt Controller}    &     Generic (GIC)           &    Generic (GIC)         \\  
%        \hline
%        \textbf{On-Chip Processor RAM}    &    256 KB              &      64 KB       \\  
%        \hline
%        \textbf{DMA Controller}    &    8-channel ARM DMA330             &    8-channel ARM DMA330         \\  
%        \hline
%        \textbf{External Memory Controller}    &    Yes          &     Yes        \\  
%        \hline
%        \textbf{Memory Support}    &      LPDDR2, DDR2, DDR3L, DDR3            &    LPDDR2, DDR2, DDR3L, DDR3         \\  
%        \hline
%        \textbf{Ext. Memory Bus Max. Freq}    &     533 MHz             &    533 / 400 MHz         \\  
%        \hline
%        \textbf{Peripherals}    &    2x SPI, 2x I2C, 2x 10/100/1000 Ethernet, 2x USB 2.0, 2x UART, 2x CAN, 2x 16 bit timers    
%        &   2x SPI, 4x I2C, 2x 10/100/1000 Ethernet, 2x USB 2.0, 2x UART, 2x CAN, 4x 32 bit timers\\
%        \hline
%        \textbf{FPGA Fabric}    &     Artix, Kintex             &    Cyclone V, Arria V         \\  
%        \hline
%        \textbf{FPGA Logic density}    &     25K to 462 K LE             &    25K to 462 K LE         \\  
%        \hline
%        \textbf{High speed transceivers}    &    Higher-density devices only             &    Available at all densities
%        \\  
%        \hline
%        \textbf{Analog Mixed Signal}    &    2x 12bit, 1 MSPS ADC             &   Not available 
%        \\  
%        \hline
%    \end{tabularx}
%    \label{tab:soc_comp}
%\end{table}
%
%As far as parameters are concerned this table shows that Altera SoC are more advanced devices. Nevertheless, at the 
%time of writing, no affordable Cycone V or Arria V development kits were available, and first production samples 
%of the device were being sold to customers. Xilinx SoC was a more mature device at this time with broad documentation. Having had experience with previous Xilinx
%products, the Zynq SoC has been chosen. Furthermore, the use of Zynq SoC allows for using any kind of combination of an ARM A9 processor and FPGA. Assuming
%that the throughput of a bus between the devices is high enough and that an FPGA or an Application processor supports high
%speed data transmission. Nevertheless, one has to bear in mind that using a single chip is more cost effective
%and energy efficient.
%
%
%\subsubsection{Xilinx Zynq SoC}
%
%The Xilinx Zynq 7000 is the latest and one of most advanced SoC FPGA~\cite{WWW:SOC_FPGA} devices available on the market.  
%An FPGA SoC is a combination of programmable logic, dual core ARM Cortex A9 Application processor, multigigabit
%transceivers and many peripherals like Ethernet SPI, I2C, CAN, USB etc. This combination eases the development of an
%embedded system and decreases the cost as well. Additionally, Zynq SoC can support two ranks of DDR3
%memory, one connected via integrated memory controller and the second generated in programmable logic. This gives a 
%high throughput between an FPGA and APU. The fabric and processors are connected with each other via four High Performance
%AXI buses, two General Purpose buses, and a built in DMA engine PL330. The performance of each of these buses is
%listed below:
%
%\begin{itemize}
%    \item AXI High Performance - 1200 MB/s   
%    \item AXI General Purpose - 600 MB/s   
%    \item PL330 DMA - 100 MB/s   
%\end{itemize}
%
%The High Performance buses are used for data transmission, whereas the General Purpose ones are used for control of
%IP Cores.   
%
%\paragraph{APU Architecture}
%
%Xilinx Zynq SoC ARM Cortex A9 MPCore processor is equipped with 32 KB of L1 Cache for Data and Instruction, MMU, FPU
%and a NEON SIMD module. The Cores share L2 Cache of size 512 KB, General Interrupt controller (GIC), On-Chip-Memory
%and a DDR3 Memory controller. 
%
%What is crucial for the camera framework is that the dual core ARM A9 MPCore can work in an AMP\footnote{Asymmetric
%Multiprocessing} scenario where two cores can run independent operating systems. Figure~\ref{FIG:ZYNQ} presents the block diagram of Zynq SoC. 
%
%\begin{figure}[h!]
%    \centering
%    \includegraphics[width=16cm]{img/concept/zynq.png}
%    \caption{Diagram of Zynq SoC~\cite{PIC:ZYNQ_DIAG}}
%    \label{FIG:ZYNQ}
%\end{figure} 
%
%\subsection{Main data transmission interface}
%%TODO Correct style 
%One of the main requirements of the camera framework is the ability to transmit or store acquired video data using a multigigabit
%interface. Among numerous interfaces Serial ATA and Ethernet were chosen as main means of transmitting data.
%These interfaces were chosen because of their high throughput, versatility and maturity. Ethernet currently is becoming 
%more and  more popular in every market, not only for networking applications~\cite{WWW:ETHERCAT}. 
%Serial ATA is widely used in storage devices like Hard Disk Drives and Solid State Drives.
%
%Another key aspect is the way those interfaces can be incorporated into the camera framework. Chosen MPU supports
%multigigabit interfaces using GTP or GTX transceivers, but the design of the IP Core has to be done by the developer,
%which can be time consuming. The easiest approach is to use tested and verified solutions.  
%
%Xilinx provides an IP Core for 1000Base-X Ethernet~\cite{XIL:PCS_PMA} communication, which is supported by Zynq 
%and comes with numerous application notes~\cite{XIL:PG047,XIL:XAPP1082} describing the way of setting up the
%communication. Additionally, it is possible to upgrade the core
%to a 10 GbE version.
%
%On top of this, the camera framework can be easily upgraded to send data through 1 GbE Ethernet or 10 GbE Ethernet using the same SFP
%connector. For the 1 GbE Ethernet to work an IP core is used: Xilinx PCS/PMA 1 GbE IP
%Core~\cite{XIL:PCS_PMA}, which works as a PHY\footnote{Physical layer} for the MAC\footnote{Media Access Controller} 
%incorporated into Zynq SoC~\cite{DSH:TRM}. The pros of this solution, apart from relying on external PHY are presented
%in table\ref{tab:phy}.
%
%%Pros:
%%
%%\begin{itemize}
%%    \item lower cost (1G/10G PHY chips are expensive)
%%    \item more versatile - one can choose 1G or 10GbE depending on application
%%\end{itemize}
%%
%%Cons:
%%\begin{itemize}
%%    \item longer development time 
%%    \item less reliable  
%%\end{itemize}
%%
%\begin{table}
%    \centering
%    \caption{Pros and cons of using an IP based PHY}
%    %\begin{tabular}{|p{3cm}||p{5cm}|p{5cm}|}
%    \begin{tabularx}{\textwidth}{|l|X|X|}
%        \hline
%        & Pros & Cons  \\ 
%        \hline
%        \hline
%        \textbf{Physical PHY}    & 
%        \vspace{0.1cm}
%        \begin{itemize}
%            \item more reliable  
%            \item shorter development time
%        \end{itemize}
%
%        & 
%        \begin{itemize}
%            \item high cost
%            \item fixed solution 
%        \end{itemize}
%
%        \\  
%        \hline
%        \textbf{IP Core PHY}    &  
%
%        \vspace{0.1cm}
%        \begin{itemize}
%            \item lower cost 
%            \item more versatile 
%        \end{itemize}
%
%        & 
%        \begin{itemize}
%            \item longer development time 
%            \item less reliable  
%        \end{itemize}
%
%        \\  
%        \hline
%    \end{tabularx}
%    \label{tab:phy}
%\end{table}
%
%
%
%As for 10GbE, one can use a FADE protocol~\cite{WWW:FADE}. 
%For Serial ATA there is an Open Source IP Core which allows for the data transmission using this interface~\cite{WWW:SATA}.  
%
%\subsubsection{1 GbE with PCS/PMA IP Core}
%1 GbE Ethernet can be used in the framework and the PHY is incorporated via the PCS/PMA IP Core. This idea requires
%the use a SFP transceiver module due to the fact that the transmission is driven directly from the FPGA.  
%
%%  \subsubsection{10 GbE with FADE IP Core}
%%  10 GbE Ethernet can be used in the framework and the PHY is incorporated via the FADE IP Core. This idea requires
%%  to use a SFP transceiver module to work due to the fact that the transmission. 
%
%\subsubsection{Serial ATA}
%The Serial ATA IP Core that is used in this project is available for anyone~\cite{WWW:SATA}. Due to the fact that it
%was designed to be used for Series 6 Xilinx Virtex devices some design changes had to be made. 
%%The clock network was redesigned by mgr inz. Adrian Byszuk. 
%
%%\subsection{Data transfer}
%
%%As for the connectivity the data from the sensor can be send directly via 1 GbE Ethernet using dedicated hardware in 
%%Zynq. This solution provides only 1 Gbps of throughput for the Ethernet, but is very well supported by the 
%%software (drivers are available).
%%
%%I have also tried using SATA interface from Open Cores Repository (opencores.org), but unfortunately this IP core 
%%behaves in a very unstable way and had to be discarded. 
%%
%%The solution that I have chosen, that would provide 10 Gbps of throughput is to use a dedicated FADE protocol IP, 
%%which allows for simple FPGA-PC 10 Gbps Ethernet-based connectivity. This IP core is also available on the opencores. 
%%It was designed by PhD. Wojciech Zabolotny and is extensively used at this moment.  
%%
%%I can only add that PCIe also can be used in this design, but it needs further development.
%%
%%I would like to move further and describe the processing and control. 
%
%
%
%\subsection{Sensor}
%
%%  \subsubsection{CMOSIS (ON Semi) CMV4000} 
%There is a wide variety of sensors available on the market, which might be used in a high speed camera.
%The choice of the sensor must correspond to the most common use of the camera framework. For this project two types of
%sensors were chosen:
%\begin{itemize} 
%    \item standard CMOS sensor - CMOSIS CMV4000\cite{WWW:CMV4000} 
%    \item counting CMOS sensor
%\end{itemize} The different interfaces on both devices allowed for juxtaposing the different methods of interfacing sensors to the 
%Main Processing Unit.
%
%\subsubsection{CMOSIS (ON Semi) CMV4000}~\label{CH3:CMV}
%
%
%The CMV4000\cite{WWW:CMV4000} is a high sensitivity, pipelined global shutter CMOS image sensor with $2048 x 2048$ pixel resolution capable
%of HD format. Pipelining allows exposure during read out. The state-of-the-art pixel architecture offers true correlated
%double sampling (CDS), reducing the fixed pattern noise and dark noise significantly. The imager integrates 16 LVDS
%channels each running at $ 480 Mbps $ resulting in a $ 180 fps $ frame rate at full resolution at 10 bits per pixel. Driving and
%read-out are programmed over a serial peripheral interface. An internal timing generator produces the signals needed for
%read-out and exposure control of the image sensor. External exposure triggering remains possible. A 12 bit per pixel
%mode is available at reduced frame rate.
%
%%
%%Sensor użyty w projekcie to przetwornik video o rozmiarze 1”, wykonany w technologii CMOS. Charakteryzuje się rozdzielczością 2048x2048 punktów i maksymalną szybkością przetwarzania ramek obrazu o wartości 180 klatek/s przy zegarze o częstotliwości 480 MHz (maksymalna częstotliwość zegara dla tego sensora). Sensor posiada 2 wejścia zegarowe – zegar systemowy oraz zegar LVDS. Programowanie rejestrów sensora odbywa się przy pomocy interfejsu SPI. Dane z przetwornika wystawiane są na 16 (lub mniej w zależności od ustawień) wyjść LVDS. Dodatkowo sensor posiada jeszcze 1 wyjście sterujące LVDS, na które wystawiane są sygnały sterujące strumieniem danych oraz 1 wyjście zegara LVDS danych. Dane wystawiane są z częstotliwością równą połowie częstotliwości zegara podanego na wejście zegarowe LVDS na obu zboczach zegara (DDR). Dane na poszczególnych liniach LVDS są wysyłane przez sensor w postaci szeregowej – ze współczynnikiem 10:1 lub 12:1 (w zależności od ustawień przetwornika AC). Wysyłanych jest równolegle 16 serializowanych fragmentów ramki obrazu – po jednym na każde wyjście LVDS. Aby odtworzyć pierwotną postać obrazu, należy najpierw zdeserializować, a następnie przegrupować dane otrzymane od sensora.
%%
%\begin{figure}[h!]
%    \centering
%    \includegraphics[width=8cm]{img/concept/cmv4000.jpg}
%    \caption{CMV4000 picture\cite{WWW:CMV4000} }
%    \label{fig:CMV4000}
%\end{figure} 
%
%\subsubsection{Counting silicon sensor} 
%In order to properly design the camera framework so that it is flexible in terms of connecting a different sensor,
%an additional device with different architecture has to be used. The camera framework is intended for the use
%in both scientific and commercial markets. Apart from commercial CMOS sensors like CMOSIS CMV4000
%another common type of sensor is CCD or a counting sensor (detector). The first requires the design of a custom Analog
%Front End. This type of design is well described in literature~\cite{MASTER:GK}. For this reason a custom counting
%silicon sensor has been used as a second device. These kinds of sensors are widely used in X-ray imaging and synchrotron
%radiation applications~\cite{WWW:CNT_SENSOR1}. 
%
%The architecture of the readout for this sensor is different from CMOSIS CMV4000, which also adds more in-depth
%information on how it can be implemented in hardware. Data from such a sensor has to be acquired in the form of a continuous stream of data. 
%Physically there is a frame to read, but the architecture is different and more similar to a long shift register, where each pixel
%contains a counter. 
%
%\begin{figure}[h!]
%    \centering
%    \includegraphics[width=8cm]{img/concept/counting.jpg}
%    \caption{Counting sensor picture\cite{PIC:UFXC}}
%    \label{fig:counting}
%\end{figure} 
%
%
%
%\section{Camera control}
%
%The control of the camera has to be performed over a secure, reliable and high throughput interface. It also should not
%add unnecessary complexity to the design. 
%
%Nowadays, more and more embedded systems are equipped with Ethernet or WLAN so that the communication with the device can
%be performed using a well known TCP/IP protocol. That is why Ethernet was chosen as a main interface for control of the
%camera. Control commands are sent over TCP/IP to the camera in a text format for ease of development. Due to
%security vulnerability, an encryption of the commands can be added as a feature in future versions of the design.
%
%The commands are executed by a Linux socket server, which is the main program running on the OS. The execution of commands
%is done directly by the server application, or by an RTOS. Another key function is the control of the logic fabric through the
%operating system. This is achieved by using a register connected to the main General Purpose AXI bus of Zynq SoC,
%which can be accessed both by operating system and logic fabric.  
%
%%TODO: ADD diagram showing how does the control works
%
% 
%\begin{figure}[h!]
%    \centering
%    \includegraphics[width=8cm]{img/concept/control_block_diagram.png}
%    \caption{Camera control diagram}
%    \label{FIG:CONTROL_BLOCK_DIAG}
%\end{figure} 
%
%
%
%%\section{Video data processing}
%
%%DMA is responsible for transferring of the data from the sensor to the DDR3 memory and optionally to the 
%%processing system. 
%%
%%This way the data can be buffered in order to send them through dedicated 1 GbE Hardware IP available in Zynq or 
%%e.g. PCIe IP Core given by Xilinx. 
%%
%%In case we want to use a 10 GbE interface there is no need for DMA, because the translation would have to be made. 
%%When 10 GbE interface is needed data are being send directly to this IP Core without the use of DMA.  
%%
%%So at this point we know how are the data acquired form the sensor.
%%
%
%
%%\subsubsection{Serial ATA}
%
%\section{Operating system}\label{ch2:proc_sys}
%
%The operating system is one of the most crucial aspects of any embedded system. The decision of using Zynq SoC with 
%ARM Cortex-A9 which is ARMv7 architecture narrows the possible operating systems that can be run on the processor. 
%ARM Cortex A9 can run baremetal code and many commercial and open source "low-level" operating systems such 
%as: FreeRTOS~\cite{WWW:FREERTOS}, Micrium $\mu$OS~\cite{WWW:MICRIUM}, and GreenHills~\cite{WWW:GREENHILLS}. 
%
%The choice for a specific OS is design dependent. In this thesis, the two most common operating systems were chosen:
%Linux/GNU and FreeRTOS. Both are supported by Xilinx Zynq. Furthermore, three is a way to design an AMP OS where
%on one core there is a Linux and on the other there is FreeRTOS. The communication between Linux OS and RTOS is done by using a
%dedicated \emph{rpmsg} protocol~\cite{XIL:AMP_LINUX_FREERTOS}. 
%
%%As far as processing and control is concerned - Cortex A9 is used for this purpose. Because it's a two core chip, 
%%the designer (that's me) can take advantage of asymmetric multiprocessing. A scenario in which two cores are running 
%%different operating systems in parallel. This approach eases the development, due to the fact that Linux can easily 
%%provide high system level features like TCP/IP stack whereas RTOS is running in a deterministic way and eases the 
%%development of sometimes complex sensor IP control and interrupt handling.
%%
%%Linux is responsible for control of camera operation as well as data processing (optional), and FreeRTOS 
%%is responsible for sensor data acquisition. 
%
%\section{Multichannel operation}\label{ch2:multichannel}
%
%Another key requirement is multichannel operation. In order to provide synchronization between multiple cameras, 
%a PTP (Precision Time Protocol) interface can be used. It is based on Ethernet, which will be available in the design. 
%PTP is widely used by the industry solution which allows for microsecond 
%synchronization. An Open-Source daemon for PTP is evaluated, running on embedded Linux. It allows for synchronisation
%between the cameras in one common Local Area Network. For simple synchronisation, one camera can be a master, whereas
%for more advanced applications a dedicated hardware master with GPS synchronisation should be used.  
%
%
%%Both of this approaches are available in my design, but are not fully implemented and tested. 
%
%%\subsection{PTP}
%
%
%
%%\section{Modelling and desing methodology}~\label{ch2:methodology}
%%
%%  \subsection{Xilinx's Ultra Fast Design Methodology}
%%  
%%  \subsection{Six-Sigma DMAIC}
%%  
%%  \subsection{Waterfall}
%%  
%%  \subsection{Spiral waterfall}
%%  
%%  \subsection{Agile}
%%  
%%  \subsection{UML}
%%  
%%  \subsection{SysML}
%
%\section{Realization methodology}
%%This section presents the approach of the camera framework design. Firstly, the determination of key project decisions
%%a \emph{design process} needs to be planned. 
%%
%%
%%\begin{enumerate}
%%
%%    \item Digital system design - Ethernet and software - hardware control
%%    \item Basic baremetal software for testing of the basic hardware functions
%%    \item Integration of software and hardware
%%    \item Design of Petalinux OS as a main operating system
%%    \item Integration of Petalinux into the hardware subsystem (Control) 
%%
%%\end{enumerate}
%%
%%
%This section presents the realisation phase of the camera framework project, given the requirements set in the 
%concept phase as well as key project decisions. The framework design is very complex, which is why some methodology
%had to be applied in order to successfully realise the project. The key goal of the project is to provide an engineer
%with a basis for development of his own camera system. In order to do so, key design concepts of Zynq
%7 series SoC design methodology are shown first, then the use of those concepts in realizing given functionalities is 
%presented. This way, a reader can both understand the theory and the factual design of the camera framework.
%
%In order to properly design a camera system on Zynq SoC, a certain methodology had to be used. Features and functions
%described in the concept phase were developed one after another, first using a baremetal operating system and, when it was
%ready, it was moved to the camera server application. Baremetal OS allows for direct access to the memory when the cache
%is set specifically so that MMU is not interfering in the operation. 
%
%The methodology can be summed up in the following points:
%
%\begin{itemize}
%
%    \item Develop a feature / function in digital system
%    \item Design a testbench and test the proper operation of digital system IP
%    \item Prepare the control of the IP using AXI Control Register and test it using Baremetal program
%    \item Integrate the control to the OS 
%
%\end{itemize}
%
%
%\section{Project use case}
%This thesis project is meant to be used as a base for a scientific camera design prototype. A designer wishing to use it has
%to choose the required functions and build the system adding needed custom features. Any engineering design has to
%consider how the project will be used before it is realised, so that during development phase it can be done properly. 
%An example of a scientific camera design prototype scenario using the framework is presented below.
%
%The design team needs to verify whether a specific sensor meets the requirements in terms of speed and sensitivity.  
%It might be that a CCD sensor that needs to be cooled to a very low temperature and the mechanical design is extremely
%complicated. Usually, the design would have to be done from scratch even when using off-the-shelf electronics. 
%With the use of the framework, the design team can quickly prototype working electronics acquisition and control system
%for the camera and focus on the design of mechanics (for example). The explanatory process can be arranged in the
%following way:  
%
%\begin{enumerate}
%    \item Design electronics (AFE) to drive the sensor
%    \item Add IP Core for control and use ready digital system for data acquisition from the sensor 
%    \item Modify the control and status information of the system so that it works correctly with the sensor
%    \item Design mechanics with the possibility of using a development board as main electronics
%\end{enumerate}
%
%This is a simplified approach to camera design, but it is obvious that when using the framework, a large
%portion of the work doesn't have to be done for a prototype. This gives a huge advantage to any design team and
%decreases the time-to-market and costs of a scientific design. Obviously there are drawbacks in the form of a fixed type
%of main processing unit (Xilinx Zynq SoC), but in this kind of project the most versatile device has to be used and it
%can be viewed as a temporary solution in a prototype design.   
%
%%The design team can use the framework by adding needed functionality to the project.  a Development Board Xilinx ZC706 and decrease the time to bring
%%the prototype to work.  
%
%%\section{Project team}~\label{ch3:team}
%%This is clearly stated in the sections describing the specific parts. 
%%People who greatly helped me during the development are those mentioned: Grzegorz Kasprowicz, Damian Krystkiewicz,
%%Maciej Trochimiuk, Andrzej Abramowski, Bartłomiej Juszczyk, Adrian Byszuk oraz Krzysztof Sielewicz. 
%
%%For clearance the following modules in the Camera Framework were designed directly by me:
%%\begin{itemize}
%%    \item Petalinux operating system configuration 
%%    \item FreeRTOS implementation on ZC706  
%%    \item modified Serial ATA IP Core in two SSD drive configuration
%%    \item 1000Base-X support with PCS/PMA 
%%    \item Deserialisation of data from silicon counting sensor  
%%    \item baremetal software for digital system control 
%%    \item PTP synchronisation between multiple ZC706 Development Boards
%%    \item system tests
%%\end{itemize}
%%
%


%----------------------------------------------------------------------------------------------------------
\chapter{Realisation}

\section{Firmware}
\label{chapt:realisation}
\input{./chap/sw.tex}

\section{Digital system design}
\label{chapt:fpga}
\input{./chap/fpga.tex}

%----------------------------------------------------------------------------------------------------------
\chapter{Tests}
\label{chapt:Tests}
\chapter{Tests}
\label{ch6:tests}

This chapter presents tests carried out in order to verify the proper working of the realised framework. 
%In order to test the developed framework a new hardware was used with a dedicated sensor. 

%Hardware and sensor was provided by Creotech Instruments S.A. which was developing a scientific camera with cooperation with Cracow University
%of Technology. The framework was adapted to the new hardware and acquisition from the dedicated sensor was performed
%using one of the before mentioned techniques. 


\section{Custom Silicon Sensor - Ethernet data transmission}

\subsection{Setup}
\subsection{Test results}


\section{CMV4000 - Serial ATA SSD video data storage}

\subsection{Setup}
The development system was configured with CMV4000 acquisition IP and Serial ATA. The video data correctness was tested
using Ethernet interface which was sensing acquired frames using RTP\footnote{Real-Time Transport} protocol. 

\subsection{Test results}
Video data reception from the CMV4000 CMOSIS sensor was successful. Figure~\ref{PIC:OUTPUT0} presents the acquired image
of the fan which was used for testing the speed of the camera reception.

\begin{figure}[H]
    \centering
    \includegraphics[width=13cm]{img/tests/output0.png}
    \caption{Acquired image from the CMV4000 sensor}
    \label{FIG:OUTPUT0}
\end{figure}


Unfortunately the SSD data acquisition was unreliable. Small portions of data were possible to be written to both of the
hard drives, nevertheless the IP Core constantly lost communication and was resetting itself. As it turned out, the
reason for that was an interal mechanism for sustaining the \textbf{Link-Up} signal high. This signal, when it is high,
indicated that the calibration of the IP Core was successful and that the data can be written or read. The internal
mechanism detected the state of this signal and in a situation where the signal was low, an \textbf{internal reset} was
performed, which \textbf{disrupted} the video data storing. 


\section{Summary}

As far as testing of the camera framework is concerned, the results partially fulfil the requirements, but more work
has to be performed in order to mitigate the limited performance and functionality in some aspects. These limitation
result mainly in wrong assumptions in terms of correctness of operation of some IP Cores which was not possible to
foresee before development. This also underlines a crucial aspect of embedded system design, that before a hypothesis is
not tested and verified, one can not be certain whether some design works or not. 

The most important functions of the framework work correctly and provide a valuable foundation for camera development.  





%----------------------------------------------------------------------------------------------------------
\chapter{Summary}
\label{chapt:remarks}
\input{./chap/remarks.tex}

%DODATKI
\begin{appendices}
\label{chapt:appendix}
\chapter{Developement version of pixel ordering}
\section{Image data}
Nominal size of CCD sensor is 4096x4096 pixels. Each pixel is sampled with 18-bit resolution and consists of two samples: one of video level and another of black level. This gives in total 4096x4096x2x2 = 64MB of data. Due to high SNR requirements each pixel has to be oversampled. Two agreed oversampling factors are 4x and 16x. Taking into account oversampling, final frame sizes are respectively 256MB and 1024MB. \\
Image data is contained in one continuous memory block but due to CCD readout and oversampling method, order of image pixels in memory is different than physical one in CCD matrix. CCD matrix is divided into four identical groups of pixels which have independent readout channels as shown on figure \ref{fig:readout}. \\
Due to order of ADC triggering final image data layout is different. Image data ordering in memory (for 4x oversampling mode) is presented on figure \ref{fig:oversample4}.

%\begin{figure}[H]
%\centering
%\includesvg[clean, pretex=\relscale{0.7}, width=\textwidth, svgpath = pict_ipc/]{image_data_format}
%\caption{Image data format for 4x oversampling}
%\label{fig:oversample4}
%\end{figure}

\begin{figure}[H]
\centering
\includegraphics[width=\textwidth]{pict_ipc/image_data_format.png}
\caption{Image data format for 4x oversampling}
\label{fig:oversample4}
\end{figure}

\emph{pix-n-H-v-0} means: n-th pixel from section H video level sample 0. Similarly \emph{pix-n-E-b-2} means: n-th pixel from section E black level sample 2.

\section{Sensors list}

\begin{table}
\footnotesize
\begin{tabular}{ | l | l | l | l | l | l | l | l | l | }
\hline
	Board & Parameter & Sensor & Resolution & Accuracy & Range & Ext. Range & \begin{tabular}{@{}c@{}} Max Read \\ Freq. [Hz] \end{tabular} & Qty \\ \hline
	Comm. & Humidity & STH25 & 0,04 \%RH & +/- 1,8 \%RH & 10 – 90 \%RH & 0-100 \%RH & 0.125 & 1 \\ \hline
	 & Temperature &  & 0,04 \degree C & +/- 0,2 \degree C & 0-60 \degree C & -40-120 \degree C & 0,2 – 0,033 & 1 \\ \hline
	 & Temperature & MAX6639 & 0,125 \degree C & +/- 1\degree C & 0 – 125 \degree C & - & 8(1) & 1 \\ \hline
	Main & Humidity & HDC1000 & 0,006\%RH & +/- 3 \%RH & 10 – 80 \%RH & 0-100 \%RH & 6.6E-2 & 1 \\ \hline
	 & Temperature &  & 0,01 \degree C & +/- 0.2 \degree C & -40-125 \degree C & - & - & 1 \\ \hline
	 & Acceleration & ADXL343 & 0,004g & +/- 1\% & \begin{tabular}{@{}c@{}}+/-16g,+/-8g \\ +/-4g/+/-2g\end{tabular} & - & 0.1 – 3200 & 1 \\ \hline
	 & Temperature & ADS1248 & 5,96u \degree C & +/- 1\degree C & +/- 50\degree C & - & 5-2000 & 3 \\ \hline
	 & Coolant flow & MAX6639 & - & +/- 3\degree C & 2000-16000RPM & - & - & 1 \\ \hline
	 & T. alarm (set) & LM75 & 0,3515625\degree C & +/- 2,\degree C & -55 – 125 \degree C & - & 0.33 & 1 \\ \hline
	CCD & Humidity & HDC1000 & 0,006\%RH & +/- 3 \%RH & 10 – 80 \%RH & 0-100 \%RH & 6.6E-2 & 1 \\ \hline
	 & Temperature &  & 0,01 \degree C & +/- 0.2 \degree C & -40-125 \degree C & - & - & 1 \\ \hline
\end{tabular}
\end{table}

\end{appendices}
%BIBLIOGRAFIA / LITERATURA
%----------------------------------------------------------------------------------------------------------
\bibliographystyle{unsrt}
\bibliography{bib}
\addcontentsline{toc}{chapter}{Bibliography}
\end{document}

