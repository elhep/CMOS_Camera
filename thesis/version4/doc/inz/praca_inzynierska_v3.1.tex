% TeX encoding = utf8
% TeX spellcheck = pl_PL

% Piotr Zdunek - praca inżynierska

%\documentclass{praca_inz_pw}
\documentclass{praca_mgr_pw_mod}
\usepackage[english, polish]{babel}
\usepackage{polski}
\usepackage{graphicx}
\usepackage{caption}
\usepackage{subcaption}
\usepackage{amsfonts}
\usepackage{amsmath}
\usepackage{amsthm}
\usepackage{wrapfig}
\usepackage{listings}
\lstset{language=C}

\usepackage{moreverb}
\usepackage{booktabs}
\usepackage{eurosym}
\usepackage{indentfirst}
\usepackage{pdfpages}
\usepackage[utf8]{inputenc}
\usepackage{enumitem}
\usepackage{multirow}
\usepackage{rotating}
\usepackage{float}
\usepackage{tocloft}
\usepackage{color}
%\usepackage{xcolor,colortbl}
\usepackage{mathtools}
%\usepackage{tablefootnote}
\usepackage{setspace}\setstretch{1.2}
\usepackage{tabularx}
\usepackage{tgtermes}
\usepackage[titletoc]{appendix}
\usepackage{booktabs}
\usepackage{enumitem}
\setlist[1]{itemsep=-5pt}
\newcommand{\ra}[1]{\renewcommand{\arraystretch}{#1}}
\usepackage{color, colortbl}
\usepackage{array}
\usepackage{ragged2e}
\newcolumntype{P}[1]{>{\RaggedRight\hspace{0pt}}p{#1}}


\usepackage{footnote}
\usepackage{ctable}

\usepackage{hyperref}
\hypersetup{%
    pdfborder = {0 0 0}
}

\usepackage[nonumberlist,nopostdot]{glossaries}
\deftranslation{Glossary}{Słownik terminów}
\newcommand{\dictentry}[2]{%
  \newglossaryentry{#1}{name=#1,description={#2}}%
  \glslink{#1}{}%
  \glsgroupskip
}%
\makeglossaries
\cleardoublepage

%\author{\Huge\href{p.zdunek@stud.elka.pw.edu.pl}{\textbf{Piotr Zdunek}}}
\nralbumu{229417}
\title{Rekonfigurowalny akcelerator obliczeniowy z procesorami DSP i FPGA w standardzie AMC}

\tytulang{The aim of this Bachelor Thesis was to design a high performance processing AMC (\textit{Advanced Mezzanine Card}) module, consisting of a modern FPGA \textit{Xilinx} Artix-7 XC7A200T and two 8-core DSPs \textit{\textit{Texas Instruments}} TMS320C6678. Additionally, high speed SRIO and PCIe links reconfigurability between chips is possible due to the use of switches ADN4604 and PEX8616. The first chapter describes the characteristics of high performance processing embedded systems and presents a review of available devices on the market. Next chapters present module's units specification, schematics, PCB design and firmware. In the end signal and power integrity module's simulations were carried out which verifiy proper design. }

\kierunek{Elektronika i Inżynieria Komputerowa}

\titlepl{Rekonfigurowalny akcelerator obliczeniowy z procesorami DSP i FPGA w standardzie AMC}

\abs_text{Poniższa praca przedstawia opis projektu akceleratora obliczeniowego z dwoma ośmio-rdzeniowymi procesorami DSP TMS320C6678 firmy \textit{Texas Instruments} oraz układem FPGA \textit{Xilinx} XC7A200T w standardzie AMC (\textit{Advanced Mezzanine Card}),  służącego do przetwarzania danych. Dzięki zastosowaniu przełączników szybkich interfejsów szeregowych, układów ADN4604 firmy \textit{Analog Devices} i PEX8616 firmy \textit{PLX}, istnieje możliwość modyfikacji połączeń pomiędzy układami. Wstęp zawiera charakterystykę systemów wbudowanych dedykowanych przetwarzaniu sygnałów oraz zestawienie urządzeń tego samego typu, dostępnych na rynku. W kolejnych rozdziałach przedstawiona została koncepcja poszczególnych systemów, projekt schematów, obwodów drukowanych oraz oprogramowania uruchamiającego. Na końcu przedstawione zostały symulacje integralności sygnałowej i zasilania zaprojektowanego obwodu drukowanego.}

\keywords{FPGA, DSP, PCB, AMC, PERG, MTCA}

% miesi?c i rok
\date{Warszawa, 2014}

% koniec definicji

\begin{document}
\def\tablename{Tabela}%
\maketitle
\clearpage
\tableofcontents

\newpage
\listoffigures
\listoftables
\clearpage

\dictentry{PICMG}{PCI Industrial Computer Manufacturers Group}
\dictentry{AMC}{Advanced Mezzanine Card}
\dictentry{FMC} {FPGA Mezzanine Card}
\dictentry{BGA}{Ball Grid Array}
\dictentry{PCie}{ Peripherial Component Interconnect Express }
\dictentry{SRIO}{Serial Rapid IO}
\dictentry{SGMII}{ Serial Gigabit Media Interconnect Interface}
\dictentry{GTP}{Gigabit Transceiver Port}
\dictentry{AC}{Alternating Current}
\dictentry{DC}{Direct Current}
\dictentry{DSP}{Digital Signal Processor}
\dictentry{FPGA}{Field Programmable Gate Array}
\dictentry{IC}{ Integrated Circuit}
\dictentry{IPMI}{Intelligent Platform Management Interface}
\dictentry{GEM}{Gas Electron Multiplier}
\dictentry{WEST} {W Environment Stedy-state Tokamak}
\dictentry{JET} {Joint European Torus}
\dictentry{MIL} { $\frac{1}{1000} cala = 0.254  mm$}
\dictentry{JTAG}{Joint Test Action Group}
\dictentry{DDR}{Double Data Rate}
\dictentry{PCB}{Printed Circuit Board}
\dictentry{PERG}{Photonics and Web Engineering Group}
\dictentry{SAS} {Serial Attached SCSI}
\dictentry{SFP+} {Small form-factor pluggable transceiver}
\dictentry{Gbps} {Gigabits per second}
\dictentry{GFLOPs} {Giga Floating Point Operations Per Second}
\dictentry{SPI} {Serial Peripherial Interconnect}
\dictentry{I2C} {Inter Integrated Circuit}
\dictentry{ppm} {points per milion}
\dictentry{TCLK} {Telecom Clock}
\dictentry{FCLK} {Fabric Clock}
\dictentry{HCSL} {High Speed Current Steering Logic}
\dictentry{LVDS} {Low Voltage Differential Signaling}
\dictentry{GB} {Gigabyte}
\dictentry{EMIF} {Extended memory interface}
\dictentry{EDA} {Electronic Design Automation}
\dictentry{MTCA} {Micro Telecommunications Computing Architecture}
\dictentry{OHWR} {Open Hardware Repository}
\dictentry{TTM} {Time To Market}
\dictentry{ASIC} {Application-specific integrated circuit}
\dictentry{SoC} {System-On-Chip}
\dictentry{PC}	{Personal Computer}
\dictentry{GPU}	{Graphics Processing Unit}
\dictentry{GPGPU}	{General Purpose Graphics Processing Unit}
\dictentry{AIF}	{Antenna Interface}
\dictentry{TSIP}	{Telecom Serial Interface Port}
\glsnogroupskiptrue
\clearpage
\printglossary[style=index]
\clearpage

%WSTĘP
\chapter{Wstęp}
\label{chapt:wstep}
% TeX encoding = utf8
% TeX spellcheck = pl_PL

%ROZDZIAL 1 - WPROWADZENIE

Współczesne systemy pomiarowe w eksperymentach fizyki wysokich energii są zaawansowanymi systemami analogowo-cyfrowymi o niestandardowych parametrach. Eksperymenty fizyczne JET \cite{JET} czy WEST \cite{WEST}  podczas pracy generują terabajty danych na sekundę i aby móc obsłużyć taką ilość danych potrzebne są rozwiązania dedykowane, nierzadko przełomowe. Przykładem takiego systemu może być system pomiarowy służący do detekcji promieniowania X z detektora GEM \cite{GEM} wykorzystywany w eksperymencie JET i pozwalający na analizę plazmy w reaktorze termojądrowym. 

 Niniejsza praca stanowi opis projektu urządzenia służącego do przetwarzania danych -\textbf{akceleratora obliczeniowego}, który zostanie wykorzystany w nowym systemie pomiarowym projektowanym przez zespół PERG \textit{Beam Position Monitor Digital Back End} \cite{BPMDBM}.
 
  %W pierwszym rozdziale opisany został system pomiarowy detektora GEM oraz charakterystyka i porównanie dostępnych na rynku urządzeń do przetwarzania danych. Następnie przedstawiona została koncepcja projektu akceleratora obliczeniowego, opis realizacji schematów elektrycznych oraz projekt obwodów drukowanych. Kolejne rozdziały zawierają opis wykonanych symulacji integralności sygnałowej i integralności sieci zasilającej oraz  projekt oprogramowania uruchamiającego. 
  
  W pierwszym rozdziale zawarto opis systemu pomiarowego detektora GEM oraz ogólną charakterystykę akceleratorów obliczeniowych wraz z porównaniem dostępnych na rynku urządzeń tego typu. Następnie przedstawiona jest geneza, cele i założenia projektowe. Rozdziały trzeci, czwarty i piąty zawierają opis koncepcji projektu, realizacji schematów elektrycznych oraz projekt obwodów drukowanych. Dalej, w rozdziałach szóstym i siódmym zaprezentowano wyniki wykonanych symulacji integralności sygnałowej i zasilania oraz projekt oprogramowania uruchamiającego. Ostatni rozdział zawiera wnioski końcowe podsumowujące wykonany projekt. Na końcu znajdują się dwa dodatki, pierwszy opisujący pliki projektowe znajdujące się na dołączonej płycie, a drugi zawiera charakterystykę standardów MTCA i AMC.
  
%##################################################################################
\section{System pomiarowy detektora GEM}
Detektor GEM służy do pomiaru promieniowania w eksperymentach fizycznych. Zjonizowane cząstki o określonej energii w przepływającym przez detektor gazie są powielane przez specjalne płyty z otworami, do których przyłożone jest wysokie napięcie. Wewnątrz odbywa się pomiar energii przepływających elektronów. Pomiar promieniowania jest możliwy dzięki wzmocnieniu jakie można uzyskać w detektorze. Przykładem wykorzystania takiego detektora jest eksperyment JET gdzie detektor GEM jest częścią detekcyjną spektrometru X. 

	\begin{figure}[!here]
	\begin{center}
	\includegraphics[width=10cm]{grafika/detectors.png}
	\end{center}
	\caption{Detektor GEM w JET}
	\label{GEM_PHOTO}
	\end{figure}

Do analizy danych z detektora zaprojektowany został system pomiarowy przez zespół PERG z Instytutu Systemów Elektronicznych Wydziału Elektroniki i Technik Informacyjnych Politechniki Warszawskiej. System pomiarowy składa się z:
\begin{itemize}
\item detektora GEM
\item  analogowych kart pomiarowych Analog Front End (AFE)
\item kart FMC z przetwornikami ADC dołączonymi do płyty-matki  (\textit{Carrier-board})
\item płyty głównej w standardzie Mini ITX z procesorem x86 i systemem operacyjnym Linux 
\item jednostki zasilającej
\end{itemize}

	\begin{figure}[!ht]
	\begin{center}
	\includegraphics[width=14cm]{grafika/GEM_block_diagram.png}
	\end{center}
	\caption{Struktura systemu pomiarowego GEM}
	\label{GEMSTRUCTURE}
	\end{figure}


	
System posiada 256 wejściowych kanałów pomiarowych, które zamienia w jeden wyjściowy będący histogramem energii widma fotonów w czasie ustalonym podczas pomiaru. Przetwarzanie danych w systemie pomiarowym GEM odbywa się w kilku etapach. Początkowo dane zebrane przez analogowe karty pomiarowe (AFE) są zamienione na postać cyfrową za pomocą kart FMC z przetwornikami ADC. Na kartach odbywa się również kontrola sygnałów wyzwalających oraz identyfikacja ładunków wykonana w FPGA. Następnie dane z kart są analizowane w układach FPGA znajdujących się na \textit{Carrier Board}, których zadaniem jest integracja histogramów. Ostatnim etapem jest zebranie wszystkich danych pomiarowych z czterech \textit{Carrier Board} i ich finalna integracja w \textit{Backplane}. Tam też odbywa się zapis pomiaru do pamięci DDR. Cały system jest podłączony do komputera PC gdzie dane mogą być diagnozowane za pomocą oprogramowania Matlab firmy Mathworks \cite{MATLAB}.  

Detekcja promieniowania o danej energii odbywa się poprzez pomiar napięcia z detektora. Aby rozróżnić energię fotonów, które nakładają się na siebie, potrzebna jest dokładniejsza analiza sygnału z zastosowaniem cyfrowego przetwarzania sygnałów. Obecny system pozwala na to w ograniczonym zakresie i wiele pomiarów jest odrzucanych. Dlatego zdecydowano się na zaprojektowanie dedykowanego urządzenia, które będzie służyło do przetwarzania i analizy danych pomiarowych z detektora w czasie rzeczywistym. 

\subsection{Nowa wersja systemu}
Obecny system pomiarowy detektora GEM jest bazą do nowego systemu pomiarowego \textit{Beam Position Monitor Digital Back End}. System BPM DBE składać się będzie z kasety MTCA oraz z kart rozszerzeń AMC oraz FMC. Projekt jest umieszczony na ogólnodostępnym repozytorium \textit{Open Hardware Repository} \cite{BPMDBM}. 


	\begin{figure}[!here]
	\begin{center}
	\includegraphics[width=6cm]{grafika/bpm.png}
	\end{center}
	\caption{Wizualizacja kasety MTCA, nowej wersji systemu pomiarowego detektora GEM}
	\label{GEM_BPM}
	\end{figure} 
  
   
    
Karty FMC wykorzystywane w poprzedniej wersji systemu, mogą być wykorzystane w BPM DBE dzieki zaprojektowanemu modułowi AMC FMC Carrier \cite{AFC}, który jest płytą-matką dla kart FMC. Dzięki temu system posiada taką samą funkcjonalność jak poprzedni. Standard AMC staje się coraz bardziej popularny w zastosowaniach fizyki wysokiej energii \cite{MTCA_4} ze względu na niskie koszty i dużą elastyczność systemu. 

Do systemu potrzebne jest urządzenie służące do przetwarzania danych - \textbf{akcelerator obliczeniowy}, który pozwoli na analizę sygnałów czasie rzeczywistym. 

%##################################################################################
\section{Charakterystyka akceleratorów obliczeniowych}

Akcelerator obliczeniowy jest to system wbudowany (\textit{ang. embedded system}) służący przetwarzaniu danych. Charakteryzuje się dużą mocą obliczeniową przy relatywnie niskim poborze mocy oraz dużą przepustowością interfejsów wejść/wyjść. Do akceleratorów obliczeniowych możemy zaliczyć np. karty graficzne czy dedykowane moduły np. w standardzie AMC, FMC lub PCIe posiadające specjalizowane układy przetwarzające DSP lub FPGA. 

Urządzenia tego typu znajdują zastosowanie np. w nowoczesnych systemach pomiarowych fizyki wysokich energii, gdzie analiza sygnałów musi być wykonywana w czasie rzeczywistym. Innym przykładem mogą być szybkie kamery o wysokiej rozdzielczości.

 
\subsection{Układy przetwarzania} % 
Przetwarzanie w akceleratorach obliczeniowych jest wykonywane przez procesory, posiadające odpowiednią architekturę dzięki której można uzyskać bardzo dużą wydajność obliczeń. Poniższej przedstawiono charakterystykę poszczególnych typów układów, które mogą być wykorzystane do przetwarzania danych. 

\subsubsection{Procesory DSP}
Procesory DSP charakteryzują się architekturą zaprojektowaną specjalnie dla przetwarzania sygnałów (stąd też nazwa \textit{Digital Signal Processor}). Dzięki temu potrafią wykonywać skomplikowane operacje matematyczne równolegle. Specjalizowana architektura minimalizuje pobór mocy. Przykładowo procesor DSP firmy Texas Instruments \cite{COMPANY:TEXAS} TMS320C6678 posiada teoretyczną moc obliczeniową 160 GFLOPs przy ok. 10 W TDP (\textit{ ang. Total Dissipated Power}). 

\begin{itemize}
\item Zalety:
\begin{itemize}
\item bardzo korzystny stosunek mocy obliczeniowej do pobieranej energii
\item obsługa wielu szybkich interfejsów szeregowych
\item niski koszt układu
\item niski pobór mocy
\end{itemize}

\item Wady:
\begin{itemize}
\item rozwój oprogramowania jest skomplikowany i długotrwały
\end{itemize}

\end{itemize}

\subsubsection{Układy programowalne FPGA}
Dużą popularnością w systemach przetwarzania sygnałów cieszą się układy programowalne FPGA (\textit{Field Programmable Gate Array}). Największą zaletą tych układów jest możliwość dowolnego łączenia komórek logicznych wewnątrz układu co pozwala na tworzenie zaawansowanych systemów cyfrowych np. filtrów, bloków kryptograficznych itp..  Dodatkowo nowoczesne FPGA obsługują szybkie interfejsy szeregowe \cite{FPGA:GTP}. Rozwój tych układów w ostatnich latach pozwolił na minimalizację pobieranej mocy i znaczne zwiększenie ilości jednostek logicznych dzięki czemu, układy te znajdują coraz szersze zastosowanie w systemach przetwarzania sygnałów. 

\begin{itemize}
\item Zalety:
\begin{itemize}
\item możliwość realizacji dowolnych systemów cyfrowych
\item obsługa szybkich interfejsów szeregowych
\item średni pobór mocy
\end{itemize}

\item Wady:
\begin{itemize}
\item wysoki koszt układu
\item  rozwój oprogramowania jest skomplikowany i długotrwały
\end{itemize}

\end{itemize}

\subsubsection{Układy GPGPU}
GPGPU (\textit{ang. General Purpose Graphics Processing Unit}) są to jednostki obliczeniowe stosowane w kartach graficznych, komputerów klasy PC. Układy te charakteryzują się bardzo dużą wydajnością ze względu na wieloprocesorową architekturę. Na czele zastosowań przetwarzania stoi firma NVIDIA \cite{NVIDIA} z architekturą CUDA \cite{CUDA}, która posiada bardzo dopracowane środowisko do rozwoju oprogramowania (tzw. SDK \textit{Software Developement Kit}).  

\begin{itemize}
\item Zalety:
\begin{itemize}
\item bardzo duża wydajność
\item dopracowane środowisko programistyczne 
\end{itemize}

\item Wady:
\begin{itemize}
\item bardzo duży pobór mocy
\item komunikacja ograniczona do interfejsu PCIe
\item nieopłacalny rozwój własnych urządzeń
\end{itemize}
\end{itemize}

\subsubsection{Układy ASIC}
ASIC są specjalizowanymi układami zaprojektowanymi od podstaw do zastosowania w specyficznych zastosowaniach np. do kompresji wideo. Wykonanie takiego układu jest bardzo kosztowne opłaca się to tylko w wypadku produkcji idącej w tysiącach sztuk. Z drugiej strony dzięki temu układ jest nieporównywalnie szybszy i oszczędniejszy w zużyciu energii od zwykłych procesorów. Rozwój oprogramowania ogranicza się zwykle do konfiguracji rejestrów wewnętrznych, co bardzo skraca czas realizacji projektu. 


\begin{itemize}
\item Zalety:
\begin{itemize}
\item bardzo duża moc obliczeniowa w dedykowanym zastosowaniu np. kompresja wideo
\item mały pobór energii
\item łatwa konfiguracja
\end{itemize}

\item Wady:
\begin{itemize}
\item kosztowna realizacja układu
\end{itemize}

\end{itemize}

\subsubsection{Procesory x86}
Procesory z architekturą x86 są sercem każdego komputera klasy PC. Na rynku istnieją dwie firmy oferujące układy tego typu: Intel \cite{Intel} oraz AMD \cite{AMD}. Największą zaletą tych procesorów jest uniwersalność, dzięki swojej architekturze i superskalarności posiadają bardzo dużą wydajność w większości zastosowań. Dodatkowo układy te są kompatybilne wstecz względem zestawu instrukcji dzięki czemu rozwój oprogramowania jest  łatwy.

Mimo bardzo korzystnego stosunku pobieranej energii do mocy obliczeniowej, układy te bardzo rzadko wykorzystuje się w systemach wbudowanych ze względu na znaczne wymagania mocowe. Przykładowo procesor firmy \textit{Intel} z serii Core i7 odznacza się maksymalną pobieraną mocą na poziomie 140 W TDP (\textit{ang. Total Dissipated Power}). Dodatkowo procesory te często wymagają dodatkowych układów do obsługi szybkich interfejsów szeregowych.  

\begin{itemize}
\item Zalety:
\begin{itemize}
\item bardzo duża wydajność
\item łatwy rozwój oprogramowania.
\end{itemize}

\item Wady:
\begin{itemize}
\item bardzo duży pobór mocy
\item brak bezpośredniej obsługi szybkich interfejsów szeregowych
\item nieopłacalny rozwój własnych urządzeń
\end{itemize}

\end{itemize}


\subsubsection{Porównanie}
Wszystkie układy dedykowane zastosowaniom przetwarzania mają swoje wady i zalety. Wybór konkretnego typu zależy głównie od zastosowania.
Warto podkreślić, że układy DSP i FPGA dzięki swojej architekturze pozwalają na wykonanie tych samych algorytmów równie szybko (bądź szybciej) jak procesory x86 czy GPGPU jednak jest to okupione czasem realizacji oprogramowania. Z drugiej strony często nie ma możliwości wykorzystania procesorów x86 czy GPGPU w systemach wbudowanych ze względu na bardzo duży pobór mocy. Kolejną kwestią jest też nasycenie rynku, zwykle nie opłaca się projektować dedykowanego urządzenia z układami GPGPU, ponieważ na rynku istnieje tak duża ilość dostępnych produktów, że wykonanie konkurencyjnego urządzenia jest bardzo trudne. 
%
%W tabeli [\ref{tbl:cpu_comp}] zamieszczone jest porównanie wcześniej omówionych układów pod względem wydajności, pobieranej mocy, ceny i łatwości programowania. 
%
%\begin{table}[here]
%\centering
%\scriptsize
%    \begin{tabular}{p{2.5cm}| p{2.5cm} |c|c|c|c}
%	\toprule
%    \textbf{Układ} & \textbf{Wydajność} & \textbf{Pobierana moc} & \textbf{Time To Market}  & \textbf{Rozwój oprogramowania}  & \textbf{Kosz}\\
%    \midrule
%    FPGA 	& 	Wysoka		& 	Średnia		&	Długi		&	Trudny			&	Średni\\
%    DSP		& 	Wysoka   		&	Średnia		& 	Długi		&	Trudny			&	Niski\\
%    ASIC 	& 	Bardzo wysoka	& 	Mała			&	Krótki		&	Łatwy				&	Niski\\
%    GPGPU	& 	Bardzo wysoka	& 	Bardzo duża		&	Średni		&	Relatywnie łatwy		&	Średni\\
%    x86		&	Bardzo wysoka	&	Bardzo duża		& 	Średni		&	Łatwy				& 	Wysoki\\
%    \end{tabular}
%	\caption{Porównanie układów dedykowanych przetwarzaniu}
%	\label{tbl:cpu_comp}
%\end{table}
 

\subsection{Interfejsy komunikacyjne}
Bardzo istotnym aspektem w systemach przetwarzania są interfejsy komunikacyjne. Najczęściej spotykanymi interfejsami w systemach wbudowanych są PCI Express, USB, Ethernet oraz RapidIO. Najważniejszym parametrem interfejsów komunikacyjnych jest ich przepustowość, rodzaj transmisji, kontrola błędów w pakietach itp. Istotna jest też kwestia zastosowania, jedne interfejsy zostały zaprojektowane jako sposób komunikacji między układami wewnątrz urządzenia, inne natomiast pozwalają na przesył danych na duże odległości. Poniżej przedstawiono opis najbardziej popularnych interfejsów w systemach wbudowanych.  

\subsubsection{PCI Express} 
PCIe to szybki interfejs szeregowy będący rozwinięciem standardu PCI. Interfejs zapewnia znacznie wyższą przepustowość, zmniejsza ilość połączeń między układami, zapewnia również korekcję błędów transmisji oraz możliwość podłączania urządzeń w czasie pracy (\textit{tzw. hot-plug}). Dane przesyłane są różnicowo za pomocą standardu LVDS z wykorzystaniem kodowania 8b/10b \cite{8B_10B} (w wersji 1.0). Interfejs pozwala na pracę z wykorzystaniem wielu linii (\textit{ang. links}), co pozwala na uzyskanie większej przepustowości. 

 PCIe jest wykorzystywany jako podstawowy interfejs komunikacyjny np. w komputerach PC - między kartą graficzną a chipsetem, jak również w dedykowanych systemach wbudowanych. Zarówno współczesne układy FPGA jak i DSP pozwalają na komunikacje za pomocą tego interfejsu.  
%
%Obecna wersja standardu PCIe to wersja 3.0, kolejna (4.0) jest w trakcie realizacji. PCI Express 3.0 zmienia standard kodowania na 128b/130b i zapewnia transfer na poziomie 985 MB/s na linię.  
%
%Sieć połączeń w standardzie PCIe jest strukturą drzewiastą, gdzie jeden z układów pełni rolę korzenia (\textit{Root Complex}) a pozostałe są do niego bezpośrednio dołączone (\textit{ang. End Point}). 

\subsubsection{Serial Rapid IO}
SRIO jest interfejsem zaprojektowanym specjalnie dla systemów wbudowanych. Zapewnia równie dużą przepustowość jak PCIe i dodatkowo posiada mniej skomplikowany sposób przesyłania sygnałów kontrolnych, co minimalizuje opóźnienia i jitter. Dzięki temu pozwala na przesyłanie sygnałów, które wymagają synchronizacji czasowej bądź nawet pracy w systemach czasu rzeczywistego. 

Interfejs ten korzysta z standardu LVDS i pozwala na uzyskanie 5 Gbps przepustowości na linię. 

\subsubsection{Ethernet}
Ethernet pozwala na przesył danych wykorzystując protokół np. TCP/IP \cite{TCP}. Przepustowość tego interfejsu wynosi od 100 Mbps do 10 Gbps. Istotną kwestią jest kontrola przesyłu pakietów, która jest wykonywana na poziomie sieci (\textit{network layer}), a nie na warstwie fizycznej (\textit{physical layer}). Interfejs ten jest najczęściej wykorzystywany do przesyłu niekrytycznych danych między urządzeniami, często na większą odległość. 


\subsubsection{Inne}
Do innych interfejsów komunikacyjnych wykorzystywanych w systemach wbudowanych możemy zaliczyć interfejsy związane z pamięciami jak SATA czy SAS. Zapewniają one bardzo dużą przepustowość, ale służą głównie do przesyłu danych z dysków twardych, nie wykorzystuje się ich do komunikacji między układami scalonymi wewnątrz systemu. 

\subsection{Standardy urządzeń przetwarzania}
System wbudowany jest zwykle projektowany względem określonego standardu. Określa on specyfikację elektryczno-mechaniczną jakie musi spełniać urządzenie. Przykładem takich standardów są np. VME, AMC, FMC czy ATCA. 

\subsubsection{ATCA}
Standard ATCA (\textit{Advanced Telecommunication Computing Architecture}) specyfikuje systemy przetwarzania wykorzystywane w telekomunikacji, przemyśle i wojsku. System składa się z kart rozszerzeń ATCA (\textit{blade}) i kasety (\textit{shelf}) ATCA łączącej wszystkie moduły za pomocą tzw. \textit{Backplane}. Specyfikacja obejmuje komunikację między modułami, wymagania mechaniczne dla kart rozszerzeń etc. Organizacja PICMG \cite{PICMG}, odpowiedzialna za przygotowanie standardu, zrzesza ponad 100 organizacji na całym świecie. Największą zaletą tego systemu jest jego modułowość, w kracie można umieścić bardzo zróżnicowane moduły dzięki czemu możliwa jest realizacja systemu dopasowanego do potrzeb. 


 	\begin{figure}[!ht]
	\begin{center}
	\includegraphics[width=4cm]{grafika/blade.jpg}
	\end{center}
	\caption{Moduł ATCA wyposażony w dwa procesory Intel Xeon}
	\end{figure}
	 

\subsubsection{AMC}
Standard AMC (\textit{Advanced Mezzanine Card}) specyfikuje moduły rozszerzeń do systemów MTCA \cite{MICRO_TCA} i ATCA \cite{ATCA}. Powstał on w celu zmniejszenia kosztów w porównaniu do systemów ATCA przy zachowaniu porównywalnej funkcjonalności.  Standard opisano szczegółowo w Dodatku B  [\ref{AMC_APP}]. 

\subsubsection{FMC}
Moduły FMC (\textit{FPGA Mezzanine Card}) znajdują szerokie zastosowanie jako karty rozszerzeń do płyt-matek (\textit{carrier-board}). Pozwalają na rozszerzenie funkcjonalności działającego systemu bez potrzeby projektowania go od początku. Dzięki małym wymiarom i nieskomplikowanej  specyfikacji standardu wykonanie urządzeń jest stosunkowo łatwe. FMC wyróżnia się specjalnym, wielopinowym złączem które pozwala na transmisję z zastosowaniem szybkich interfejsów szeregowych. 

	\begin{figure}[!ht]
	\begin{center}
	\includegraphics[width=4cm]{grafika/fmc1.png}
	\end{center}
	\caption{Karta FMC firmy Creotech Instruments SA}
	\end{figure}
	

\section{Przegląd urządzeń dostępnych na rynku}
Obecnie dostępne jest wiele urządzeń, które można wykorzystać w systemie pomiarowym jako akceleratory numeryczne do przetwarzania danych. Poniżej przedstawiono charakterystyki kilku takich urządzeń dostępnych na rynku.

\paragraph {AMC V7-2c6678 firmy \textit{CommAgility}}
AMC-V7-C6678 \cite{COMMAGILITY} jest kartą przetwarzania danych zawierająca dwa procesory DSP TMS320C6678 i układ FPGA Xilinx XC7VX415T-2 zamkniętą w formacie AMC pojedynczej szerokości i pełnej wysokości. Posiada złącze SFP+ oraz złącze miniSAS. Procesory DSP, układ FPGA, złącza na panelu oraz złącze krawędziowe są połączone ze sobą siecią połączeń interfejsu SRIO. Koszt tego urządzenia to 16 tys. EUR. 

	\begin{figure}[!ht]
	\begin{center}
	\includegraphics[width=5cm]{grafika/amc_commagility.jpg}
	\end{center}
	\caption{Akcelerator obliczeń firmy Commagility}
	\label{COMMAGILITY}
	\end{figure}

\paragraph{FMC6678}
Firma 4DSP specjalizuje się w projektowaniu modułów FMC i posiada w swojej ofercie kartę z jednym procesorem TMS320C6678 i 1 GB pamięci dynamicznej DDR3. Na module znajdziemy również gniazdo Gigabit Ethernet, USB oraz pełne 60-pinowe złącze \textit{Trace} do programowania procesora poprzez dedykowany debugger. Interfejsy komunikacyjne SRIO i PCIe są dołączone do złącza FMC. Koszt modułu FMC to 2900 EUR.

	\begin{figure}[!ht]
	\begin{center}
	\includegraphics[width=5cm]{grafika/FMC667.png}
	\end{center}
	\caption{Akcelerator obliczeń firmy 4DSP w formacie FMC}
	\label{4DSP}
	\end{figure}
	
	


\paragraph {Urządzenia firmy ADVANTECH} 
Firma Advantech jest twórcą modułu ewaluacyjnego do procesora TMS320C6678 i posiada w swojej ofercie szereg urządzeń z tym procesorem. Przykładem jest 8 procesorowa karta PCIe DSPC-8682 (64 rdzenie) albo karta ACTA z 20 procesorami (80 rdzeni).

	\begin{figure}[!ht]
	\begin{center}
	\includegraphics[width=4cm]{grafika/advantech20dsp.jpg}
	\end{center}
	\caption{Akcelerator obliczeń firmy Advantech}
	\label{Advantech20DSP}
	\end{figure}

\paragraph {PDAK2H}
Moduł PDAK2H jest modułem AMC z najnowszymi układami SoC firmy \textit{Texas Instruments} z rodziny Keystone II zawierającymi 8 rdzeni DSP C66x oraz czterordzeniowy procesor ARM Cortex-A15. Twórcą modułu jest firma Prodrive. Do procesora może być dołączone do 26 GB pamięci DDR3-1333. Do złącza AMC doprowadzone jest 12 linii SRIO a na przednim panelu zamontowano złącza Gigabit Ethernet oraz UART.  
	\begin{figure}[!ht]
	\begin{center}
	\includegraphics[width=5cm]{grafika/pdak2h.jpg}
	\end{center}
	\caption{Akcelerator obliczeń firmy Prodrive}
	\label{PRODRIVE}
	\end{figure}

\paragraph{Porównanie}	
W tabeli [\ref{tbl:cots}] zestawiono wymienione wcześniej moduły wraz z porównaniem peryferiów i ceny.
\begin{table}[here]
\centering
\scriptsize
	\caption{Zestawienie komercyjnych systemów przetwarzania}
    \begin{tabular}{p{2.5cm}| p{2.5cm} |c|c|c|c}
	\toprule
    \textbf{Urządzenie} & \textbf{Procesory} & \textbf{Standard} & \textbf{Złącza}  & \textbf{Interfejsy MCH}  & \textbf{Cena}\\
    \midrule
    AMC V7-2C6678 	& 	Xilinx XC7VX415T-2  2xTMS320C6678  	& 	AMC Single width, full size		&	SFP, SAS	&	 	SRIO				&	16 000 USD\\
    FMC6678		& 	TMS320C6678  				& 	FMC single width			&	GbE		&	 	SRIO,PCIe, EMIF16		&	2900  EUR\\
    DSPC-8682 	& 	TMS320C6678, Xilinx XC3S200AN  	& 	PCIe					&	SFP, SAS	&	 	GbE				&	BD\\
    PDAK2H 		& 	66AK2H12				  	& 	AMC Single width, full size		&	GbE, UART	&	 	SRIO, PCIe			&	BD\\
    \end{tabular}

	\label{tbl:cots}
\end{table}

\section{Podsumowanie} 

Ilość rodzajów układów scalonych, interfejsów komunikacyjnych oraz standardów dla systemów wbudowanych daje duże możliwości instytucjom tworzącym takie urządzenia. Dzięki szybkiemu rozwojowi, możliwe jest zaprojektowanie akceleratorów obliczeniowych posiadających bardzo dużą moc obliczeniową oraz przepustowość akwizycji danych dopasowaną do konkretnego zastosowania. Szczególna uwaga należy się urządzeniom w standardzie AMC i FMC, które pozwalają na stosunkowo łatwe rozszerzenie funkcjonalności systemów opartych o standard MTCA. 

\subsection{Wymagania funkcjonalne akceleratora obliczeniowego}
Urządzenie do przetwarzania danych, które ma pracować w systemie \textit{BPM DBE} musi spełniać określone wymagania oraz posiadać funkcjonalności, które pozwolą na analizę danych pomiarowych. Akcelerator obliczeń powinien być kartą zgodną ze standardem AMC i posiadać układy o dużej mocy obliczeniowej w zastosowaniach przetwarzania sygnałów - FPGA i DSP. Zastosowanie dwóch typów układów pozwoli na elastyczne podejście do analizy danych. Ponadto, ze względu na to, iż system generuje bardzo dużą ilość danych (obecnie 1 Gbps na kanał) istotne jest aby akcelerator obliczeniowy posiadał dużą ilość pamięci operacyjnej. 

Zgodnie ze standardem MTCA, karty AMC komunikują się z kasetą za pomocą interfejsów komunikacyjnych np. SRIO i PCIe. Dodatkowo w systemie \textit{BPM DBE} zastosowano złącza miniSAS do komunikacji na większe odległości, dlatego projektowane urządzenie powinno posiadać możliwość komunikacji również za pomocą tego interfejsu.  Kolejnym, istotnym kryterium jest możliwość rekonfiguracji połączeń między układami. Ta funkcjonalność wydłuża czas życia urządzenia (\textit{ang. End Of Life}) ze względu na to, iż pozwala na dopasowanie właściwości do danego zastosowania. 

Na podstawie analizy dostępnych urządzeń (COTS - Commercial of the shelf) zdecydowano się na projekt własnego modułu do przetwarzania obliczeń. Głównym powodem jest minimalizacja kosztów oraz zwiększenie funkcjonalności w porównaniu do urządzeń oferowanych na rynku.

%GENEZA, CEL, ZAŁOŻENIA
\chapter{Geneza, cel i założenia pracy}
\label{chapt:geneza}
Projektowany system pomiarowy w zespole PERG Instytutu Systemów Elektronicznych \textit{Beam Position Monitor Digital Back End} jest oparty o standard MTCA. Zwiększanie funkcjonalności tego systemu polega na dołączeniu specjalizowanych modułów AMC. Jednym z potrzebnych funkcji jest przetwarzanie danych w czasie rzeczywistym, do którego wymagany jest akcelerator obliczeń. Dostępne na rynku urządzenia nie posiadają odpowiednich wejść/wyjść np. SAS/SATA lub nie pozwalają na modyfikację połączeń między układami wewnątrz urządzenia tj. nie są elastyczne; ponadto są one bardzo kosztowne. Dlatego zdecydowano się na zaprojektowanie dedykowanego urządzenia.  \\

Celem niniejszej pracy inżynierskiej było zaprojektowanie modułu w formacie AMC pozwalającego na akwizycję i analizę dużej ilości danych pomiarowych w czasie rzeczywistym, na co składa się:
\begin{itemize}
\item projekt schematów elektrycznych
\item projekt obwodów drukowanych
\item symulacje integralności sygnałowej i zasilania
\item napisanie oprogramowania uruchamiającego
\end{itemize}

Akcelerator obliczeniowy będzie rozszerzeniem projektowanego systemu pomiarowego i pozwoli na przetwarzanie i analizę danych  z zastosowaniem zaawansowanych technik cyfrowego przetwarzania sygnałów. Przykładowym zastosowaniem będą eksperymenty JET i WEST, w których obecnie działa poprzednia wersja systemu. Ze względu na to iż system pomiarowy, w którym ma pracować projektowany moduł jest w trakcie realizacji wstrzymano się z produkcją modułu.\\
 
Urządzenie zaprojektowano przy następujących założeniach:


\begin{itemize}
		\item
		zgodność ze standardem AMC
		
		\item
		możliwość rekonfiguracji połączeń między układami
		
		\item
		możliwość komunikacji poprzez interfejs SAS/SATA
		
		\item
		minimalizacja kosztów

\end{itemize}




%KONCEPCJA
\chapter{Koncepcja konstrukcji}
\label{chapt:koncepcja}
% TeX encoding = utf8
% TeX spellcheck = pl_PL

% KONCEPCJA
Poniższy rozdział przedstawia koncepcję konstrukcji akceleratora obliczeniowego. Zawarto w nim opis funkcjonalny projektowanego urządzenia wraz wyszczególnieniem poszczególnych modułów.  

\section{Opis funkcjonalny akceleratora obliczeniowego}

Akcelerator obliczeniowy można podzielić na szereg modułów. Rysunek [\ref{BLOCK_SYSTEM}] przedstawia blokowy schemat funkcjonalny projektowanego urządzenia. 

\begin{figure}[!ht]
\begin{center}
\includegraphics[width=11cm]{grafika/system_block_diagram2.png}
\end{center}
\caption{Schemat blokowy akceleratora obliczeniowego}
\label{BLOCK_SYSTEM}
\end{figure}

Lista poszczególnych modułów, w projektowanej karcie:
\begin{itemize}
\item \textbf{Moduł akceleracji sprzętowej} akwizycja i wstępne przetwarzanie danych pomiarowych
\item \textbf{Moduł przetwarzania} - przetwarzanie danych pomiarowych wymagające dużej mocy obliczeniowej
\item \textbf{Moduł przełączania interfejsów} przełączanie interfejsów szeregowych pomiędzy układami i \textit{Backplane/MCH}
\item \textbf{Układ dystrybucji sygnałów zegarowych} generacja i dystrybucja sygnałów zegarowych 
\item \textbf{Zasilanie} - generacja odpowiednich napięć
\item \textbf{IPMI} obsługa standardu IPMI i zarządzanie sygnałami kontrolnymi
\item \textbf{JTAG} programowanie i testowanie układów scalonych
\end{itemize} 



%\subsection{Zasada działania}
%Akcelerator obliczeniowy jest kartą w formacie AMC działającą w systemie MTCA. Moduł \textbf{akceleracji sprzętowej} zajmuje się akwizycją i wstępnym przetwarzaniem danych. Dane mogą być przesłane poprzez \textit{Backplane} lub poprzez złącza na panelu przednim. 
%
% Następnie przesyła dane do \textbf{jednostek przetwarzania}, w których następuje dalsza obróbka sygnałów. Przetworzone dane mogą zostać przesłane do \textit{Backplane} znajdującego się w kracie MTCA.
%
%Komunikacja między układami i kratą odbywa się za pomocą szybkich interfejsów szeregowych. Zastosowane zostały inteligentne przełączniki, aby umożliwić możliwość rekonfiguracji połączeń pomiędzy układami.
% 
%System dystrybucji zegara, IPMI, JTAG oraz sekcja zasilania zapewniają poprawną pracę urządzenia. 
%
%Standard AMC, w którym projektowano kartę opisano szczegółowo w Dodatku B [\ref{AMC_APP}].

%##################################################################################
\section{Koncepcja konstrukcji modułu akceleracji sprzętowej}
Moduł akceleracji sprzętowej składa się z układu FPGA oraz szybkiej pamięci SDRAM. Zastosowanie nowoczesnego układu FPGA daje możliwość komunikacji za pomocą wielu interfejsów szeregowych. Do modułu dołączono bezpośrednio złącza miniSAS oraz moduł przełączania interfejsów, co pozwala na komunikację ze światem zewnętrznym za pomocą interfejsów SAS, PCIe i SRIO. Zebrane dane mogą zostać zapisane na dołączonej pamięci podręcznej i poddane przetwarzaniu. Dane mogą być przekazywane dalej do jednostek przetwarzania lub do \textit{Backplane}. 
%Współczesne układy FPGA idealnie nadają się do zbierania danych ponieważ posiadają wiele transceiverów szybkich interfejsów szeregowych. 
\begin{figure}[!h]
\centering
\includegraphics[width=5cm]{grafika/acquis_block_diagram.png}
\caption{Schemat blokowy modułu akceleracji sprzętowej}
\label{BLOCK_ACKQUIS}
\end{figure}
%%##################################################################################
\section{Koncepcja modułów przetwarzania}
%Sercem modułu przetwarzania jest procesor DSP \textit{Texas Instruments} \cite{COMPANY:TEXAS} TMS320C6678 \cite{DEV:C6678}, jest to 8 rdzeniowy układ dedykowany do aplikacji wymagających dużej mocy obliczeniowej. Do procesora dołączony jest 1 GB pamięci dynamicznej SDRAM, jak również PHY interfejsu Gigabit Ethernet, co daje możliwość komunikacji z światem zewnętrznym bezpośrednio z procesora. Dodatkowo procesor wyposażony jest w szereg pamięci, które pozwalają na uruchomienie oprogramowania przygotowanego przez producenta, przykładem jest dedykowany system operacyjny Linux OS \cite{LINUX_TMS}. Razem z układem FPGA znajdującym się w jednostce akwizycji danych moduły przetwarzania tworzą system pozwalający na bardzo elastyczne podejście do przetwarzania danych i posiadają wystarczającą moc obliczeniową do zaawansowanych obliczeń w czasie rzeczywistym. W akceleratorze znajdują się dwa takie moduły połączone ze sobą szybkim interfejsem \textit{Hyperlink} posiadającym przepustowość do 50Gbps.
Projektowana karta zawiera dwa bliźniacze moduły przetwarzania, których sercem jest procesor DSP. Każdy z procesorów  jest wyposażony w pamięć dynamiczną SDRAM, jak również PHY interfejsu Gigabit Ethernet. Dodatkowo procesory wyposażono w szereg pamięci nieulotnych, które pozwalają na uruchomienie oprogramowania przygotowanego przez producenta; przykładem jest dedykowany system operacyjny Linux OS \cite{SOFT:LINUX}. 

Razem z układem FPGA znajdującym się w jednostce akceleracji sprzętowej moduły przetwarzania tworzą system pozwalający na elastyczne podejście do przetwarzania sygnałów i posiadają wystarczającą moc obliczeniową do zaawansowanych obliczeń w czasie rzeczywistym. 

Moduły połączone ze sobą interfejsem \textit{Hyperlink} \cite{HYPER} o przepustowości 50 Gbps.
\begin{figure}[here]
\begin{center}
\includegraphics[width=14cm]{grafika/processing_unit.png}
\end{center}
\caption{Schemat blokowy modułów przetwarzania}
\label{BLOCK_PROCESS}
\end{figure}
%##################################################################################
\section{Koncepcja modułu przełączania interfejsów szeregowych} 
Moduły akceleracji sprzętowej i przetwarzania oraz \textit{Backplane} są połączone ze sobą dwoma przełącznikami interfejsów szeregowych, statycznym SRIO oraz inteligentnym PCIe. Dzięki takiemu rozwiązaniu możliwa jest transparentna transmisja pomiędzy układami jak i \textit{Backplane}.

\begin{figure}[!h]
\begin{center}
\includegraphics[width=9cm]{grafika/switch_unit.png}
\end{center}
\caption{Schemat blokowy modułu przełączania interfejsów szeregowych}
\label{BLOCK_SWITCH}
\end{figure}

\section{Pozostałe moduły i peryferia}
System dystrybucji sygnałów zegarowych generuje sygnały zegarowe o odpowiedniej częstotliwości wymagane do poprawnej pracy układów znajdujących się na karcie. Zarządzaniem sygnałami sterującymi zajmuje się mikrokontroler, który również obsługuje standard IPMI wymagany do pracy karty w systemie MTCA. 

Komunikacja ze światem zewnętrznym odbywa się poprzez złącza znajdujące się na panelu przednim oraz za pomocą złącza krawędziowego.

Jak każde urządzenie elektroniczne akcelerator obliczeniowy zawiera układy zasilające generujące odpowiednie napięcia. Programowanie procesorów DSP, układu FPGA i mikrokontrolera wykonuje się poprzez zewnętrzny programator dołączony do złącz JTAG znajdujących się na układzie. 

%\section{Koncepcja systemu dystrybucji zegara}
%System dystrybucji zegara powinien pozwalać na pracę modułu \textit{ang. standalone} oraz na synchronizację z CB. Zdecydowano się na połączenie układów bufurujących i multipleksujących zegary wraz z dedykowanymi syntezerami zegara. Szczególną kwestią jest zegar interfejsu PCIe, którego jednym z trybów pracy jest tryb wspólnego zegara \textit{ang. system synchronous}, ten zegar ma oddzielny system dystrybucji na module.  
%\begin{figure}[here]
%\begin{center}
%\includegraphics[width=8cm]{grafika/clock_distribution.png}
%\end{center}
%\caption{Schemat blokowy systemu dystrybucji zegara}
%\label{BLOCK_CLK}
%\end{figure}
%
%\section{Koncepcja obsługi standardu IPMI}
%Standard IPMI jest obsługiwany przez mikrokontroler, który komunikuje się z CB w skrzyni \textit{$\mu$icroTCA}. Dodatkowo układ ten steruje sygnałami kontrolnymi oraz interfejsami konfigurującymi I2C.
%\begin{figure}[here]
%\begin{center}
%\includegraphics[width=8cm]{grafika/BLOCK_IPMI.png}
%\end{center}
%\caption{Schemat blokowy obsługi standardu IPMI}
%\label{BLOCK_IPMI}
%\end{figure}
%
%\section{Koncepcja modułu programowania układów}
%Procesory DSP, układ FPGA oraz mikrokontroler są programowane za pomocą złącz JTAG umieszczonych na module. Dwa procesory DSP są połączone ze sobą w tzw. \textit{ang. daisy-chain}. Poza złączem JTAG dla układu FPGA, sygnały tego standardu zostały dołączone do złącza AMC.  
%
%\section{Koncepcja sekcji zasilania}
%Na złączu zasilania znajdują się dwa źródła napięcia 12V i 3.3V. Niższe napięcie jest dedykowane zasileniu systemu standardu IPMI. 12V źródło napięcia musi zostać przekształcone w niższe napięcia wymagane przez poszczególne układy. Źródła napięciowe które charakteryzują się dużymi wymaganiami na prąd są przekształcone za pomocą przetwornic impulsowych, natomiast te które nie wymagają dużej wydajności prądowej są przekształcone za pomocą dedykowanych liniowych regulatorów. 

%Część układowa
\chapter{Realizacja konstrukcji akceleratora obliczeniowego}
\label{chapt:czesc_ukladowa}
% TeX encoding = utf8
% TeX spellcheck = pl_PL

Niniejszy rozdział zawiera opis projektów elektrycznych poszczególnych modułów akceleratora obliczeniowego. Głównymi modułami w projektowanym urządzeniu są moduły akceleracji sprzętowej, jednostek przetwarzania i przełączania interfejsów. Pozostałe moduły tj. system generacji sygnałów zegarowych, IPMI, sekcji zasilania zapewniają poprawną pracę całego urządzenia. 


\section{Moduł akceleracji sprzętowej}
%------------------------------------------------------------------------------------------------------------------------------------------------------------------%
\subsection{Układ FPGA XC7A200T}
Moduł akceleracji sprzętowej oparty jest o układ FPGA firmy Xilinx \cite{COMPANY:Xilinx} z serii 7 - Artix 7 XC7A200T w obudowie FFG1156, posiadający 16 transceiver'ów GTP, 215 tys. komórek logicznych oraz 10 banków udostępniających 600 wejść/wyjść (IO) dla projektanta. Zdecydowano się na ten układ ze względu na doświadczenia przy pracy z nim w innych projektach w laboratorium PERG. Układy z serii Artix odznaczają się niskim poborem mocy oraz niską ceną. Wybrany został układ w obudowie FFG1156, bo jako jedyny z serii posiada wystarczającą ilość transceiver'ów GTP.

\begin{figure}[!ht]
\centering
\includegraphics[width=10cm]{grafika/xc7a200t_banks.png}
\caption{Rozmieszczenie banków w układzie FPGA XC7A200T}
\label{ARTIX_BANKS}
\end{figure}

Projekt schematów elektrycznych został wykonany w oparciu o projekt \textit{AMC FMC Carrier} gdzie został wykorzystany ten sam układ. Projekt jest dostępny na \textit{Open Hardware Repository} \cite{OHWR}.


%------------------------------------------------------------------------------------------------------------------------------------------------------------------%
\subsubsection{Konfiguracja układu FPGA}
Układy Xilinx FPGA z serii 7 konfigurują się poprzez załadownie programu do wewnętrznej pamięci po doprowadzeniu zasilania. Źródłem programu może być zewnętrzna pamięć nieulotna bądź inny układ scalony np. mikrokontroler lub procesor DSP. Tryb konfiguracji jest zależy od stanu pinów \textbf{M[2:0]} znajdujących się w banku 0, który ustala się poprzez rezystory podciągające do zasilania albo masy (\textit{ang. pull-up, pull-down}).

Układ FPGA w akceleratorze jest uruchamiany w trybie \textit{Slave-serial} gdy piny \textbf{M[2:0]} są podciągnięte do zasilania 3.3 V (\textbf{M[2:0] = 111}). W tym stanie wymagane jest doprowadzenie zewnętrznego zegara \textbf{CCLK}, w akceleratorze obliczeniowym jest on doprowadzony z mikrokontrolera zarządzającego, układu LPC1764, który steruje załadowaniem programu do układu FPGA z pamięci FLASH. Wykorzystane zostały dwa układy pamięci nieulotnej firmy Micron \cite{COMPANY:MICRON} M25P128 \cite{M25P128} w obudowie DFN8, zawierające po 16 MB pamięci typu NOR FLASH i komunikujących się poprzez interfejs SPI. Zastosowano dwa układy pamięci aby mieć dostęp do niej zarówno z FPGA jak i mikrokontrolera oraz w celu umieszczenia tam danych konfiguracyjnych np. tablic kalibracyjnych. Istnieje również możliwość wykorzystania drugiej pamięci jako pamięci konfiguracyjnej FPGA fail-safe. W wypadku gdy program zapisany na jednej kości ulegnie uszkodzeniu, istnieje możliwość załadowania programu z drugiej kości, bez potrzeby ingerencji w obwód drukowany. Dodano również alternatywny footprint w razie problemów z dostępnością pamięci w obudowie DFN8. Układ FPGA jest dołączony do pamięci poprzez piny z banku 14: \textbf{MOSI}, \textbf{DIN}, \textbf{D2}, \textbf{D3} i \textbf{FCS\_B}.


Załadowanie pamięci programu FPGA w trybie Slave-Serial jest wykonywane przy pomocy mikrokontrolera (LPC1764), który przesyła sygnał zegarowy CCLK do FPGA. Pamięć programu jest załadowana bit po bit'cie zgodnie z narastającym zboczem CCLK. 

Pin \textbf{CFGBVS} (\textit{ang. Configuration Banks Voltage Select Pin}) w banku 0 jest ustawiony w stan wysoki, co determinuje dozwolone napięcia zasilania banków 0 oraz 14 i 15 gdzie znajdują się dodatkowe piny konfiguracyjne. Gdy \textbf{CFGBVS} jest w stanie wysokim, dozwolone napięcia to 3.3 V oraz 2.5 V; gdy w niskim odpowiednio 1.8 V oraz 1.5 V. W projekcie bank 0 oraz 14 są zasilane z 3.3 V, natomiast bank 15 z napięcia 2.5 V.

Dodatkowo układ FPGA można zaprogramować (załadować pamięć programu) poprzez interfejs JTAG za pomocą programatora i komputera PC z odpowiednim oprogramowaniem. Interfejs JTAG składa się z wejść/wyjść: \textbf{TMS}, \textbf{TCK}, \textbf{TDI} oraz \textbf{TDO} (opcjonalnie \textbf{TRST}). 

Piny \textbf{PROGRAM\_B} oraz \textbf{INIT\_B} są, zgodnie z dokumentacją, dołączone do zasilania poprzez rezystory 4.7 k$\Omega$. Dodatkowo \textbf{INIT\_B} jest dołączony do mikrokontrolera w celu możliwości wykonania ponownego załadowania programu bez konieczności resetowania całego urządzenia. Pin \textbf{DONE} jest dołączony do rezystora podciągającego 330 $\Omega$ oraz do obwodu diody LED indykującego poprawną aktualizację zawartości pamięci flash z konfiguracją FPGA.

Pin \textbf{PUDC\_B} jest dołączony do masy przez 10k$\Omega$ rezystor, pin ten uruchamia wewnętrzne podciągnięcie do zasilania pinów (\textit{SelectIOs}) po dołączeniu zasilania i podczas konfiguracji.

Bank 0 zawiera również piny \textbf{DXP}, \textbf{DXN} które są wyjściami czujnika temperatury. Dołączone zostały do układu firmy Maxim \cite{COMPANY:MAXIM}, MAX6642ATT90 \cite{MAX6642ATT90} kontrolowanego przez mikrokontroler LPC1764 poprzez interfejs I2C.

Więcej informacji dotyczących konfiguracji układów FPGA firmy Xilinx z serii 7 można znaleźć w dokumentacji producenta \cite{FPGA:UG470}.

%------------------------------------------------------------------------------------------------------------------------------------------------------------------%
\subsubsection{Opis połączeń banków 14, 15, 17, 34}

 \paragraph{Bank 14}
Bank 14 jest zasilany z napięcia 3.3 V i dołączono do niego następujące peryferia:
\begin{itemize}
\item
interfejs komunikacyjny ze złącz miniSAS
\item
MLVDS
\item
I2C (dwa oddzielne interfesy dla pamięci flash i LPC1764)
\item
diody LED
\end{itemize}

 \paragraph{Bank 15}
Bank 15 jest zasilany z napięcia 2.5V i dołączone są do niego sygnały kontrolne z układu przełącznika PCIe PEX8616  \cite{PEX8616} firmy PLX \cite{COMPANY:PLX}. 

 \paragraph{Bank 17 i 34}
Banki 17 i 34 służą do komunikacji z procesorami DSP. Banki są zasilane z napięcia 1.8 V. Dołączone interfejsy to:
 \begin{itemize}
\item
Piny GPIO sterujące trybem uruchomienia procesora
\item
Piny resetu
\item
Timer
\item
SPI
\end{itemize}

\begin{figure}[here]
\begin{center}
\includegraphics[width=12cm]{grafika/banks_fpga.png}
\end{center}
\caption{Połączenia banków układu FPGA z peryferiami}
\label{FPGA_BANKS_CONN}
\end{figure}
%------------------------------------------------------------------------------------------------------------------------------------------------------------------%
\subsubsection{Połączenia transceiverów GTP}
Układ XC7A200T w obudowie FFG1156 zawiera 4 tzw. GTP Quads \cite{FPGA:GTP} 113, 116, 213 oraz 216 każdy zawierający po 4 linie szybkich interfejsów szeregowych \textbf{GTP} (\textit{Gigabit Transceiver Port}). Do Quadów dołączone są interfejsy szeregowe SAS, SRIO i PCIe oraz sygnały zegarowe. 
\begin{itemize}
\item
Quad 113 - SRIO
\item
Quad 213 - PCIe Gen. 2
\item
Quad 216 - SAS
\item
Quad 116 - SAS

\end{itemize}
%------------------------------------------------------------------------------------------------------------------------------------------------------------------%
\subsubsection{Sygnały zegarowe}
Układ FPGA potrzebuje do poprawnej pracy 9 sygnałów zegarowych:
\begin{itemize}
\item
sygnał zegarowy PCIe dołączony do GTP Quad 213
\item
2 sygnały zegarowe do interfejsu SAS dołączone do GTP Quad 216 i 116
\item
sygnał zegarowy interfejsu SRIO dołączony do GTP Quad 213
\item
2 sygnały zegarowe dla kontrolerów pamięci SDRAM
\item
zegar CCLK (tylko podczas konfiguracji)
\end{itemize}
Sygnały zegarowe są wygenerowane przez układy CDCUN1208 \cite{CDCUN1208} oraz AD9522 \cite{AD9522}.
%------------------------------------------------------------------------------------------------------------------------------------------------------------------%
\subsubsection{Zasilanie}
Układ Xilinx Artix 7 jest zasilany z napięć 1.0 V, 1.2 V, 1.8 V, 2.5 V oraz 3.3 V, gdzie napięcia 1.0 V, 1.2 V i 3.3 V służą do zasilania wewnętrznych systemów układu, a pozostałe zasilają poszczególne banki \cite{FPGA:DS181}. 
% \begin{figure}[here]
%\begin{center}
%\includegraphics[width=15cm]{grafika/dc_table_fpga.png}
%\end{center}
%\caption{Napięcia zasilania układu FPGA \cite{DS181}}
%\label{FPGA_SUPPLY}
%\end{figure}


\begin{table}[h]
	\caption{Napięcia zasilające i pobór prądu przez układ FPGA }
    \begin{tabular}{c p{7.5cm} c c}
	\toprule
    \textbf{Linia} & \textbf{Opis} & \textbf{Napięcie [V]} & \textbf{Prąd [A]}\\
    \midrule
   VCCINT & 		Internal supply voltage 						&	1.0			&	2.926\\
    VCCAUX & 	Auxiliary supply voltage 						& 	1.8			&	0.303\\		
    VCCBRAM & 	Block RAM supply voltage 						& 	1.0			&	0.043\\
    VCCO & 		Supply voltage for 3.3 V HR I/O banks 				& 	3.3 			&	0.003\\
    VIN & 		Input voltage 							&	 (-0.2, VCCO+0.2)	&	0.003\\
    VCCBATT & 	Battery voltage 							& 	1.8			&	BD\\
    VMGTAVCC & 	Analog supply voltage for GTP transceivers			& 	1.0			&	1.116\\
    VMGTAVTT & 	Analog supply voltage for GTP termination			& 	1.2			&	0.952\\
    VDCADC & 	XADC supply relative to GNDADC 				&	1.8			&	0.025\\
    VREFP & 		Externally supplied reference voltage 				& 	1.25			&	BD\\
    \toprule
    \end{tabular}

	\label{tbl:fpga_power}
\end{table}
Wrażliwe napięcia zostały oddzielone przez zastosowanie filtrów. Ilość i wielkość kondensatorów blokujących jest zgodna z zaleceniami producenta \cite{UG483}. 

W tabeli [\ref{tbl:fpga_power}] znajduje się zestawienie wszystkich linii zasilających układ FPGA oraz estymowany pobór prądu wyliczony za pomocą arkusza kalkulacyjnego dostarczonego przez producenta \textit{Xilinx Power Estimator} \cite{XPE} \cite{XPE:UG1} \cite{XPE:UG2}. Wyliczenia zostały przeprowadzone dla typowego wykorzystania jednostek logicznych (70\%), 16 działających transceiver'ów GTP (3.3 Gbps) oraz dwóch kontrolerów pamięci SDRAM. Arkusz dołączono na płycie CD [\ref{CDROM}].

\subsubsection{Nieużywane banki}
Banki 16, 35, 36 nie zostały wykorzystane w projekcie.
%------------------------------------------------------------------------------------------------------------------------------------------------------------------%
\subsection{Pamięć SDRAM}
Pamięć SDRAM o wspólnych liniach adresowych jest nazywana (\textit{rank}). Do układu FPGA dołączone zostały dwa ranki synchronicznej pamięci dynamicznej SDRAM DDR3-1600, każda po 4 kości firmy Micron MT41J512M8RA-125:D \cite{MT41}. Szerokość szyny adresowej to 16 bitów, a szyny danych to 32 bity (po 8 bitów na kość), częstotliwość pracy to 800 MHz. Każdy rank obsługiwany jest przez dwa banki. Na bankach znajdują się tzw. \textit{Byte Groups} do których należy dołączyć sygnały danych, oraz strobe z poszczególnych kości. Nie można pamięci dołączyć w dowolny sposób. Jeden bank obsługuje sygnały kontrolne (\textit{Control}), poleceń (\textit{Command}), sygnał zegarowy, oraz sygnały adresowe. Do drugiego natomiast dołączone są sygnały danych z pamięci. 

Banki obsługujące pamięci pracują w standardzie sygnałowym SSTL. Wymagane jest, aby piny VREF były dołączone do napięcia referencyjnego.  Maksymalnie do układu można dołączyć 8 GB pamięci. Cztery kości dołączone są do banków 12 i 13, następne cztery do 32 i 33.

\begin{figure}[here]
\begin{center}
\includegraphics[width=12cm]{grafika/sdram_block.png}
\end{center}
\caption{Sposób dołączenia pamięci SDRAM do układu FPGA}
\label{FPGA_SDRAM_BLOCK}
\end{figure}



%##################################################################################
\section{Moduły przetwarzania danych}
%------------------------------------------------------------------------------------------------------------------------------------------------------------------%
\subsection{Procesor DSP}
Sercem jednostek przetwarzania jest układ \textit{Texas Instruments} TMS320C6678. Jest to ośmiordzeniowy procesor DSP pracujący na częstotliwości 1250 MHz. Układ posiada wiele interfejsów komunikacyjnych takich jak Hyperlink, SRIO, PCIe Gen. 2.0 , SGMII, TSIP. Główną zaletą tego układu jest bardzo duża moc obliczeniowa, która wynosi teoretycznie 160 GFLOPs przy niskim poborze mocy wynoszącym około 10 W. Pod tym względem układ nie ma sobie równych na rynku dlatego zdecydowano się na zastosowanie go w tym projekcie. 

Zadaniem procesorów jest przetwarzanie danych odebranych od układu FPGA \cite{DATASHEET:TMS}. Projekt schematów jest oparty o referencyjny projekt modułu ewaluacyjnego wykonany przez firmę Advantech \cite{TMDXEVM6678L} oraz o dokumentację producenta \cite{DSP:HDG}.  

\subsubsection{PCIe}
Procesor zawiera dwie linie \textit{PCIe Gen. 2.0}. Oba dołączone są do układu PEX8616. Połączenie ma sprzężenie pojemnościowe. Interfejs jest bardzo popularny w systemach wbudowanych jako ewolucja magistrali  równoległej PCI. Maksymalna przepustowość interfejsu to 2.5 Gbps na linię. 

\subsubsection{Serial Rapid IO}
Układ posiada 4 interfejsy SRIO (\textit{Serial Rapid IO}), które są dołączone do układu przełącznika gigabitowego  firmy Analog Devices \cite{COMPANY:ANALOG}, układu ADN4604 \cite{ADN4604}. SRIO jest to szybki interfejs szeregowy pozwalający na przesył danych z prędkością \textit{5 Gbps} na linię, czyli sumarycznie \textit{20 Gbps}. 

\subsubsection{Gigabit Ethernet}
Procesor dysponuje dwoma interfejsami SGMII, które pozwalają na komunikację w sieciach lokalnych z prędkością 1 Gbps. Jeden interfejs jest dołączony do gniazda RJ45 poprzez PHY firmy Vitesse \cite{COMPANY:VITESSE} układ VSC8221 \cite{VSC8221}. Pozwala to na komunikację  z procesorami DSP bezpośrednio poprzez sieć lokalną. Drugi jest dołączony bezpośrednio do złącza krawędziowego.
\subsubsection{Hyperlink}
Hyperlink jest szybkim interfejsem szeregowym obsługiwanym przez procesory DSP firmy \textit{Texas Instruments}. Przepustowość tego interfejsu to aż 12.5 Gbps na linię. Zgodnie ze specyfikacją \cite{DSP:HDG} interfejs ten ma sprzężenie DC, jednak według informacji umieszczonej na oficjalnym forum \textit{Texas Instruments}  \url{http://e2e.ti.com/} \cite{HYPERLINK_AC} interfejs może również pracować ze sprzężeniem AC. Hyperlink w akceleratorze obliczeniowym służy do przesyłu danych pomiędzy procesorami, maksymalna przepustowość to 50 Gbps (4 x 12.5 Gbps). Przydatną właściwością interfejsu jest możliwość dowolnego zamieniania linii oraz par różnicowych w celu ułatwienia prowadzenia połączeń na PCB. 
Poza liniami przesyłającymi dane, interfejs zawiera dodatkowe linie dedykowane specjalnemu protokołowi komunikacyjnemu. 

\subsubsection{Pozostałe interfejsy}
Procesor posiada ponadto interfejsy TSIP, AIF, I2C, SPI, oraz liczniki. TSIP oraz AIF nie są wykorzystane w projekcie. Wyprowadzenia pozostałych interfejsów zostały dołączone bezpośrednio do układu FPGA.
\subsubsection{SPI}
DSP komunikuje się z pamięcią NOR oraz FPGA za pomocą interfejsu SPI. Sygnał zegara jest rozdzielony za pomocą bufora SN74AUC2G07 \cite{SN74AUC2G07}. Sygnał zegarowy do układu FPGA jest dołączony do wejścia MRCC, które jest dedykowanym wejściem zegarowym \cite{FPGA:UG472}. Schemat połączeń pozwala na uruchomienie DSP z pamięci NOR (\textit{Second Level Bootloader}) oraz na komunikację z FPGA poprzez interfejs SPI. 

%------------------------------------------------------------------------------------------------------------------------------------------------------------------%
\subsubsection{Sygnały zegarowe}
Procesor DSP TMS320C6678 potrzebuje do pracy, w zależności od wykorzystywanych peryferiów, 6 sygnałów zegarowych.
\begin{table}[h]
\ra{1.3}
\centering
	\caption{Sygnały zegarowe procesora DSP}
    \begin{tabular}{p{3cm}  p{7.5cm}  c}
	\toprule
    \textbf{Sygnał} & \textbf{Opis} & \textbf{Czestotliwość [MHz]}\\
    \midrule
    CORECLK & 		sygnał zegarowy rdzenia procesora & 											100\\
    DDRCLK & 		sygnał zegarowy kontrolera pamięci DDR3 & 										66.67 \\
    SRIOSGMIICLK & 	sygnał zegarowy kontrolera interfejsów SGMII oraz SRIO & 							312.5\\
    PCIECLK & 		sygnał zegarowy dla kontrolera magistrali PCI Express Gen 2 & 							100\\
    MCMCLK & 		sygnał zegarowy magistrali Hyperlink &											312.5\\
    PASSCLK & 		sygnał zegarowy dla koprocesora sieciowego & 									100\\
	\toprule
    \end{tabular}

	\label{tbl:dsp_clocks}
\end{table}

Wszystkie sygnały zegarowe są typu LVDS o sprzężeniu pojemnościowym. Możliwe jest też dołączenie zegarów HCSL przy zastosowaniu odpowiedniej terminacji. Sygnały zegarowe są generowane przez system dystrybucji zegara, a dokładniej przez układy CDCM6208 \cite{CDCM6208} i 5V41068A \cite{5V41068A}. Dokładny opis wejść zegarowych procesora DSP można znaleźć w dokumentacji producenta \cite{DSP:CLOCK}.
%------------------------------------------------------------------------------------------------------------------------------------------------------------------%
\subsubsection{Zasilanie}
Do pracy procesor DSP TMS320C6678 wymaga 4 podstawowych napięć zasilających oraz dodatkowych wymagających filtracji.

\begin{table}[h]
\ra{1.3}
\centering
	\caption{Napięcia zasilające i pobór prądu przez procesor DSP}
    \begin{tabular}{p{3cm} p{7cm} c c}
	\toprule
    \textbf{Linia} & \textbf{Opis} & \textbf{Napięcie [V]} & \textbf{Prąd [A]}\\
    \midrule
    CVDD & 		core logic adjustable supply & 			(0.9 - 1.05)	&	7.3\\
    CVDD1 & 		fixed internal supply & 				1.0		&	1.5\\
    VDDTn& 		filtered SerDes termination voltage & 		1.0		&	BD\\
    DVDD18 & 	LVCMOS buffers and PLL supply & 		1.8		&	0.021\\
    AVDDAn & 	filtered PLL supply &				1.8		&	BD\\
    DVDD15& 		DDR3 buffers supply & 				1.5		&	0.414\\
    VDDRn & 		filtered SerDes supply & 				1.5		&	BD\\
    VTT & 		DDR3 termination supply &			0.75		&	BD\\
    VREF & 		DDR3 reference supply &				0.75		&	BD\\
	\toprule
    \end{tabular}

	\label{tbl:dsp_voltages}
\end{table}


\begin{figure}[!ht]
\begin{center}
\includegraphics[width=10cm]{grafika/dsp_power_hdg.png}
\end{center}
\caption{Schemat napięć zasilających procesora DSP}
\label{DSP_POWER}
\end{figure}


\subsubsection{Smartreflex}   
Napięcie rdzenia procesora DSP wykorzystuje specjalny interfejs \textbf{SmartReflex} typu C, który reguluje napięcie rdzenia po wyjściu procesora ze stanu resetu. W przypadku procesorów TMS320C6678 napięcie rdzenia jest uzależnione od procesu produkcji i może zawierać się w przedziale od 0.9 V do 1.05 V. Stan pinów VID[0:3] określa napięcie rdzenia. Do obsługi tego interfejsu zostały wyprodukowane dedykowane układy kontrolerów przetwornic oraz samych tranzystorów przełączających z serii UCD92xx oraz UCD72xx. W akceleratorze zastosowano te same układy jak w module ewaluacyjnym z tą różnicą iż oba wyjścia UCD7242 \cite{UCD7242} generują napięcie regulowane rdzenia (\textit{adjustable core voltage supply}) do każdego z procesorów (w module ewaluacyjnym UCD7242 generuje napięcie CVDD i CVDD1). Ponieważ układ UCD9222 \cite{UCD9222} pracuje z napięciem zasilania 3.3 V, a procesor DSP 1.8 V wymagany jest translator poziomów. Zastosowano układ firmy \textit{Texas Instruments} SN74AVC4T \cite{SN74AVC4T}. CVDD1 w projekcie akceleratora obliczeniowego odpowiada linii P1V0.
\subsubsection{Pamieci flash procesora DSP}
Każdy z procesorów DSP posiada dostęp do trzech układów pamięci nieulotnej FLASH: EEPROM, NOR i NAND. Dostęp do pamięci EEPROM jest zapewniony poprzez interfejs I2C, do NOR poprzez SPI, a NAND wykorzystuje specjalny interfejs komunikacyjny EMIF16. Konfiguracja pamięci została zaadaptowana z modułu ewaluacyjnego TMS320C6678. Dodatkowo piny \textbf{WRITE\_PROTECT} są dołączone do układu FPGA w celu zabezpieczenia zmiany pamięci nieulotnej. Taka konfiguracja pozwala na uruchamianie procesora w różnych trybach pracy, jak również na uruchomienie systemu operacyjnego Linux z pamięci NANDa \cite{DSP:BOOT}.
\subsubsection{Pamięć SPI NOR FLASH}
Pamięcią NOR jest układ N25Q128 \cite{N25Q128} o pojemności 16 MB. Układ komunikuje się z procesorem za pomocą interfejsu SPI.
\subsubsection{Pamięć EEPROM}
Zastosowana pamięć EEPROM jest to układ M24M01 \cite{M24M01} firmy STMicroelectronics \cite{STM}. Układ wykorzystuje interfejs I2C do komunikacji z procesorem DSP.
\subsubsection{Pamięć NAND FLASH }
Zastosowano pamięć NAND FLASH MT29F2G16 \cite{MT29F2G16} o pojemności 64 MB. Różni się ona od tej zastosowanej w module ewaluacyjnym szerokością szyny danych (16 linii zamiast 8). Taka zmiana będzie wymagała modyfikacji kodu uruchamiającego procesor z EMIF16 udostępnionego przez producenta. Powodem zmiany był brak kompatybilnej pamięci o 8 liniach danych na rynku. 
%------------------------------------------------------------------------------------------------------------------------------------------------------------------%
\subsection{Pamięć SDRAM }
Procesor DSP ma możliwość zapisu danych na szybkiej pamięci dynamicznej SDRAM-1333 o częstotliwości pracy 667 MHz. Szyna adresowa ma 16 bitów, natomiast szyna danych 64 bity. Istnieje możliwość dołączenia piątej kości w celu dodania funkcji ECC. 
%------------------------------------------------------------------------------------------------------------------------------------------------------------------%
\subsection{Interfejs Gigabit Ethernet}
\subsubsection{PHY Vitesse VSC8221}
Układ Vitesse VSC8221 \cite{VSC8221} jest tzw. PHY (\textit{ang. OSI PHYsical layer}) interfejsu Gigabit Ethernet. Układ łączy warstwę MAC procesora DSP ze światem zewnętrznym poprzez złącze RJ45. Komunikacja między procesorem DSP a układem realizującym warstwę fizyczną odbywa się przez interfejs SGMII (\textit{Serial Gigabit Media Independent Interface}) wraz z dodatkowymi informacjami przesyłanymi liniami \textbf{MDI, MDO}. 

Zdecydowano się na ten układ ze względu na kompaktową obudowę i mały pobór mocy (\textless 700 mW). Dodatkowym czynnikiem była dostępność układu wykorzystanego w module ewaluacyjnym procesora TMS320C6678,  88E1111 \cite{88E1111} firmy Marvell \cite{COMPANY:MARVELL}, którego zamówienie w małej ilości jest problematyczne. 

Układ jest zasilany z napięcia 3.3 V i posiada wewnętrzne stabilizatory generujące napięcie 1.2 V eliminując tym samym konieczność podłączenia kolejnego obciążenia do linii 1.2 V akceleratora. Istnieje możliwość dołączenia nieulotnej pamięci EEPROM, którą przewidziano w projekcie jako opcjonalną. Układ jest skonfigurowany do pracy w trybie SGMII bez sygnału zegara referencyjnego; tryb jest ustalany za pomocą pinów \textbf{CMODE[0:3]}. Dodatkowo w celu kontroli pracy układu piny \textbf{MODEDEF0}, \textbf{SIGDET} i \textbf{SRESETz} zostały dołączone do mikrokontrolera LPC1764. 

 Pin \textbf{MODEDEF0} jest pinem wyjściowym sygnalizującym poprawną inicjalizację układu, \textbf{SIGED} jest sygnałem wyjściowym indykującym stan transmisji. \textbf{SRESET} natomiast resetuje PHY z zachowaniem konfiguracji rejestrów wewnętrznych. 

%##################################################################################
\section{Moduł przełączania interfejsów}
%------------------------------------------------------------------------------------------------------------------------------------------------------------------%
\subsection{Układ przełącznika interfejsu \textit{Serial Rapid IO} }
Istnieje możliwość modyfikacji połączeń pomiędzy jednostkami przetwarzania, akceleracji sprzętowej i \textit{Backplane} dzięki zastosowaniu w akceleratorze przełącznika gigabitowego (\textit{gigabit crosspoint switch}), układu ADN4604 \cite{ADN4604}. Przełącznik posiada 16 wejść i wyjść różnicowych, pomiędzy którymi można się przełączać na zasadzie każdy z każdym oraz jeden do wszystkich. Układ jest zasilany z napięcia 3.3 V, przełączanie jest sterowane za pomocą mikrokontrolera LPC1764 poprzez interfejs I2C. 

Procesory DSP sa dołączone do ADN4604 za pomoca interfejsu SRIO (4 linie), natomiast FPGA wykorzystuje 4 linie GTP, które również obsługują interfejs SRIO. Do przełącznika jest ponadto dołączona magistrala Fat Pipe 2 ze złącza AMC. Fat Pipe jest grupą portów na złączu AMC służąca połączeniu wielo-liniowych interfejsów, takich jak np. PCIe czy SRIO [patrz \ref{AMC_CON}]. Na złączu AMC znajdują się dwie grupy Fat Pipe po cztery linie Tx/Rx. Taka sieć połączeń pozwala na elastyczną komunikację o dużej przepustowości miedzy układami. 

Przykładowym scenariuszem pracy może być sytuacja kiedy FPGA przetwarza wstępnie dane otrzymane z interfejsu SAS i przesyła je do jednej jednostki przetwarzania za pomocą wszystkich linii, gdyż wymagane przetwarzanie potrzebuje dużej przepustowości (4 linie SRIO), z drugiej strony może zdarzyć się sytuacja kiedy przepustowość nie będzie istotna a moc obliczeniowa będzie kluczowym parametrem. Wtedy jednostka akwizycji ma możliwość przesłania danych do obu procesorów DSP, jak również jednocześnie do DSP i płyty matki. Zastosowanie tego układu zwiększa elastyczność i uniwersalność akceleratora.


%------------------------------------------------------------------------------------------------------------------------------------------------------------------%
\subsection{Układ przełącznika interfejsu \textit{PCI Express 2.0}}

Inteligentny przełącznik PCI Express (\textit{PCIe Switch}) to układ PEX8616 \cite{PEX8616}, który posiada 4 porty \textit{PCIe Gen 2.0}, co pozwala na dołączenie do niego obu procesorów DSP, układu FPGA i złącza AMC.  Układ TMS320C6678 przesyła dane do przełącznika za pomocą dwóch linii, FPGA za pomocą 4 portów GTP, które mogą być skonfigurowane do pracy w interfejsie \textit{PCIe}. Czwarty port jest podłączony do złącza AMC, do portu Fat Pipe 1. Na złączu krawędziowym znajdują się cztery linie interfejsu \textit{PCIe}. Układ pozwala na transparentną transmisję danych między portami.


Schematy przełącznika zostały zaprojektowane w oparciu o referencyjny moduł ewaluacyjny \textbf{PEX8616 RDK} oraz o projekt OHWR \textit{Beam Position Monitor, Digital Back End} \cite{BPMDBM} gdzie został wykorzystany podobny układ jednak o większej ilości portów.

Porty są skonfigurowane jako 4 liniowe, za pomocą rezystorów podciągających dołączonych do pinów \textbf{STRAP\_STN0\_PORTCFG1}, \textbf{STRAP\_STN1\_PORTCFG0}. Mimo tego że DSP są dołączone za pomocą dwóch linii do przełącznika, układ PEX8616 dzięki funkcji autonegocjacji sam zmniejszy ilość pracujących linii. Funkcja ta przydatna jest również podczas prowadzenia połączeń na PCB gdyż kolejność linii oraz polaryzacja par różnicowych może być dowolna. 

Układy DSP, FPGA oraz złącze AMC są węzłami typu \textit{End Point}, a PEX8616 jest węzłem typu \textit{Root Complex}. Do układu jest doprowadzony referencyjny sygnał zegarowy o częstotliwości $f_{clk}=100MHz$ z układu CDCUN1208LP. 

Istnieje możliwość konfiguracji wewnętrznych rejestrów układu poprzez interfejs I2C (z FPGA) oraz zapis konfiguracji w opcjonalnie montowanej pamięci nieulotnej. Zaawansowane funkcje układu są dostępne jedynie poprzez EEPROM. Układ będzie spełniał swoją funkcję bez konfiguracji jednak warto mieć możliwość rozszerzenia funkcjonalności.

Układ, ponadto, pozwala na obsługę funkcji \textit{Hot Plug} pozwalającej na dołączanie oraz odłączanie układów do niego dołączonych podczas pracy. Ta funkcja może być szczególnie przydatna, gdy wystąpi konieczność aktualizacji oprogramowania, nie będzie wtedy konieczny reset systemu operacyjnego kontrolera karty.


%##################################################################################
\section{System dystrybucji sygnałów zegarowych}
%%------------------------------------------------------------------------------------------------------------------------------------------------------------------%
% 
Dystrybucja i generacja sygnałów zegarowych na karcie jest wykonana za pomocą szeregu układów dedykowanych takim zastosowaniom. Sygnały zegarowe są krytyczne dla poprawnej pracy układów scalonych, dlatego zastosowano sprawdzone i wcześniej wykorzystywane w innych projektach układy aby zapewnić poprawność działania systemu. Zamieszczony diagram przedstawia system dystrybucji sygnałów zegarowych na akceleratorze obliczeniowym. 
 \begin{figure}[here]
\begin{center}
\includegraphics[width=12cm]{grafika/clock_distribution_detail.png}
\caption{Szczegółowy schemat blokowy systemu dystrybucji sygnałów zegarowych}
\end{center}
\end{figure}

Sygnały zegarowe można podzielić na te dedykowane interfejsowi PCIe, układowi FPGA i procesorom DSP.  

\subsection{Generacja sygnałów zegarowych interfejsu \textit{PCI Express 2.0}} 

Sygnały zegarowe interfejsu PCIe są generowane z jednego źródła, aby umożliwić pracę układów w trybie \textit{common refclk} zsynchronizowanego z sygnałem zegarowym MCH. Standard PCIe specyfikuje również sposób dystrybucji sygnału zegarowego \textit{separate refclk} oraz \textit{data clocked refclk}. Pierwszy występuje wtedy, kiedy układy posiadają lokalne źródło sygnału zegara PCIe; korzysta się w tym przypadku z faktu, że standard PCIe specyfikuje iż przesunięcie między sygnałami zegarowymi może zawierać się w przedziale $+/- 300 ppm$. Minusem tego rozwiązania jest brak możliwości korzystania z \textit{SSC} (\textit{Spread Spectrum Clocking}). Ostatni tryb, jak sama nazwa wskazuje, zaszywa zegar w przesyłanych danych. Częstotliwość sygnału zegarowego PCIe to $100 MHz$.

SSC jest to metoda zmniejszania generowanych zakłóceń elektromagnetycznych z linii zegarowej (EMI) oraz redukcji wpływu szumów na linię poprzez modulację sygnału zegara wokół częstotliwości nośnej. Częstotliwość modulacji jest niska (33 kHz).

 \begin{figure}[here]
\begin{center}
\includegraphics[width=12cm]{grafika/pcie_clk_distr.png}
\caption{Rodzaje dystrybucji sygnału zegarowego interfejsu \textit{PCI Express 2.0}}
\end{center}
\end{figure}

Warunkiem poprawnej pracy systemu dystrybucji \textit{common refclk} jest dopasowanie sygnałów zegarowych do 12 ns przesunięcia (\textit{skew}) \cite{PCIE_REF_CLK} między liniami. W przypadku modułu AMC trudno jest nie spełnić tego wymogu, ze względu na małe wymiary mechaniczne.

Dedykowanym wyjściem zegarowym dla interfejsu PCIe na złączu AMC jest wyjście FCLK \cite{AMC_BASE}. Zegar ten (typu HCSL lub LVDS \cite{AMC_BASE}) dołączony jest do bufora 1:8 układu CDCUN1208, który powiela sygnał zegarowy do wszystkich układów obsługujących interfejs PCIe. Do układu FPGA i przełącznika PCIe zegary są dołączone bezpośrednio, natomiast do procesorów DSP poprzez przełącznik (multiplekser) zegarowy 2:1 interfejsu PCIe układ 5V41068A sterowany za pomocą mikrokontrolera LPC1764. Pozostałe zegary zostały wyprowadzone na panel przedni modułu AMC do złącz HDMI i służą jako wyjście nieużywanych sygnałów zegarowych.  Sygnał zegarowy dołączony do przełącznika PCIe jest typu HCSL i dlatego należało zastosować odpowiednią terminację.

Układ CDCUN1208, jest konfigurowany za pomocą pinów wejściowych:
\begin{itemize}
\item
\textbf{OE} - uruchomienie wyjść, stan wysoki, wyjścia uruchomione
\item
\textbf{MODE} - tryb programowania, stan open drain, tryb pracy układu \textit{pin programming mode}
\item
\textbf{DIVIDE} - dzielnik częstotliwości na wyjściu, stan open drain, dzielnik $= 1$
\item
\textbf{ERC} - szybkość narastania zboczy sygnałów wyjściowych, stan open drain, tryb \textit{FAST}
\item
\textbf{ITTP} - rodzaj wejścia zegara referencyjnego, stan wysoki, wejście \textbf{HCSL}
\item
\textbf{INSEL} - wybór wejścia referencyjnego, stan niski, wejście \textbf{IN1} aktywne
\item
\textbf{OTTP} - tryb pracy wyjść, stan niski, wyjścia typu \textbf{LVDS}
\end{itemize}

\subsection{Generacja sygnałów zegarowych dla układu FPGA}
Złącze AMC poza wyjściem zegarowym FCLK posiada cztery wejścia/wyjścia zegarowe TCLK[0:3]. Zgodnie ze specyfikacją standardu AMC \cite{AMC_BASE} są to sygnały zegarowe o niskiej częstotliwości, które mogą służyć jako zegary referencyjne bądź \textit{wyjściowe} tj. generowane na module i przesyłane do \textit{MCH}. Oryginalnym zastosowaniem tych sygnałów jest synchronizacja w systemach telekomunikacyjnych. W przypadku akceleratora obliczeniowego zegary TCLK[0:3] uznane zostały za wyjściowe i dołączone do multiplexera 4:1 układu SY89544U \cite{DATASHEET:SY89544U} firmy \textit{Micrel Inc.} \cite{COMPANY:MICREL}, którego wyjście zostało dołączone do jednego z wejść referencyjnych układu AD9522 \cite{AD9522}. Wyjście układu SY89544U jest sterowane za pomocą mikrokontrolera.
 
 Układ AD9522 generuje sygnały zegarowe wymagane przez układ FPGA. Dodatkowo aby uniezależnić się od parametrów zegarów TCLK, do układu dołączono oscylator TCXO 10 MHz. Konieczne było również dodanie filtru pętli PLL. Filtr zaprojektowano za pomocą oprogramowania udostępnionego przez producenta ADIsimCLK. Ustalono maksymalną częstotliwość wyjściową na 600 MHz (\textit{maximum bandwidth}), taka jest maksymalna częstotliwość pracy wejść zegarowych transceiver'ów GTP układu XC7A200T.  
 

 \begin{figure}[here]
\begin{center}
\includegraphics[width=12cm]{grafika/adisimclk.png}
\caption{Projekt pętli sprzężenia zwrotnego PLL układu AD9522}
\end{center}
\end{figure}
  
 
 Sygnały zegarowe generowane przez układ to:
 \begin{itemize}
\item
6 zegarów do transceiverów GTP układu XC7A200T
\item
2 zegary kontrolerów pamięci SDRAM FPGA
\item
4 zegary wyjściowe dołączone do gniazd HDMI
\end{itemize}
 
  Wyjścia/wejścia kontrolne \textbf{STATUS}, \textbf{SYNC} i \textbf{RESET}, zostały dołączone do układu LPC1764.

\subsection{Generacja sygnałów zegarowych procesora DSP}
Procesor DSP wymaga do poprawnej pracy 6 sygnałów zegarowych [patrz \ref{tbl:dsp_clocks}]; do ich generacji w module zastosowano układy CDCM6208 oraz 5V41068A. 


\subsubsection{Dobór układu generującego sygnały zegarowe}

Dedykowanymi układami dystrybucji zegara dla procesorów DSP z rodziny TMS320C66x są układy CDCE6205, CDCL6010 i CDCM6208. Zdecydowano się na wykorzystanie układu CDCM6208 gdyż pozwala on na generację wszystkich potrzebnych sygnałów zegarowych do DSP.  Dla porównania, moduł ewaluacyjny TMDXEVM6678L \cite{TMDXEVM6678L} posiada system generacji sygnałów zegarowych zbudowany na dwóch układach CDCE62005 oraz zewnętrznym multiplekserze sygnałów zegarowych PCIe 2:1 ICS557. Wykorzystanie układu CDCM6208 pozwala uprościć ten system. Generuje on 6 sygnałów zegarowych, z czego 5 doprowadzonych jest bezpośrednio do procesora DSP, a jeden PCIECLK do multipleksera sygnału zegarowego interfejsu PCIe układu 5V41068A (dedykowanego PCIe Gen. 2.0). 

\subsubsection{Schemat elektryczny układu dystrybucji sygnałów zegarowych CDCM6208}

Układ jest zasilany z napięcia 1.8 V. Źródłem referencyjnym sygnału zegara jest oscylator kwarcowy 25 MHz dołączony do wejścia \textbf{PRI\_REFP/N} (wejście różnicowe),  wybierany pinem \textbf{REF\_SEL} (stan niski). Pin \textbf{SYNCN} jest ustawiony w stan wysoki, aby wyjścia były aktywne. Funkcja wyłączenia układu nie jest wykorzystywana, dlatego \textbf{PDN} jest ustawiony w stan wysoki. Zgodnie z zaleceniami producenta dołączono kondensator opóźniający uruchomienie układu (aby wewnętrzne PLL ustabilizowało się).  Do pinu resetu układu dołączono układ opóźniający RC, reset układów zegarowych jest \textbf{oddzielony} od resetu pozostałych układów.  

Konfiguracja odbywa się poprzez interfejs I2C. Ponieważ w akceleratorze znajdują się dwa układy generacji zegara dla procesorów DSP, różne są ustawienia pinów \textbf{AD[0:1]} które ustalają adres urządzenia.


\begin{itemize}
\item
adres CDCM6208 DSP0 0x54 
\item
adres CDCM6208 DSP1 0x55
\end{itemize}



Producent udostępnia specjalne oprogramowanie, dzięki któremu można wygenerować odpowiednie wartości wewnętrznych rejestrów oraz elementów pętli sprzężenia zwrotnego. Układ został skonfigurowany w trybie \textit{Synthesiser Mode} i pozwala na generację zegarów typu LVDS o częstotliwościach wymaganych przez procesor DSP. 
\begin{figure}[!h]
\centering
\includegraphics[width=10cm]{grafika/cdcm6208v2.jpg}
\caption{Konfiguracja częstotliwości i typów wyjść zegarowych układu CDCM6208}
\end{figure}

Pętla sprzężenia zwrotnego układu CDCM6208 jest przedstawiona na rysunku [\ref{fig:cdcm6208_loop}].

\begin{figure}[!h]
\centering
\includegraphics[width=10cm]{grafika/cdcm6208v2_loop_filter.jpg}
\caption{Konfiguracja filtru pętli sprzężenia zwrotnego układu CDCM6208}
\label{fig:cdcm6208_loop}
\end{figure}

Projekt pętli został wykonany w oparciu o dokumentację producenta  \cite{CDCM6208:INFO1} \cite{CDCM6208:INFO2}. 

\paragraph{Multiplekser sygnału zegarowego PCIe Gen. 2.0 5V41068A}


Układ 5V41068A pełni taką samą rolę jak ICS557 w module ewaluacyjnym procesora TMS320C6678. Zasilany jest z napięcia 3.3 V odseparowanego od linii \textbf{P3V3} filtrem typu CLC.  Zdecydowano się na zmianę układu gdyż ten jest dedykowany interfejsowi \textit{PCIe Gen. 2.0} a taki obsługuje procesor DSP. Układ pełni rolę przełącznika sygnału zegarowego interfejsu PCIe do DSP sterowanego z mikrokontrolera LPC1764 poprzez piny \textbf{PD}, \textbf{OE}, \textbf{SEL}. Takie rozwiązanie jest konieczne, aby zapewnić synchronizację we wcześniej wspomnianym systemie dystrybucji zegara \textit{common refclk}. %Terminacje zostały zaadaptowane z modułu ewaluacyjnego.


%~\\*
%
%Wszystkie wyjscia ukladu sa typu LVDS o sprzezeniu AC.
%------------------------------------------------------------------------------------------------------------------------------------------------------------------%

%%------------------------------------------------------------------------------------------------------------------------------------------------------------------%
\section{IPMI i zarządzanie peryferiami}
Moduł akceleratora obliczeniowego jest wyposażony w mikrokontroler LPC1764, którego zadaniem jest obsługa standardu IPMI \cite{IPMI} \cite{LPC1764}, sterowanie wejściami i interfejsami konfiguracyjnymi układów. Jest to mikrokontroler firmy NXP w obudowie LQFP100 z rdzeniem Cortex-M3. Układ jest zasilany z linii P3V3\_MP,  dedykowanej tylko IPMI zgodnie ze standardem AMC \cite[4.22]{AMC_BASE}. Do układu dołączona jest bezpośrednio pamięć EEPROM - układ AT24MAC602 \cite{AT24MAC602} firmy Atmel \cite{ATMEL} oraz układ RTC MCP79410 \cite{MCP79410} firmy Microchip.

%\subsection{IPMI}
%IPMI \textit{Intelligent Platform Management Interface} jest interfejsem zarządzającym urządzeniami w systemach telekomunikacyjnych. Standard MTCA wymaga by karta AMC wpinana w kratę obsługiwała ten protokół komunikacyjny, w celu np. przesyłu informacji na temat zapotrzebowania na pobieraną moc (\textit{payload}). Obsługa tego standardu została napisana w języku C przez zespół PERG. 

\subsection{I2C - konfiguracja układów}
Jedną z funkcji mikrokontrolera jest obsługa interfejsów I2C. LPC1764 zawiera 4 linie I2C pełniące następujące funkcje:
\begin{itemize}
\item
komunikacja z \textit{MCH}
\item
obsługa czujników temperatury i RTC
\item
konfiguracja układów CDCM6208 (1.8V)
\item
konfiguracja układów ADN4604, CDCUN1208, AD9522
\end{itemize}


\begin{figure}[!ht]
\centering
\includegraphics[width=10cm]{grafika/lpc_i2c.png}
\caption{Schemat blokowy połączeń interfejsu I2C}
\label{I2C_BLOCK}
\end{figure}

Linia konfigurująca układy zegarowe CDCM6208 jest dołączona do nich przez translator poziomów układ PCA9306DCTR \cite{PCA9306DCTR} gdyż są one zasilane z napięcia 1.8 V. Dodatkowo dwie linie interfejsów zostały wyprowadzone na panel przedni do złącz HDMI w celach diagnostycznych. 

\subsection{Zarządzanie peryferiami}
Moduł akceleratora obliczeniowego jest wyposażony w mikrokontroler, który zajmuje się uruchomieniem całego urządzenia i zarządzaniem innymi układami. Do funkcji tego układu należy: 
\begin{itemize} 
\item konfiguracją rejestrów wewnętrznych CDCM6208 
\item sterowanie wejściami konfiguracyjnymi układów VSC8221, SY8544U i AD95222 
\end{itemize}

\begin{figure}[!ht]
\centering
\includegraphics[width=10cm]{grafika/lpc_control.png}
\caption{Schemat blokowy połączeń sygnałów sterujących układu LPC1764}
\end{figure}

Pozostałymi funkcjami jakie pełni układ LPC1764 jest obsługa przycisku RESETu znajdującego się na panelu przednim akceleratora. Zarządzaniem ładowaniem pamięci programu układu FPGA z pamięci FLASH poprzez interfejs SPI, zgodnie ze scenariuszem \textit{Slave-Serial}. Ponadto układ steruje sygnałami \textbf{GA[0:2]} ustalającymi adres I2C modułu w skrzyni MTCA. 

\begin{figure}[!ht]
\centering
\includegraphics[width=10cm]{grafika/lpc_control2.png}
\caption{Schemat blokowy połączeń pozostałych sygnałów sterujących układu LPC1764}
\end{figure}
%
%\subsection{Obsługa resetu}
%Mikrokontroler LPC1764 zajmuje się obsługą resetu całego systemu. Po naciśnięciu przycisku na panelu przednim wyłącza wszystkie linie zasilające w odpowiedniej kolejności. 

\section{Złącza wejść/wyjść}

\subsection{miniSAS}
Dane do urządzenia są przekazywane za pomocą dwóch złącz miniSAS firmy MOLEX \cite{MINISAS} umieszczonych na panelu przednim. Każde ze złącz zawiera 4 linie RX/TX i interfejs komunikacyjny. Głównym zastosowaniem złącz SAS są centra danych. W porównaniu do złącz SATA, standard SAS jest wytrzymalszy mechanicznie i pozwala na łączenie urządzeń dłuższym przewodem. Maksymalna przepustowość danych w tym złączu to 4 x 12.0 Gbps czyli sumarycznie 48 Gbps.

 \begin{figure}[here]
\begin{center}
\includegraphics[width=5cm]{grafika/sas.jpg}
\end{center}
\caption{Złącze miniSAS firmy MOLEX}
\label{SAS}
\end{figure}


\subsection{RJ45}
Gniazdo RJ45 \textit{SI-61001-F} z wbudowaną izolacją magnetyczną dołączone jest do PHY VSC8221. Schemat połączeń i terminacji został zaadaptowany z projektu modułu ewaluacyjnego procesora TMS320C6678.  


\subsection{HDMI}
Niewykorzystane wyjścia zegarowe w układach AD9522 i CDCUN1208 wyprowadzono na panel przedni z wykorzystaniem złącz HDMI typu D. Dodatkowo do złącz doprowadzone są interfejsy I2C i sygnał OVERTEMP. Sygnały zostały zabezpieczone przed ESD/EMI poprzez dołączenie dedykowanych układów TPD12S016 \cite{TPD12S016}.

\begin{figure}[!ht]
\centering
\includegraphics[width=10cm]{grafika/hdmi.png}
\caption{Schemat blokowy połączeń złącz HDMI}
\end{figure}



%
%Vbatt
%Wewnetrzna pamiec nieulotna jest dodatkowo zasilana z baterii.
%
%Projekt zasilania
%Wsjo z afc i evalboardu.
%
%projekt pcb
%
%W ninejszym rozdziale opisano projekt obwodow drukowanych akceleratora wykonanego w programie altium designer.
%
%W pierwszej czesci opisany zostal standard mechaniczny AMC oraz projekt warstw pcb. Nastepnie opisano sposob rozmieszczenia elementow i zasady prowadzenia połączeń.
%
%Standard amc
%
%Projekt warstw 
%Przy projektowaniu bardzo skomplikowanych urzadzen elektronicznych zawierajacych duze uklady w obudowie bga konieczne jest zastosowanie wielowarstwowego obwodu drukowanego.
%
%Projekt
%%------------------------------------------------------------------------------------------------------------------------------------------------------------------%
\section{MLVDS}
Złącze AMC zawiera interfejs komunikacyjny MLVDS służacy do dystrybucji sygnałów zegarowych, jak również \textit{triggerów} i \textit{interlocków}. Linie różnicowe MLVDS wychodzące ze złącza AMC są dołączone do układów tansceiverów MLVDS \textit{SN65MLVD040} \cite{SN65MLVD040} których wyjścia są dołączone do FPGA. 
%%------------------------------------------------------------------------------------------------------------------------------------------------------------------%
\section{JTAG}
Protokół JTAG układu FPGA jest doprowadzony do złącza AMC oraz równolegle do złącza goldpin w celu ułatwienia uruchomienia urządzenia. 

Protokoły JTAG procesorów DSP są połączone w łańcuch tzw. \textit{Daisy Chain} zgodnie z zaleceniami producenta \cite{DSP:HDG} \cite{WIKI:TI_XDS}. Dodano również bufor SN74ALVC125PW \cite{SN74ALVC125} na poszczególne sygnały aby zapewnić poprawną pracę protokołu. Procesor DSP TMS320C6678 poza standardowym protokołem JTAG wspiera również dodatkowe protokoły \textit{HS\_RTDX} oraz \textit{Trace} które pozwalają na szybsze programowanie i dokładniejszą analizę oprogramowania i pracy procesora. Złącze programatora w module ewaluacyjnym ma aż 60 sygnałów. W przypadku projektu akceleratora obliczeniowego nie potrzebujemy aż tak dokładnego protokołu programowania dlatego zdecydowano się na złącze 20 pinowe, które dodatkowo zajmuje najmniej miejsca na PCB (w porównaniu do np. złącza 14-pinowego) i wspiera część dodatkowego interfejsu programatora dla procesorów DSP. 

Programowanie mikrokontrolera LPC1764 odbywa się poprzez standardowe złącze JTAG.

\begin{figure}[!ht]
\centering
\includegraphics[width=12cm]{grafika/jtag.png}
\caption{Schemat blokowy połączeń protokołu JTAG}
\end{figure}

\section{Zasilanie}
Sekcja zasilania w module składa się z sześciu przetwornic impulsowych oraz dwóch stabilizatorów LDO (\textit{Low Dropout Regulator}) generujących wszystkie wymagane napięcia. Schemat blokowy [\ref{POWER_BLOCK}] przedstawia system zasilający akceleratora obliczeniowego. 

Przetwornice impulsowe posiadają wysoką sprawność zmiany napięcia wynoszącą typowo ok. 80\%. Dlatego wykorzystane są do generacji napięć wymagających dużego prądu. Wadą przetwornic jest generacja zakłóceń na częstotliwości pracy układu przełączającego, które potrafią zakłócić pracę działania układów.

  LDO wykorzystuje się w sytuacjach gdy pobór prądu przez obciążenie jest dostatecznie mały. Sprawność LDO jest niska, ale dzięki swojej zasadzie działa nie generuje żadnych zakłóceń. LDO zostały wykorzystane w projekcie do generacji napięć referencyjnych i napięcia terminacji pamięci SDRAM. Jednym z wymogów napięcia terminacji pamięci SDRAM jest to aby zmiana napięcia podążała za zmianą napięcia zasilania pamięci tj. 1.5V. Aby spełnić te wymagania zastosowano dedykowane układy (TPS51200), które śledzą napięcie 1.5 V i względem niego ustalają napięcie wyjściowe 0.75 V.  

\begin{figure}[!ht]
\centering
\includegraphics[width=12cm]{grafika/power.png}
\caption{Schemat blokowy sekcji zasilania akceleratora obliczeniowego}
\label{POWER_BLOCK}
\end{figure}

\subsection{Estymacja poboru mocy}
Estymację  poboru mocy przez poszczególne układy wykonano poprzez analizę danych z not katalogowych oraz wykorzystując specjalnie przygotowane przez producentów arkusze kalkulacyjne. Dla układu FPGA jest to dokument \textit{Xilinx Power Estimator}, szczegółowo opisany w dokumentacji \cite{XPE}  \cite{XPE:UG1} \cite{XPE:UG2}. Dla układu DSP  również wykorzystano arkusz kalkulacyjny obliczający pobór mocy \textit{C6678 Power Consumption Model (Rev. C)} \cite{DSP:POWER}. Zestawienie poboru prądu przez poszczególne układy znajduje się w tabeli [\ref{tbl:dsp_power}].

  \begin{sidewaystable}

  \centering
	\caption{Estymacja pobieranej mocy przez akcelerator obliczeniowy}
    \begin{tabular}{| c| c| c | c | c | c | c | c | c}
    \hline
    \textbf{Linia} & \textbf{Układ} & \textbf{Napięcie [V]} & \textbf{Prąd [A]} & \textbf{Ilość}  & \textbf{Sumar. prąd} [A] &   \textbf{Moc [W]} &  \textbf{Komentarz}\\
    \hline
    \hline

    CVDD DSP0  	& 				& 	1 	&	8	& 		& 	7.3 	& 	7.3 	& 	UCD9222+UC7242		\\
         \hline
    			& 	TMS320C6678 	& 		& 	7.3	&	1	& 	7.3 	& 	7.3 	& 	1000 MHz, 50 C		\\
     \hline
    CVDD DSP1  	& 				& 	1 	&	8	& 		& 	7.3 	& 	7.3	& 					\\
         \hline
    		   	& 	TMS320C6678 	& 		& 	7.3 	&	1	& 	7.3 	& 	7.3 	& 	1000 MHz, 50 C 		\\
     \hline
    P1V0		& 				&	1	&	14	&		& 	10.2	& 	10.2	& 	TPS53353DQPT		\\
     \hline
			& 	TMS320C6678	&		& 	1.5 	&	2 	&	3	&	3	&					\\
			& 	XC7A200T 		&		& 	3 	&	1	&	3	&	3	&					\\
			& 	PEX8616		&		& 	4.2 	&	1 	&	4.2	&	4.2	&	WORST CASE, 1.7W TYP.	\\
     \hline
    P1V2		& 				&	1.2	&	3	&		& 	1	& 	1.2	& 	NCP3170ADR2G		\\
     \hline
			& 	XC7A200T 		&		& 	1 	&	1	&	1	&	1.2	&					\\
     \hline
    P1V5		& 				&	1.5	&	11	&		& 	6.11	& 	9.165	& 	TPS53126RGET		\\
     \hline
			& 	TMS320C6678	&		& 	0.4 	&	2 	&	0.8	&	1.2	&					\\
			& 	K4B2G1646		&		& 	0.24 	&	10 	&	2.4	&	3.6	&					\\
			& 	MT4J512M8RA-125	&		& 	0.33 	&	8 	&	2.64	&	3.96	&					\\
			& 	XC7A200T 		&		& 	0.27	&	1 	&	0.27	&	0.405	&					\\
\hline
     VTT		&				&     0.75	&	2	&		&	0.9	&	0.675	&	TPS51200DRCT		\\
     \hline
			&	K4B2G1646		&    		&	0.05	&	10	&	0.5	&	0.375	&					\\
			&	MT4J512M8RA	&    		&	0.05	&	8	&	0.4	&	0.3	&					\\
     \hline
    P1V8		& 				&	1.8	&	3	&		& 	0.88	& 	1.584	& 	NCP3170ADR2G		\\
     \hline
			& 	TMS320C6678	&		& 	0.02 	&	2 	&	0.04	&	0.072	&					\\
			& 	XC7A200T 		&		& 	0.3 	&	1	&	0.3	&	0.54	&					\\
			& 	CDCM6208		&		& 	0.27 	&	2 	&	0.54	&	0.972	&					\\
     \hline
    P2V5		& 				&	2.5	&	3	&		& 	0.61	& 	1.525	& 	NCP3170ADR2G		\\
     \hline
			& 	PEX8616		&		& 	0.5 	&	1 	&	0.5	&	1.25	&					\\
			& 	XC7A200T		&		& 	0.11 	&	1 	&	0.11	&	0.275	&					\\
     \hline
    P3V3		& 				&	3.3	&	3	&		& 	0.85	& 	2.805	& 	TPS53126RGET		\\
     \hline
			& 	XC7A200T		&		& 	0.05 	&	1 	&	0.11	&	0.275	&					\\
			& 	ADN4604		&		& 	0.54	&	1 	&	0.54	&	1.782	&					\\
			& 	CDCUN1208		&		& 	0.2 	&	1 	&	0.2	&	0.66	&					\\
     \hline
     \hline
    Moc sumaryczna		& 			&	12 	&	3.443 &		& 	   	& 	41.316	& 		\\
     \hline
    \end{tabular}

	\label{tbl:dsp_power}
\end{sidewaystable}

\subsection{Linie zasilania P1V0, P1V2, P1V8, P1V5, P3V3, VTT}
Projekty przetwornic linii \textit{P1V0, P1V2, P1V8, P1V5, P3V3, VTT} zostały zaadaptowane z projektu \textbf{AFC} \cite{AFC}. Spełniają one wymagania prądowe do tego projektu. Dodane zostały zworki na wyjściach w celach testowych. 

Przy projekcie linii napięcia terminacji pamięci SDRAM zastosowano dwa stabilizatory LDO zlokalizowane po dwóch stronach PCB, aby uniezależnić się od spadku napięcia wynikającego z przepływu prądu po znacznej odległości w module. 

\subsection{Linia zasilania P2V5}
Napięcie +2.5V jest wymagane dla układu PEX8616 \cite{PEX8616} firmy PLX Technology \cite{COMPANY:PLX}. Projekt przetwornicy został wykonany wykorzystując układ NCP3170 \cite{NCP3170} firmy \textit{ON Semiconductor} \cite{COMPANY:ON}, taki sam jaki został wykorzystany do generacji napięć  1.2V oraz 1.8V. Dobór elementów do otrzymania przetwornicy o odpowiednich parametrach został wykonany przy wykorzystaniu arkusza kalkulacyjnego udostępnionego przez producenta oraz sprawdzony wykorzystując znaną teorię projektowania przetwornic impulsowych typu \textit{Step-Down}. Zgodnie z przewidywanym poborem dla linii \textit{P2V5}, możemy się spodziewać bardzo małego poboru mocy; typowy pobór mocy przez przełącznik PEX8616 wynosi ok. 0.5A. Jednak nie bierzemy tu pod uwagę prądu pobieranego przez bank układu FPGA. Przetwornica jest zaprojektowana na maksymalny pobór prądu 3A zgodnie z tabelą dostępną w dokumentacji układu \cite[str. 19]{NCP3170}. Na schematach został opcjonalnie umieszczony liniowy regulator LDO, który można zamienić z przetwornicą w przyszłych rewizjach projektu jeśli pobór prądu przez układ FPGA dla tej linii również będzie mały.

\subsection{Regulowane napięcie zasilania rdzenia procesora DSP - CVDD}
Projekt sekcji zasilającej napięcia rdzenia procesorów DSP  oparty jest o układy zalecane przez producenta \cite{DSP:HDG} tj. UCD9222 oraz UCD7242 które są układami specjalizowanymi do zasilania procesorów TI z rodziny TMS320C66x. Zgodnie z wymaganiami każdy procesor musi mieć oddzielną linię zasilania CVDD. Projekt zasilania został zaadaptowany z modułu referencyjnego TMDXEVM6678L, z tą różnicą iż zmieniono wyjście CVDD1 na CVDD dla drugiego procesora DSP (CVDD1 to stałe napięcie 1V, a CVDD to napięcie regulowane). Konfiguracja przetwornicy odbywa się za pomocą interfesju PMBus, która może być wykonana zarówno za pomocą układu LPC1764, który został przystosowany do obsługi tego interfejsu, jak i poprzez programator USB-TO-GPIO \cite{GPIO}.




%Projekt PCB
\chapter{Projekt obwodów drukowanych}
\label{chapt:projekt_pcb}
\section{Wstęp}
Poniższy rozdział zawiera opis projektu obwodów drukowanych akceleratora obliczeniowego. Pierwsza część zawiera opis wymagań i ograniczeń producenta, według których ustalone zostały reguły projektowe. Kolejne części zawierają proces projektowania PCB, na co składa się projekt warstw, rozmieszenia elementów oraz opis zasad prowadzenia ścieżek.

\section{Wymagania producenta}
Produkcję obwodów drukowanych zlecono firmie \textit{Brandner PCB} \cite{BRANDNER} z Estonii. Firma ta pozwala na szybką produkcję zaawansowanych projektów PCB. Strona producenta zawiera zestawienie możliwości produkcyjnych \cite{BRANDNER_PRODUCTION}. 


Względem tych zasad ustalone zostały podstawowe reguły projektowe:

\begin{table}[h]
\centering
	\caption{Podstawowe reguły projektowe PCB}
    \begin{tabular}{p{7cm} | p{2cm} |  p{2.5cm}}
	\toprule
    \textbf{Wymaganie} & \textbf{Min [mm]} & \textbf{Maks. [mm]}\\
    \midrule
    Odstęp między ścieżkami 		& 	0.1		&			\\
    Wymiary przelotki (otwór/średnica)	& 	0.2/0.4	&	0.4/0.8	\\
    Odległość między przelotkami 		& 	0.1		&			\\
    Odległość otworów od polygonów 	& 	0.1		&			\\
    Szerokość ścieżek			&	0.08		&	10		\\
    Odległość ścieżek od krawędzi		&	0.2		&			\\
	\toprule
    \end{tabular}

	\label{tbl:pcb_design_rules}
\end{table}

Szczegółowe zasady projektowe zostały wprowadzone do programu \textit{Altium Designer} w zakładce \textit{Design/Rules}.

\section{Oprogramowanie EDA}
Programem wykorzystanym do zaprojektowania obwodów drukowanych jest \textit{Altium Designer} \cite{ALTIUM}. Program ten jest zintegrowanym środowiskiem pozwalającym na tworzenie schematów, obwodów drukowanych, a nawet posiada takie funkcje jak możliwość programowania układów FPGA.  
\section{Projekt warstw}
Ze względu na znaczną ilość linii zasilających oraz ilość układów scalonych projekt akceleratora obliczeniowego wykonano na 16 warstwowym obwodzie drukowanym. Wykorzystano standardowy stackup producenta oparty o materiał FR4 o stałej dielektrycznej $\epsilon = 4.6$. 


\begin{figure}[!ht]
\centering
\includegraphics[width=14cm]{grafika/brandner.pdf}
\caption{Standardowa konfiguracja warstw dla obwodu 16 warstwowego firmy Brandner}
\end{figure}

Dobór ilości warstw jest szeroko opisany w profesjonalnej literaturze dotyczącej projektowania szybkich systemów cyfrowych \cite{BOOK:HIGH_SPEED} \cite{BOOK:BLACK_MAGIC} jak również w poradnikach tzw. \textit{User Guides} udostępnianych przez producentów układów scalonych \cite{UG_STACKUP1} \cite{UG_STACKUP2}. Zdecydowano się na obwód 16 warstwowy z szeregu powodów:

\begin{itemize}
\item
duża ilość szybkich interfejsów szeregowych - nie można prowadzić takich ścieżek pod/nad rozdzielonymi warstwami  (\textit{split planes})
\item
9 głównych linii zasilających
\item
duże upakowanie elementów
\item
rezygnacja z wykorzystania ślepych i zagrzebanych przelotek
\end{itemize}

Zalecaną praktyką jest, aby każda linia zasilania była prowadzona na oddzielnej warstwie. Z punktu widzenia integralności sygnałowej jest to poprawna praktyka, jednak znacznie zwiększa koszty produkcji urządzenia. Dlatego konieczne było rozdzielenie dwóch warstw zasilających, aby poprowadzić wszystkie napięcia zasilające. Ważnym aspektem jest ścieżka powrotna sygnałów, która również może być podatna na zakłócenia. Zastosowana konfiguracja warstw udostępnia każdej warstwie sygnałowej przynajmniej jedną pełną warstwę referencyjną.  

\begin{figure}[!ht]
\centering
\includegraphics[width=12cm]{grafika/stackup_16.png}
\caption{Konfiguracja warstw akceleratora obliczeniowego}
\end{figure}

\section{Rozmieszczenie elementów}
Krytycznym aspektem projektowania urządzeń elektronicznych jest poprawne rozmieszczenie elementów na obwodzie drukowanym. W akceleratorze obliczeniowym zastosowano ogólnie znane praktyki rozmieszczania elementów jak m. in. :
\begin{itemize}
\item
skupienie układów zasilających w jednym miejscu
\item
zapewnienie jak najkrótszej ścieżki powrotnej dla sygnałów
\item
minimalizacja długości połączeń
\end{itemize}

\begin{figure}[!ht]
\centering
\includegraphics[width=12cm]{grafika/pcb_layout.png}
\caption{Podział części analogowej i cyfrowej akceleratora obliczeniowego}
\end{figure}

Główną trudnością podczas projektowania było odpowiednie rozmieszczenie elementów aby zapewnić optymalne połączenia szybkich interfejsów szeregowych między układami. To one głównie decydowały o rozmieszczeniu elementów, dla przykładu: układ przełącznika SRIO ADN4604 umieszczono na drugiej stronie PCB aby możliwe było poprowadzenie ścieżek interfejsu Hyperlink między procesorami DSP. Natomiast przełącznik PCIe umieszczono blisko złącza krawędziowego AMC, ponieważ umieszczenie go pomiędzy procesorami DSP i FPGA spowodowałoby znaczne utrudnienia w prowadzeniu połączeń interfejsu SRIO wcześniej wspomnianego przełącznika. Kolejną kwestią jest dystrybucja sygnałów zegarowych; układy je generujące powinny znajdować się jak najbliżej układu docelowego co starano się wykonać w projekcie. Końcowy efekt jest zadowalający i wszystkie interfejsy spełniają wymagania producenta. 

\section{Zasady prowadzenia ścieżek szybkich interfejsów}

Poniższa część zawiera zasady, którymi kierowano się przy prowadzeniu połączeń szybkich szeregowych interfejsów takich jak: \textit{PCI Express}, \textit{SRIO}, \textit{Hyperlink} oraz sygnałów zegarowych. Sygnały płynące w obwodach drukowanych mają prędkość mniejszą od prędkości światła w przybliżeniu o połowę, dla typowego laminatu FR4. Zgodnie ze wzorem:
$$ v = \frac{c}{\sqrt{\epsilon_{r}}} = \frac{3 \cdot 10^8}{\sqrt{4.6}}  \approx 14{} {cm}/_{ns}  $$

Dokumentacja producentów układów zawiera zasady prowadzenia poszczególnych interfejsów \cite{DSP_SERDES} \cite{DOCS:HIGH_SPEED_DSP} \cite{DOCS:HIGH_SPEED_LAYOUT}. Dopasowanie par różnicowych i linii opiera się o czas propagacji sygnału w linii transmisyjnej jaką jest ścieżka w obwodzie drukowanym przy dużej prędkości transmisji, dlatego określa się je często w \textit{pikosekundach}. 

\begin{table}[h]
\scriptsize
\centering
	\caption{Reguły prowadzenia ścieżek szybkich interfejsów szeregowych}
    \begin{tabular}{c | p{2cm} |  p{2cm} | p{2cm} | p{2cm} | p{2cm} | p{2cm}}
	\toprule
    \textbf{Interfejs} & \textbf{Dopas. P/N min. [ps]} & \textbf{Dopas. P/N min. [mm]} & \textbf{Dopas. P/N maks. [mm]} & \textbf{Dopas. TXn/RXn [ps]} & \textbf{Dopas. TXn/RXn min. [mm]}  & \textbf{Dopas. TXn/RXn maks. [mm]}\\
    \midrule
    SRIO 	& 	5	&	0.7	&	0.9	&	10	& 	1.38	&	1.8\\
    PCIe 	& 	1	&	0.14	&	0.18	&	5	& 	0.7	&	 0.9\\
    Hyperlink 	& 	1	&	0.14	&	0.18	&	100	& 	13	&	18\\
    SGMII 	& 	5	&	0.7	&	0.9	&	5	& 	0.7	&	 0.9\\
	\toprule
    \end{tabular}

	\label{tbl:serdes_rules}
\end{table}

%Zależności długości linii par różnicowych oraz między samymi parami w dokumentacji producentów są opisane w \textit{mils}. Przygotowana została tabela konwertująca te wartości.
%
%
%\begin{table}[h]
%\centering
%    \begin{tabular}{c | c | c | c | c | c }
%	\toprule
%    \textbf{ps} & \textbf{mils [min]} & \textbf{mils [max]} & \textbf{mm [min]} & \textbf{[mm [max]}   \\
%    \midrule
%    1 		& 	5.464		&	7.092	&	0.9	&	10\\
%    5 		& 	27.32		&	35.46	&	0.18	&	5\\		
%    10 		& 	54.64		&	70.92	&	0.18	&	100\\
%    100 	& 	546.4		&	0.7	&	0.9	&	5\\
%	\toprule
%    \end{tabular}
%	\caption{Konwersja ps - mils - mm}
%	\label{tbl:ps}
%\end{table}


\subsection{Dobór szerokości ścieżek}
Kolejnym kluczowym elementem przy prowadzeniu połączeń szybkich szeregowych interfejsów cyfrowych jest impedancja połączeń. Szybkie interfejsy szeregowe wymagają odpowiedniej wartości terminacji przy odbiorniku (która zwykle jest wbudowana w układ). Najbardziej popularną wartością jest  $100\Omega$ lub np. w przypadku pamięci DDR3 $80 \Omega$. Z tego powodu wymagane jest aby \textit{linia transmisyjna} którą jest ścieżka różnicowa miała odpowiednią impedancję charakterystyczną (różnicową \textit{differential} oraz pojedynczą \textit{single ended}). Do wyznaczenia odpowiedniej szerokości ścieżek posłużono się darmowym programem \textit{Saturn PCB Toolkit} \cite{SATURN}.  

\begin{figure}[!ht]
\centering
\includegraphics[width=12cm]{grafika/saturn.png}
\caption{Program Saturn PCB Toolkit}
\end{figure}



Tabela [\ref{tbl:serdes_width}] przedstawia wyliczone wartości szerokości oraz odległości między parami różnicowymi dla interfejsów o impedancji różnicowej $100 \Omega$. Przygotowane zostały cztery tabele, dwie dla impedancji $100 \Omega$ oraz dwie dla impedancji $80 \Omega$, pierwsza przedstawia standardowe wartości szerokości oraz odległości a druga minimalne. Minimalne wartości są szczególnie przydatne, gdy konieczne jest poprowadzenie pary różnicowej np. pod układem w obudowie BGA.



\begin{sidewaystable}[h]
\centering
    \scriptsize
  %  \begin{tabular}{p{2cm}  | p{2.5cm} |  p{2.8cm} | p{2.5cm} | p{2.5cm} | p{2.5cm} | p{2.5cm}  |  p{2.8cm}  | p{2.5cm}}
  %  
	\caption{Szerokości ścieżek szybkich interfejsów szeregowych dla impedancji $100 \Omega$}
 \begin{tabular}{c  | c |  c | c |c | c | c  | c  | c}
\hline
    \textbf{Warstwa} & \textbf{Rodzaj} & \textbf{Grubość miedzi [$\mu m$]} & \textbf{Wys. H1 [mm]} & \textbf{Wys. H2 [mm]} & \textbf{Szer. ścieżki [mm]}  & \textbf{Odległość [mm]}   & \textbf{Impedancja różn. [$\Omega$]}   & \textbf{Impedancje poj. [$\Omega$]}\\

    \hline
    \hline
    TOP 	& 	Edge cpl. ext 		&	18	&	0.11	&	0	& 	0.15	&	0.3	&	98.562	&	51.06	\\
    GND 	& 	Plane		 		&	18	&		&	0	& 		&		&			&		\\	
    L1 		& 	Edge Cpld Int Asym 	&	18	&	0.1	&	0.11	& 	0.08	&	0.16	&	98.536	&	51.8	\\
    POWER 	& 	Plane		 		&	18	&		&	0	& 		&		&			&		\\
    L2		& 	Edge Cpld Int Asym 	&	18	&	0.1	&	0.11	& 	0.08	&	0.16	&	98.536	&	51.8	\\
    POWER 	& 	Plane		 		&	18	&		&	0	& 		&		&			&		\\
    L3		& 	Edge Cpld Int Asym 	&	18	&	0.1	&	0.11	& 	0.08	&	0.16	&	98.536	&	51.8	\\
    POWER 	& 	Plane		 		&	18	&		&	0	& 		&		&			&		\\
    L4		& 	Edge Cpld Int Asym 	&	18	&	0.1	&	0.11	& 	0.08	&	0.16	&	98.536	&	51.8	\\
    POWER 	& 	Plane		 		&	18	&		&	0	& 		&		&			&		\\
    L5 		& 	Edge Cpld Int Asym 	&	18	&	0.1	&	0.11	& 	0.08	&	0.16	&	98.536	&	51.8	\\
    POWER 	& 	Plane		 		&	18	&		&	0	& 		&		&			&		\\
    L6 		& 	Edge Cpld Int Asym 	&	18	&	0.1	&	0.11	& 	0.08	&	0.16	&	98.536	&	51.8	\\
    GND 	& 	Plane		 		&	18	&		&	0	& 		&		&			&		\\
    BTM 	& 	Edge cpl. ext 		&	18	&	0.11	&	0	& 	0.15	&	0.3	&	98.562	&	51.06	\\

    
	\toprule
    \end{tabular}

	\label{tbl:serdes_width}
%\end{sidewaystable}
%
%\begin{sidewaystable}[h]

    \scriptsize
%    \begin{tabular}{p{2cm}  | p{2.5cm} |  p{2.8cm} | p{2.5cm} | p{2.5cm} | p{2.5cm} | p{2.5cm}  |  p{2.8cm}  | p{2.5cm}}
	\caption{Szerokości ścieżek szybkich interfejsów szeregowych dla impedancji $80 \Omega$}
 \begin{tabular}{c  | c |  c | c |c | c | c  | c  | c}
\hline
    \textbf{Warstwa} & \textbf{Rodzaj} & \textbf{Grubość miedzi [$\mu m$]} & \textbf{Wys. H1 [mm]} & \textbf{Wys. H2 [mm]} & \textbf{Szer. ścieżki [mm]}  & \textbf{Odległość [mm]}   & \textbf{Impedancja różn. [$\Omega$]}   & \textbf{Impedancje poj. [$\Omega$]}\\

    \hline
    \hline
    TOP 	& 	Edge cpl. ext 		&	18	&	0.11	&	0	& 	0.22	&	0.33	&	78.202	&	40.184	\\
    GND 	& 	Plane		 		&	18	&		&	0	& 		&		&			&			\\	
    L1 		& 	Edge Cpld Int Asym 	&	18	&	0.1	&	0.11	& 	0.13	&	0.26	&	80.53		&	40.826	\\
    POWER 	& 	Plane		 		&	18	&		&	0	& 		&		&			&			\\
    L2		& 	Edge Cpld Int Asym 	&	18	&	0.1	&	0.11	& 	0.13	&	0.26	&	80.53		&	40.826	\\
    POWER 	& 	Plane		 		&	18	&		&	0	& 		&		&			&			\\
    L3		& 	Edge Cpld Int Asym 	&	18	&	0.1	&	0.11	& 	0.13	&	0.26	&	80.53		&	40.826	\\
    POWER 	& 	Plane		 		&	18	&		&	0	& 		&		&			&			\\
    L4		& 	Edge Cpld Int Asym 	&	18	&	0.1	&	0.11	& 	0.13	&	0.26	&	80.53		&	40.826	\\
    POWER 	& 	Plane		 		&	18	&		&	0	& 		&		&			&			\\
    L5 		& 	Edge Cpld Int Asym 	&	18	&	0.1	&	0.11	& 	0.13	&	0.26	&	80.53		&	40.826	\\
    POWER 	& 	Plane		 		&	18	&		&	0	& 		&		&			&			\\
    L6 		& 	Edge Cpld Int Asym 	&	18	&	0.1	&	0.11	& 	0.13	&	0.26	&	80.53		&	40.826	\\
    GND 	& 	Plane		 		&	18	&		&	0	& 		&		&			&			\\
    BTM 	& 	Edge cpl. ext 		&	18	&	0.11	&	0	& 	0.22	&	0.33	&	78.202	&	40.184	\\
    
	\toprule
    \end{tabular}

\end{sidewaystable}

Pozostałe tabele z wyliczeniami szerokości ścieżek zostały dołączone w na płycie CD [\ref{CDROM}], w pliku \textbf{AMC\_DSP\_High\_Speed\_Interfaces.xls}. 

%\section{Symulacje}

\section{Podsumowanie}
Powyższy rozdział opisuje najważniejsze reguły prowadzenia ścieżek na akceleratorze. Zastosowanie tych reguł pozwala zakładać, iż interfejsy pomiędzy układami będą pracować poprawnie. Do pełnej analizy sygnałowej wymagane są jednak symulacje elektromagnetyczne.


%Symulacje
\chapter{Symulacje integralności sygnałowej i zasilania}
\label{chapt:symulacje}
\section{Wstęp}
Rozdział ten zawiera opis przeprowadzonych symulacji integralności sygnałowej oraz integralności zasilania.  Symulacje są bardzo istotną częścią projektowania szybkich systemów cyfrowych, ponieważ minimalizują liczbę błędów w projekcie jeszcze przed produkcją zmniejszając znacznie koszta i czas realizacji urządzenia. Obecnie w projektach wykorzystujących szybkie interfejsy szeregowe symulacje są wręcz konieczne. Kolejną kwestią są symulacje zasilania urządzeń elektronicznych, które również pozwalają na znalezienie błędów w projekcie, w postaci np. zbyt wąskich ścieżek doprowadzających zasilanie do układu czy zbyt małej liczby przelotek łączących wyjście przetwornicy z płaszczyzną zasilania. Symulacje akceleratora obliczeniowego wykonano w programie Hyperlynx \cite{HYPERLYNX} w wersji 8.1.

\section{Ograniczenia}
Wykonanie wszystkich wymaganych symulacji nie było możliwe ze względu na brak modeli IBIS-AMI do poszczególnych układów: 
\begin{itemize}
\item DSP TMS320C6678
\item FPGA XC7A200T
\item ADN4604
\item PEX8616
\item złącz miniSAS
\end{itemize}

Wykonane zostały symulacje części linii zegarowych oraz symulacja spadku napięcia zasilania na poszczególnych - najważniejszych liniach zasilających. Symulacja połączeń sygnałów zegarowych częściowo weryfikuje inne połączenia, ponieważ przy projektowaniu obu rodzajów połączeń stosowano te same reguły projektowe. 

\section{Opis programu Hyperlynx}
Program Hyperlynx firmy Mentor Graphics \cite{MENTOR}, służy do symulacji integralności sygnałowej i integralności zasilania projektów obwodów drukowanych. Wykorzystywana wersja jest tzw. \textit{solverem} 2D tzn. symulacja jest wykonywana w dwóch wymiarach. Powoduje to pewne ograniczania i dopiero najnowsza wersja pozwala na pełną symulację 3D. 

Przy symulacji należy wziąć pod uwagę sposób w jaki Hyperlynx wykonuje obliczenia. Przykładowo program nie bierze pod uwagę rozdzielonych płaszczyzn oraz zakłada, że ścieżka powrotna sygnału jest idealna. 

Pomimo tych ograniczeń Hyperlynx pozwala na sprawdzenie czy zaprojektowane urządzenie spełnia podstawowe założenie projektowe. 


\section{Symulacja spadku napięcia linii zasilania }
Zasilanie jest krytyczną częścią każdego urządzenia elektronicznego. Nowoczesne układy scalone wymagają stabilnego napięcia o odpowiedniej wartości. Za pomocą programu Hyperlynx wykonano symulację spadku napięcia na najważniejszych liniach zasilania.

\begin{figure}[here]
\begin{center}
\includegraphics[width=12cm]{grafika/P1V0.png}
\end{center}
\caption{Wizualizacja symulacji spadku napięcia na linii P1V0}
\end{figure}

W tabel [\ref{tbl:pwr_hlx}] umieszczone są wyniki symulacji spadku napięcia na najważniejszych liniach zasilania. Symulacje wykonano zgodnie z wcześniej przygotowaną tabelą estymowanego poboru mocy [\ref{tbl:dsp_power}].

\begin{table}[htb]

\centering
	\caption{Wyniki symulacji integralności zasilania DC Drop Voltage}
    \begin{tabular}{c | c}
	\toprule
    \textbf{Linia zasilania} & \textbf{Maks. spadek napięcia [mV]}\\
    \midrule
   	P1V0 	& 	12.5\\
      	P1V2 	& 	12.5\\
  	P1V5 	& 	13.5\\
      	P1V8 	& 	1.8\\
        	 P2V5 	& 	4.5\\
	 P3V3 	& 	4.5\\
    	 P12V 	& 	8.3\\
	P3V3MP 	& 	4.5\\
	VTT0 	& 	1.2\\
  	VTT1 	& 	1.7\\
 	CVDD\_DSP0 	& 	21\\
	CVDD\_DSP1 	& 	3\\
    \end{tabular}

	\label{tbl:pwr_hlx}
\end{table}



\section{Symulacje sygnałów zegarowych}
Projekt akceleratora obliczeniowego zawiera wiele krytycznych sygnałów zegarowych, których jakość jest bardzo istotna dla poprawnej pracy układów. Przykładowo procesor sygnałowy wykorzystany w projekcie wymaga aby sygnały zegarowe miały określony - maksymalny, poziom wahań częstotliwości. Ze względu na ilość wykonanych symulacji przedstawiony został przykładowy jej wynik dla najszybszego sygnału zegarowego występującego w projekcie akceleratora obliczeniowego, natomiast wyniki liczbowe pozostałych zaprezentowano w postaci tabeli [\ref{tbl:clock_hlx}]. 

%\subsection{Standardy LVDS i HCSL}
%
%Sygnały zegarowe w projekcie akceleratora są przesyłane w dwóch standardach sygnałowych LVDS i HCSL. Poniżej zamieszczona jest charakterystyka tych standardów. 
%
%\begin{figure}[!ht]
%\begin{center}
%\includegraphics[width=12cm]{grafika/lvds_hcsl.png}
%\end{center}
%\caption{Porównanie standardów sygnałowychs \cite{LVDS_HCSL}}
%\end{figure}
%
%\begin{itemize}
%\item \textbf{LVDS} - różnicowy przesył sygnału, amplituda sygnału 350 mV względem składowej stałej 1 V, wymaga tylko rezystora terminującego 100 $\Omega$ 
%\item \textbf{HCSL} - różnicowy przesył sygnału, amplituda sygnału 350 mV względem składowej stałej 0.350 V, wymaga rezystorów terminujących 50 $\Omega$  dołączonych do masy i 33 $ \Omega $ szeregowo
%
%\end{itemize}


\begin{itemize}
\item Sygnał SRIOSGMIICLK, Częstotliwość pracy 312 MHz. Wysokość diagramu oka 1.26 V , szerokość diagramu oka 1.38 ns, impedancja różnicowa linii 104.4 $\Omega$. 

\end{itemize}

\begin{figure}[!ht]
\begin{center}
\includegraphics[width=12cm]{grafika/sriosgmii_dsp1.png}
\end{center}
\caption{Wykres diagramu oka dla sygnału zegarowego SGMIISRIOCLK}
\end{figure}



\begin{sidewaystable}[htb]
\centering
\scriptsize

	\caption{Wyniki symulacji integralności sygnałowej sygnałów zegarowych}
    \begin{tabular}{c|c|c|c|c|c|c|c|c|c|c|c}
	\toprule
    \textbf{Sygnal zegarowy} & \textbf{Źródło} & \textbf{Odbiornik} & \textbf{Wys. ED [V]} & \textbf{Szer. ED [ns]} & \textbf{$V_{IH}$ maks. [V]} & \textbf{$V_{IL}$ [min] [mV]}  & 		\textbf{Częst. [MHz]}	& 	\textbf{$T_{r}$ [ns]} 		& 	\textbf{$T_{f}$[ns]}    	& \textbf{Typ} & \textbf{Terminacja}\\
    \midrule
   SRIOSGMIICLK 	& 		CDCM6208 (IC45) 		& 	 	TMS320C6678 (IC43)  	&	1.26		& 	1.38	&	1.56	&	94	&	312	& 	0.558	 	&	0.553 		&		LVDS		&		AC\\	
   DDRCLK	 	& 		CDCM6208 (IC45)		& 	 	TMS320C6678 (IC43)  	&	2.04		& 	7.82	&	2.2	&	64	&	67	& 	1.9	 	&	1.8		&		LVDS		&		AC\\
   MCMCLK	 	& 		CDCM6208 (IC45) 		& 	 	TMS320C6678 (IC43)  	&	1.42		&	1.56	&	1.57	&	53	&	312	&	485		&	470		&		LVDS		&		AC\\
   CORECLK	 	& 		CDCM6208 (IC45) 		& 	 	TMS320C6678 (IC43)  	&	0.998		&	3.3	&	1.94	&	280	&	100	&	2.391		&	2.43		&		LVDS		&		AC\\
   PCIECLK	 	& 		CDCM6208 (IC45) 		& 	 	5V41068a   (IC46)   	&	0.489		& 	4.1	&	1.24	&	600	&	100	&	887		&	883		&		HCSL		&		Thevenina, AC\\
   PASSCLK	 	& 		CDCM6208 (IC45) 		& 	 	TMS320C6678 (IC43)  	&	1.13		&	3.675	&	1.97	&	256	&	100	&	2.3		&	2.3		&		LVDS		&		AC\\
   PCIECLK	 	& 		5V41068a   (IC46) 		& 	 	TMS320C6678 (IC43)  	&	1.06		&	4.744	&	1.1	&	8.95	&	100	&	461		&	461		&		HCSL		&		Thevenina, AC\\
\hline
\hline
   SRIOSGMIICLK 	& 		CDCM6208 (IC25) 		& 	 	TMS320C6678 (IC41)  	&	1.26		&	1.265	&	1.56	&	112	&	312	&	574		&	571		&		LVDS		&		AC\\	
   DDRCLK	 	& 		CDCM6208 (IC25)		& 	 	TMS320C6678 (IC41)  	&	1.44		&	5.43	&	203	&	200	&	67	&	3.5		&	3.7		&		LVDS		&		AC\\
   MCMCLK	 	& 		CDCM6208 (IC25) 		& 	 	TMS320C6678 (IC41)  	&	1.1		&	1.474	&	1.54	&	130	&	312	&	543		&	529		&		LVDS		&		AC\\
   CORECLK	 	& 		CDCM6208 (IC25) 		& 	 	TMS320C6678 (IC41)  	&	1.44		&	3.989	&	2.04	&	191	&	100	&	2.472		&	2.662		&		LVDS		&		AC\\
   PCIECLK	 	& 		CDCM6208 (IC25) 		& 	 	5V41068a   (IC26)   	&	543		&	4.7	&	543	&	33	&	100	&	673		&	636		&		HCSL		&		Thevenina, AC\\
   PASSCLK	 	& 		CDCM6208 (IC25) 		& 	 	TMS320C6678 (IC41)  	&	999		&	4.1	&	1.95	&	277	&	100	&	2.343		&	2.4		&		LVDS		&		AC\\
   PCIECLK	 	& 		5V41068a   (IC26) 		& 	 	TMS320C6678 (IC41)  	&	1.11		&	4.738	&	1.12	&	7.95	&	100	&	477		&	475		&		HCSL		&		Thevenina, AC\\
							


    \end{tabular}
	\label{tbl:clock_hlx}
\end{sidewaystable}






\section{Pozostałe symulacje i analizy}
Pomiędzy ścieżkami w obwodzie drukowanym, znajdującymi się zbyt blisko siebie, występują przesłuchy, które mogą powodować przekłamanie transmisji np. interfejsu I2C. Program Hyperlynx pozwala na znalezienie błędów projektowych, które mogą powodować to zjawisko. Wykonana symulacja Signal Integrity Batch Simulation została przeanalizowana i poprawione zostały połączenia zaproponowane przez program jako te mogące powodować zakłócenia na innych liniach. Poprawienie polega na odseparowaniu od siebie poszczególnych ścieżek na obwodzie drukowanym.  

Inną analizą jaką można wykonać za pomocą Hyperlynx jest analiza poprawności połączenia kondensatora z siecią zasilania. Wykonana symulacja przedstawiała jednak często wyniki niezgodne z projektem obwodów drukowanych. Przykładowo analiza wykazała, że dany kondensator nie jest dołączony do płaszczyzny zasilania lub zostały przeanalizowane kondensatory, które nie należą do grupy wybranej do analizy. Przedstawia to istotny problem, przy symulacjach bardzo istotną kwestią jest krytyczne podejście do wyników. 

\section{Podsumowanie}
Przeprowadzone symulacje integralności sygnałowej oraz integralności zasilania pozwalają zakładać, że obwody drukowane zostały zaprojektowane poprawnie. Niestety mimo usilnych starań nie udało się uzyskać potrzebnych modeli IBIS-AMI od producentów układów scalonych co uniemożliwiło symulacje interfejsów SerDes.




%Oprogramowanie modułu akwizycji i sterowanie
%Zamiast robić 2 oddzielne rozdziały, robię jeden i dzielę go na dwie sekcje
\chapter{Oprogramowanie uruchamiające moduł}
\label{chapt:oprogramowanie}


\section{Wstęp}
Poniższy rozdział zawiera opis oprogramowania uruchamiającego akcelerator obliczeniowy.W pierwszej części opisano sterowanie przetwornicami, a następnie opis konfiguracji poszczególnych układów.  Oprogramowanie zawiera się w dwóch plikach: \textbf{adsp.c}, \textbf{adsp.h} i dodaje funkcję obsługi akceleratora obliczeniowego do istniejącego software'u wykorzystywanego w modułach AMC w laboratorium PERG. 


\section{Opis mikrokontrolera LPC1764}

Układ LPC1764 jest 32 bitowym mikrokontrolerem z rdzeniem Cortex-M3 firmy NXP. Jest to rozbudowany układ, który posiada wiele sprzętowych peryferiów takich jak kontrolery interfejsów I2C, CAN czy przetwornik analogowo cyfrowy. Główną jego funkcją jest sterowanie procesem załadowania pamięci programu do układu FPGA oraz obsługa standardu IPMI. Poza tym układ zajmuje się sterowaniem pinów konfiguracyjnych poszczególnych układów oraz konfiguracją ich rejestrów wewnętrznych. Na rysunku [\ref{I2C_BLOCK}] przedstawiona jest struktura połączeń interfejsu I2C w projekcie akceleratora obliczeniowego. 

\subsection{Środowisko programistyczne}
Producent, firma NXP \cite{COMPANY:NXP} udostępnia środowisko programistyczne \textit{LPCXpresso} oparte o znane IDE (\textit{Integrated Developement Environment}) Eclipse \cite{ECLIPSE}. Producent udostępnia do układu kompilator ANSI C oraz podstawowe biblioteki do obsługi peryferiów.

\begin{figure}[!ht]
\centering
\includegraphics[width=12cm]{grafika/lpcxpresso.png}
\caption{Środowisko programistyczne LPCXpresso}
\end{figure}

\section{Uruchomienie linii zasilających}

Zarówno procesor DSP jak i FPGA wymagają aby poszczególne napięcia były uruchamiane sekwencyjnie. Procesor DSP ma dwa scenariusze uruchamiania zasilania \textit{Core before IO} i \textit{IO before Core} natomiast FPGA ma jeden, który jak się okazuje w praktyce nie ma większego znaczenia \cite{DATASHEET:TMS} \cite{FPGA:DS181}. Poniższy schemat prezentuje proces uruchamiania napięć zasilających tak, aby spełnić wymagania dla wszystkich układów.


 \begin{figure}[!ht]
\centering
\includegraphics[width=8cm]{grafika/power_sequence.png}
\caption{Sekwencja uruchamiania linii zasilających}
\end{figure}

 
Pierwszą linią zasilania uruchamianą na module jest napięcie 3.3V, ze względu na to że jest to napięcie wymagane przez kontroler przetwornicy układ UCD9222 do pracy. Następnie uruchamiane jest regulowane napięcie rdzenia procesorów DSP (linie CVDD\_DSP0 i CVDD\_DSP1). Kolejnym krokiem jest uruchomienie napięcia 1.0V, 1.2V, 1.8V.  W tym momencie konieczna jest konfiguracja układów generacji sygnału zegarowego tak aby procesory DSP poprawnie się uruchomiły. Po konfiguracji uruchamiana jest linia 1.5V, w tym momencie procesory DSP są poprawnie zasilone i czekają na zmianę stanu pinu \textbf{RESETn}, który rozpocznie proces ładowania \textit{bootloadera} pierwszego poziomu. Ta operacja jest wykonywana przez FPGA, które do pełnego uruchomienia potrzebuje jeszcze napięć 2.5V.  Układ FPGA kontroluje również piny konfiguracyjne \textit{GPIO} procesorów DSP, które ustalają ich tryb pracy. 

\section{Konfiguracja poszczególnych układów}

\subsection{CDCM6208}
Układy generacji sygnału zegarowego CDCM6208 są konfigurowane poprzez interfejs I2C i pracują w trybie \textit{Slave}. Ich adresy to kolejno: \textit{0x54} i \textit{0x55}. Po uruchomieniu linii zasilania \textbf{P1V8} odbywa się konfiguracja wewnętrznych rejestrów układów. Wartości rejestrów zostały wygenerowane za pomocą programu dostarczonego przez producenta \textbf{CDCM6208} \cite{CDCM6208:INFO2}.  Adresy ukladu są 16 bitowe, ale wypełnianie ich odbywa się poprzez wysłanie dwóch słów 8 bitowych.


 \begin{figure}[!ht]
\centering
\includegraphics[width=12cm]{grafika/cdcm6208_conf.png}
\caption{Proces programowania układu CDCM6208}
\end{figure}

%\subsection{AD9522}

%\subsection{ADN4604}
%Przełącznik interfejsu 


\subsection{UCD9222}

Napięcie regulowane rdzenia procesorów DSP jest kontrolowane przez układ UCD9222, który do poprawnej pracy wymaga konfiguracji poprzez interfejs PMBus. Wykorzystując oprogramowanie producenta \textit{Fusion Digital Power Designer} \cite{FUSION} oraz przykładowy plik konfiguracyjny z modułu ewaluacyjnego \cite{UCD9222_EVM} przygotowano konfigurację generującą odpowiednie napięcia dla obu procesorów DSP. 

 \begin{figure}[!ht]
\centering
\includegraphics[width=12cm]{grafika/ucd9222.png}
\caption{Konfiguracja programu Fusion Digital Power Designer}
\end{figure}

Konfiguracja układu może odbywać się za pomocą zewnętrznego programatora lub poprzez mikrokontroler LPC1764, jednak ta funkcjonalność jest opcjonalna. Kontroler przetwornicy zawiera w sobię pamięć nieulotną FLASH, na której zapisuje konfigurację rejestrów i nie potrzebna jest jej aktualizacja podczas uruchamiania układu. 

\section{Podsumowanie}

Przygotowane oprogramowanie pozwala na uruchomienie akceleratora obliczeniowego do momentu programowania układu FPGA.  


%Inicjalizacja pinów mikrokontrolera
%Uruchomienie SPI FPGA
%

%TESTOWANIE, BADANIE
%\chapter{Uruchomienie i testowanie}
%\label{chapt:uruchomienie}
%% TeX encoding = utf8
% TeX spellcheck = pl_PL

%Uruchomienie i testowanie

\section{Wstęp}
Poniższy rozdział przedstawia proces uruchomienia i testów elektrycznych zaprojektowanego urządzenia. Zawarto w nim opis oprogramowania uruchamiającego napisanego na mikrokontroler LPC1764 oraz przebieg testów uruchomieniowych poszczególnych układów.

\section{IPMI}

\section{Uruchomienie zasilania}

Układ LPC1769 ma za zadanie uruchomić poszczególne linie zasilania poprzez wysterowanie linii \textbf{enable} przetwornic i sprawdzenie linii \textbf{Power OK}. Procesory DSP jak i układ FPGA wymagają określonej sekwencji uruchamiania linii zasilających. Aby uruchomić poprawnie zarówno DSP jak i FPGA sekwencja zasilania przebiega według scenariusza \textit{Core before IO} \cite[page=129]{TMS}. 

\begin{center}
% Define block styles
\tikzstyle{decision} = [diamond, draw, fill=blue!20, 
    text width=4.5em, text badly centered, node distance=3cm, inner sep=0pt]
\tikzstyle{block} = [rectangle, draw, fill=blue!20, 
    text width=5em, text centered, rounded corners, minimum height=3em]
\tikzstyle{line} = [draw, -latex']
\tikzstyle{cloud} = [draw, ellipse,fill=red!20, node distance=3cm,
    minimum height=2em]
    
\begin{tikzpicture}[node distance = 2cm, auto]
    % Place nodes
    \node [block] (cvdd0_en) {Enable DSP0 CVDD};
    \node [cloud, left of=cvdd0_en] (section_init) {Init Power};
    
    \node [block,below of=cvdd0_en,] (cvdd1_en) {Enable DSP1 CVDD};
    
    \node [block,below of=cvdd1_en,] (p1v0_en) {Enable P1V0};    
    
     \node [block,below of=p1v0_en,] (p1v2_en) {Enable P1V2}; 
         
     \node [block,below of=p1v2_en,] (p1v8_en) {Enable P1V8};          
    
     \node [block,below of=p1v8_en,] (p3v3_en) {Enable P3V3};         
       
     \node [block,below of=p3v3_en,] (p1v5_en) {Enable P1V5}; 
     \node [decision, below of=p1v5_en,node distance=3cm] (p1v5_ok) {is power good?};
      
      \node [block, right of=p1v5_ok,node distance=4cm,fill=red!20] (error) {Error};  
      
      \node [block,fill=green!20,below of=p1v5_ok,node distance=3cm] (init_comp) {Init complete}; 
       
       \node [block, right of=error,node distance=3cm,fill=red!20] (reset) {Reset};  
    % Draw edges
    \path [line] (section_init) -- (cvdd0_en);
    
    \path [line] (cvdd0_en) -- (cvdd1_en);
    
    \path [line] (cvdd1_en) -- (p1v0_en);
    
    \path [line] (p1v0_en) -- (p1v2_en);
    
    \path [line] (p1v2_en) -- (p1v8_en);	
    
    \path [line] (p1v8_en) -- (p3v3_en);	
    
    \path [line] (p3v3_en) -- (p1v5_en);	
    
    \path [line] (p1v5_en) -- (p1v5_ok);
    
    \path [line] (p1v5_ok) -- node [near start] {no} (error);
    
    \path [line] (p1v5_ok) -- node [near start]{yes}(init_comp);
    
    \path [line] (error) -- (reset);	
         
         
\end{tikzpicture}

\end{center}

Napisane oprogramowanie wykonuje powyższą funkcjonalność. Sygnały Power Good linii 1.0V, 1.2V, 1.5V, 1.8V i 3.3V są zwarte ze sobą. 


\section{Konfiguracja układów}

\subsection{UCD9222}
\subsection{CDCM6208}
\subsection{AD9522}
\subsection{ADN4604}
\subsection{CDCUN1208}

\section{Oprogramowanie FPGA}
W tej części zawarto opis oprogramowania na układ XC7A200T, które służy do poprawnego uruchomienia procesorów DSP.
\subsection{Założenia}
\subsection{Opis oprogramowania}

\section{Uruchomienie DSP}

\section{Testy elektryczne}

\section{Podsumowanie}







%PODSUMOWANIE
\chapter{Wnioski końcowe}
\label{chapt:wnioski}
% TeX encoding = utf8
% TeX spellcheck = pl_PL

%Wnioski koncowe

Projekt rekonfigurowalnego akceleratora obliczeniowego w standardzie AMC zrealizowano zgodnie z założeniami. Urządzenie jest gotowe do produkcji i wstępnego uruchamiania. W ramach Pracy Dyplomowej Inżynierskiej zrealizowano:
\begin{itemize}
\item
Projekt schematów elektrycznych
\item
Projekt obwodów drukowanych
\item
Symulacje integralności sygnałowej i integralności zasilania
\item
Oprogramowanie uruchamiające
\end{itemize}

Zaprojektowany akcelerator obliczeniowy pozwala na analizę danych w czasie rzeczywistym. Ponadto jest bardzo elastyczny dzięki zastosowaniu nowoczesnego układu FPGA oraz układów przełączników interfejsów szeregowych. W porównaniu do urządzeń istniejących na rynku, akcelerator obliczeniowy prezentuje się bardzo konkurencyjnie, zarówno pod względem funkcjonalności jak i ceny. 

\begin{table}[here]
\centering
\scriptsize
    \begin{tabular}{p{2.5cm}| p{2.5cm} | p{2.5cm}|p{2cm}|p{2cm}|p{1.5cm}}
	\toprule
    \textbf{Urządzenie} & \textbf{Procesory} & \textbf{Standard} & \textbf{Złącza}  & \textbf{Interfejsy MCH}  & \textbf{Cena}\\
    \midrule
    AMC V7-2C6678 	& 	Xilinx XC7VX415T-2  2xTMS320C6678  	& 	AMC Single width, full size		&	SFP, SAS		&	 	SRIO				&	16 000 USD\\
    PDAK2H 		& 	66AK2H12				  	& 	AMC Single width, full size		&	GbE, UART		&	 	SRIO, PCIe			&	BD\\
   Akcelerator obliczeniowy	&	XC7A200T, 2xTMS320C6678	&	AMC Double width, medium size	&	2x SAS, 2x GbE, HDMI(Clock)	&	SRIO, PCIe, SGMII, MLVDS	&	2500 PLN (prototyp)\\ 
     \end{tabular}
	\caption{Porównanie komercyjnych systemów przetwarzania z projektowanym modułem}
	\label{tbl:cots2}
\end{table}


\paragraph{Dalszy rozwój projektu}
W ramach Pracy Magisterskiej planowane jest uruchomienie modułu oraz dalszy jego rozwój polegający na napisaniu natywnego oprogramowania do analizy danych w systemie pomiarowym \textit{Beam Position Monitor Digital Back End}.
 

%DODATKI
\begin{appendices}
\label{chapt:dodatki}
\chapter{Developement version of pixel ordering}
\section{Image data}
Nominal size of CCD sensor is 4096x4096 pixels. Each pixel is sampled with 18-bit resolution and consists of two samples: one of video level and another of black level. This gives in total 4096x4096x2x2 = 64MB of data. Due to high SNR requirements each pixel has to be oversampled. Two agreed oversampling factors are 4x and 16x. Taking into account oversampling, final frame sizes are respectively 256MB and 1024MB. \\
Image data is contained in one continuous memory block but due to CCD readout and oversampling method, order of image pixels in memory is different than physical one in CCD matrix. CCD matrix is divided into four identical groups of pixels which have independent readout channels as shown on figure \ref{fig:readout}. \\
Due to order of ADC triggering final image data layout is different. Image data ordering in memory (for 4x oversampling mode) is presented on figure \ref{fig:oversample4}.

%\begin{figure}[H]
%\centering
%\includesvg[clean, pretex=\relscale{0.7}, width=\textwidth, svgpath = pict_ipc/]{image_data_format}
%\caption{Image data format for 4x oversampling}
%\label{fig:oversample4}
%\end{figure}

\begin{figure}[H]
\centering
\includegraphics[width=\textwidth]{pict_ipc/image_data_format.png}
\caption{Image data format for 4x oversampling}
\label{fig:oversample4}
\end{figure}

\emph{pix-n-H-v-0} means: n-th pixel from section H video level sample 0. Similarly \emph{pix-n-E-b-2} means: n-th pixel from section E black level sample 2.

\section{Sensors list}

\begin{table}
\footnotesize
\begin{tabular}{ | l | l | l | l | l | l | l | l | l | }
\hline
	Board & Parameter & Sensor & Resolution & Accuracy & Range & Ext. Range & \begin{tabular}{@{}c@{}} Max Read \\ Freq. [Hz] \end{tabular} & Qty \\ \hline
	Comm. & Humidity & STH25 & 0,04 \%RH & +/- 1,8 \%RH & 10 – 90 \%RH & 0-100 \%RH & 0.125 & 1 \\ \hline
	 & Temperature &  & 0,04 \degree C & +/- 0,2 \degree C & 0-60 \degree C & -40-120 \degree C & 0,2 – 0,033 & 1 \\ \hline
	 & Temperature & MAX6639 & 0,125 \degree C & +/- 1\degree C & 0 – 125 \degree C & - & 8(1) & 1 \\ \hline
	Main & Humidity & HDC1000 & 0,006\%RH & +/- 3 \%RH & 10 – 80 \%RH & 0-100 \%RH & 6.6E-2 & 1 \\ \hline
	 & Temperature &  & 0,01 \degree C & +/- 0.2 \degree C & -40-125 \degree C & - & - & 1 \\ \hline
	 & Acceleration & ADXL343 & 0,004g & +/- 1\% & \begin{tabular}{@{}c@{}}+/-16g,+/-8g \\ +/-4g/+/-2g\end{tabular} & - & 0.1 – 3200 & 1 \\ \hline
	 & Temperature & ADS1248 & 5,96u \degree C & +/- 1\degree C & +/- 50\degree C & - & 5-2000 & 3 \\ \hline
	 & Coolant flow & MAX6639 & - & +/- 3\degree C & 2000-16000RPM & - & - & 1 \\ \hline
	 & T. alarm (set) & LM75 & 0,3515625\degree C & +/- 2,\degree C & -55 – 125 \degree C & - & 0.33 & 1 \\ \hline
	CCD & Humidity & HDC1000 & 0,006\%RH & +/- 3 \%RH & 10 – 80 \%RH & 0-100 \%RH & 6.6E-2 & 1 \\ \hline
	 & Temperature &  & 0,01 \degree C & +/- 0.2 \degree C & -40-125 \degree C & - & - & 1 \\ \hline
\end{tabular}
\end{table}

\end{appendices}
%BIBLIOGRAFIA / LITERATURA
%----------------------------------------------------------------------------------------------------------
%\input{bibliography.tex}

\bibliographystyle{unsrt}
\bibliography{bib}
\addcontentsline{toc}{chapter}{Bibliografia}
\end{document}

