\section{Wstęp}
Rozdział ten zawiera opis przeprowadzonych symulacji integralności sygnałowej oraz integralności zasilania.  Symulacje są bardzo istotną częścią projektowania szybkich systemów cyfrowych, ponieważ minimalizują liczbę błędów w projekcie jeszcze przed produkcją zmniejszając znacznie koszta i czas realizacji urządzenia. Obecnie w projektach wykorzystujących szybkie interfejsy szeregowe symulacje są wręcz konieczne. Kolejną kwestią są symulacje zasilania urządzeń elektronicznych, które również pozwalają na znalezienie błędów w projekcie, w postaci np. zbyt wąskich ścieżek doprowadzających zasilanie do układu czy zbyt małej liczby przelotek łączących wyjście przetwornicy z płaszczyzną zasilania. Symulacje akceleratora obliczeniowego wykonano w programie Hyperlynx \cite{HYPERLYNX} w wersji 8.1.

\section{Ograniczenia}
Wykonanie wszystkich wymaganych symulacji nie było możliwe ze względu na brak modeli IBIS-AMI do poszczególnych układów: 
\begin{itemize}
\item DSP TMS320C6678
\item FPGA XC7A200T
\item ADN4604
\item PEX8616
\item złącz miniSAS
\end{itemize}

Wykonane zostały symulacje części linii zegarowych oraz symulacja spadku napięcia zasilania na poszczególnych - najważniejszych liniach zasilających. Symulacja połączeń sygnałów zegarowych częściowo weryfikuje inne połączenia, ponieważ przy projektowaniu obu rodzajów połączeń stosowano te same reguły projektowe. 

\section{Opis programu Hyperlynx}
Program Hyperlynx firmy Mentor Graphics \cite{MENTOR}, służy do symulacji integralności sygnałowej i integralności zasilania projektów obwodów drukowanych. Wykorzystywana wersja jest tzw. \textit{solverem} 2D tzn. symulacja jest wykonywana w dwóch wymiarach. Powoduje to pewne ograniczania i dopiero najnowsza wersja pozwala na pełną symulację 3D. 

Przy symulacji należy wziąć pod uwagę sposób w jaki Hyperlynx wykonuje obliczenia. Przykładowo program nie bierze pod uwagę rozdzielonych płaszczyzn oraz zakłada, że ścieżka powrotna sygnału jest idealna. 

Pomimo tych ograniczeń Hyperlynx pozwala na sprawdzenie czy zaprojektowane urządzenie spełnia podstawowe założenie projektowe. 


\section{Symulacja spadku napięcia linii zasilania }
Zasilanie jest krytyczną częścią każdego urządzenia elektronicznego. Nowoczesne układy scalone wymagają stabilnego napięcia o odpowiedniej wartości. Za pomocą programu Hyperlynx wykonano symulację spadku napięcia na najważniejszych liniach zasilania.

\begin{figure}[here]
\begin{center}
\includegraphics[width=12cm]{grafika/P1V0.png}
\end{center}
\caption{Wizualizacja symulacji spadku napięcia na linii P1V0}
\end{figure}

W tabel [\ref{tbl:pwr_hlx}] umieszczone są wyniki symulacji spadku napięcia na najważniejszych liniach zasilania. Symulacje wykonano zgodnie z wcześniej przygotowaną tabelą estymowanego poboru mocy [\ref{tbl:dsp_power}].

\begin{table}[htb]

\centering
	\caption{Wyniki symulacji integralności zasilania DC Drop Voltage}
    \begin{tabular}{c | c}
	\toprule
    \textbf{Linia zasilania} & \textbf{Maks. spadek napięcia [mV]}\\
    \midrule
   	P1V0 	& 	12.5\\
      	P1V2 	& 	12.5\\
  	P1V5 	& 	13.5\\
      	P1V8 	& 	1.8\\
        	 P2V5 	& 	4.5\\
	 P3V3 	& 	4.5\\
    	 P12V 	& 	8.3\\
	P3V3MP 	& 	4.5\\
	VTT0 	& 	1.2\\
  	VTT1 	& 	1.7\\
 	CVDD\_DSP0 	& 	21\\
	CVDD\_DSP1 	& 	3\\
    \end{tabular}

	\label{tbl:pwr_hlx}
\end{table}



\section{Symulacje sygnałów zegarowych}
Projekt akceleratora obliczeniowego zawiera wiele krytycznych sygnałów zegarowych, których jakość jest bardzo istotna dla poprawnej pracy układów. Przykładowo procesor sygnałowy wykorzystany w projekcie wymaga aby sygnały zegarowe miały określony - maksymalny, poziom wahań częstotliwości. Ze względu na ilość wykonanych symulacji przedstawiony został przykładowy jej wynik dla najszybszego sygnału zegarowego występującego w projekcie akceleratora obliczeniowego, natomiast wyniki liczbowe pozostałych zaprezentowano w postaci tabeli [\ref{tbl:clock_hlx}]. 

%\subsection{Standardy LVDS i HCSL}
%
%Sygnały zegarowe w projekcie akceleratora są przesyłane w dwóch standardach sygnałowych LVDS i HCSL. Poniżej zamieszczona jest charakterystyka tych standardów. 
%
%\begin{figure}[!ht]
%\begin{center}
%\includegraphics[width=12cm]{grafika/lvds_hcsl.png}
%\end{center}
%\caption{Porównanie standardów sygnałowychs \cite{LVDS_HCSL}}
%\end{figure}
%
%\begin{itemize}
%\item \textbf{LVDS} - różnicowy przesył sygnału, amplituda sygnału 350 mV względem składowej stałej 1 V, wymaga tylko rezystora terminującego 100 $\Omega$ 
%\item \textbf{HCSL} - różnicowy przesył sygnału, amplituda sygnału 350 mV względem składowej stałej 0.350 V, wymaga rezystorów terminujących 50 $\Omega$  dołączonych do masy i 33 $ \Omega $ szeregowo
%
%\end{itemize}


\begin{itemize}
\item Sygnał SRIOSGMIICLK, Częstotliwość pracy 312 MHz. Wysokość diagramu oka 1.26 V , szerokość diagramu oka 1.38 ns, impedancja różnicowa linii 104.4 $\Omega$. 

\end{itemize}

\begin{figure}[!ht]
\begin{center}
\includegraphics[width=12cm]{grafika/sriosgmii_dsp1.png}
\end{center}
\caption{Wykres diagramu oka dla sygnału zegarowego SGMIISRIOCLK}
\end{figure}



\begin{sidewaystable}[htb]
\centering
\scriptsize

	\caption{Wyniki symulacji integralności sygnałowej sygnałów zegarowych}
    \begin{tabular}{c|c|c|c|c|c|c|c|c|c|c|c}
	\toprule
    \textbf{Sygnal zegarowy} & \textbf{Źródło} & \textbf{Odbiornik} & \textbf{Wys. ED [V]} & \textbf{Szer. ED [ns]} & \textbf{$V_{IH}$ maks. [V]} & \textbf{$V_{IL}$ [min] [mV]}  & 		\textbf{Częst. [MHz]}	& 	\textbf{$T_{r}$ [ns]} 		& 	\textbf{$T_{f}$[ns]}    	& \textbf{Typ} & \textbf{Terminacja}\\
    \midrule
   SRIOSGMIICLK 	& 		CDCM6208 (IC45) 		& 	 	TMS320C6678 (IC43)  	&	1.26		& 	1.38	&	1.56	&	94	&	312	& 	0.558	 	&	0.553 		&		LVDS		&		AC\\	
   DDRCLK	 	& 		CDCM6208 (IC45)		& 	 	TMS320C6678 (IC43)  	&	2.04		& 	7.82	&	2.2	&	64	&	67	& 	1.9	 	&	1.8		&		LVDS		&		AC\\
   MCMCLK	 	& 		CDCM6208 (IC45) 		& 	 	TMS320C6678 (IC43)  	&	1.42		&	1.56	&	1.57	&	53	&	312	&	485		&	470		&		LVDS		&		AC\\
   CORECLK	 	& 		CDCM6208 (IC45) 		& 	 	TMS320C6678 (IC43)  	&	0.998		&	3.3	&	1.94	&	280	&	100	&	2.391		&	2.43		&		LVDS		&		AC\\
   PCIECLK	 	& 		CDCM6208 (IC45) 		& 	 	5V41068a   (IC46)   	&	0.489		& 	4.1	&	1.24	&	600	&	100	&	887		&	883		&		HCSL		&		Thevenina, AC\\
   PASSCLK	 	& 		CDCM6208 (IC45) 		& 	 	TMS320C6678 (IC43)  	&	1.13		&	3.675	&	1.97	&	256	&	100	&	2.3		&	2.3		&		LVDS		&		AC\\
   PCIECLK	 	& 		5V41068a   (IC46) 		& 	 	TMS320C6678 (IC43)  	&	1.06		&	4.744	&	1.1	&	8.95	&	100	&	461		&	461		&		HCSL		&		Thevenina, AC\\
\hline
\hline
   SRIOSGMIICLK 	& 		CDCM6208 (IC25) 		& 	 	TMS320C6678 (IC41)  	&	1.26		&	1.265	&	1.56	&	112	&	312	&	574		&	571		&		LVDS		&		AC\\	
   DDRCLK	 	& 		CDCM6208 (IC25)		& 	 	TMS320C6678 (IC41)  	&	1.44		&	5.43	&	203	&	200	&	67	&	3.5		&	3.7		&		LVDS		&		AC\\
   MCMCLK	 	& 		CDCM6208 (IC25) 		& 	 	TMS320C6678 (IC41)  	&	1.1		&	1.474	&	1.54	&	130	&	312	&	543		&	529		&		LVDS		&		AC\\
   CORECLK	 	& 		CDCM6208 (IC25) 		& 	 	TMS320C6678 (IC41)  	&	1.44		&	3.989	&	2.04	&	191	&	100	&	2.472		&	2.662		&		LVDS		&		AC\\
   PCIECLK	 	& 		CDCM6208 (IC25) 		& 	 	5V41068a   (IC26)   	&	543		&	4.7	&	543	&	33	&	100	&	673		&	636		&		HCSL		&		Thevenina, AC\\
   PASSCLK	 	& 		CDCM6208 (IC25) 		& 	 	TMS320C6678 (IC41)  	&	999		&	4.1	&	1.95	&	277	&	100	&	2.343		&	2.4		&		LVDS		&		AC\\
   PCIECLK	 	& 		5V41068a   (IC26) 		& 	 	TMS320C6678 (IC41)  	&	1.11		&	4.738	&	1.12	&	7.95	&	100	&	477		&	475		&		HCSL		&		Thevenina, AC\\
							


    \end{tabular}
	\label{tbl:clock_hlx}
\end{sidewaystable}






\section{Pozostałe symulacje i analizy}
Pomiędzy ścieżkami w obwodzie drukowanym, znajdującymi się zbyt blisko siebie, występują przesłuchy, które mogą powodować przekłamanie transmisji np. interfejsu I2C. Program Hyperlynx pozwala na znalezienie błędów projektowych, które mogą powodować to zjawisko. Wykonana symulacja Signal Integrity Batch Simulation została przeanalizowana i poprawione zostały połączenia zaproponowane przez program jako te mogące powodować zakłócenia na innych liniach. Poprawienie polega na odseparowaniu od siebie poszczególnych ścieżek na obwodzie drukowanym.  

Inną analizą jaką można wykonać za pomocą Hyperlynx jest analiza poprawności połączenia kondensatora z siecią zasilania. Wykonana symulacja przedstawiała jednak często wyniki niezgodne z projektem obwodów drukowanych. Przykładowo analiza wykazała, że dany kondensator nie jest dołączony do płaszczyzny zasilania lub zostały przeanalizowane kondensatory, które nie należą do grupy wybranej do analizy. Przedstawia to istotny problem, przy symulacjach bardzo istotną kwestią jest krytyczne podejście do wyników. 

\section{Podsumowanie}
Przeprowadzone symulacje integralności sygnałowej oraz integralności zasilania pozwalają zakładać, że obwody drukowane zostały zaprojektowane poprawnie. Niestety mimo usilnych starań nie udało się uzyskać potrzebnych modeli IBIS-AMI od producentów układów scalonych co uniemożliwiło symulacje interfejsów SerDes.


