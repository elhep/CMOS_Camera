% TeX encoding = utf8
% TeX spellcheck = pl_PL

%Wnioski koncowe

Projekt rekonfigurowalnego akceleratora obliczeniowego w standardzie AMC zrealizowano zgodnie z założeniami. Urządzenie jest gotowe do produkcji i wstępnego uruchamiania. W ramach Pracy Dyplomowej Inżynierskiej zrealizowano:
\begin{itemize}
\item
Projekt schematów elektrycznych
\item
Projekt obwodów drukowanych
\item
Symulacje integralności sygnałowej i integralności zasilania
\item
Oprogramowanie uruchamiające
\end{itemize}

Zaprojektowany akcelerator obliczeniowy pozwala na analizę danych w czasie rzeczywistym. Ponadto jest bardzo elastyczny dzięki zastosowaniu nowoczesnego układu FPGA oraz układów przełączników interfejsów szeregowych. W porównaniu do urządzeń istniejących na rynku, akcelerator obliczeniowy prezentuje się bardzo konkurencyjnie, zarówno pod względem funkcjonalności jak i ceny. 

\begin{table}[here]
\centering
\scriptsize
    \begin{tabular}{p{2.5cm}| p{2.5cm} | p{2.5cm}|p{2cm}|p{2cm}|p{1.5cm}}
	\toprule
    \textbf{Urządzenie} & \textbf{Procesory} & \textbf{Standard} & \textbf{Złącza}  & \textbf{Interfejsy MCH}  & \textbf{Cena}\\
    \midrule
    AMC V7-2C6678 	& 	Xilinx XC7VX415T-2  2xTMS320C6678  	& 	AMC Single width, full size		&	SFP, SAS		&	 	SRIO				&	16 000 USD\\
    PDAK2H 		& 	66AK2H12				  	& 	AMC Single width, full size		&	GbE, UART		&	 	SRIO, PCIe			&	BD\\
   Akcelerator obliczeniowy	&	XC7A200T, 2xTMS320C6678	&	AMC Double width, medium size	&	2x SAS, 2x GbE, HDMI(Clock)	&	SRIO, PCIe, SGMII, MLVDS	&	2500 PLN (prototyp)\\ 
     \end{tabular}
	\caption{Porównanie komercyjnych systemów przetwarzania z projektowanym modułem}
	\label{tbl:cots2}
\end{table}


\paragraph{Dalszy rozwój projektu}
W ramach Pracy Magisterskiej planowane jest uruchomienie modułu oraz dalszy jego rozwój polegający na napisaniu natywnego oprogramowania do analizy danych w systemie pomiarowym \textit{Beam Position Monitor Digital Back End}.
 