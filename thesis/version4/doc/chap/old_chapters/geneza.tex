Projektowany system pomiarowy w zespole PERG Instytutu Systemów Elektronicznych \textit{Beam Position Monitor Digital Back End} jest oparty o standard MTCA. Zwiększanie funkcjonalności tego systemu polega na dołączeniu specjalizowanych modułów AMC. Jednym z potrzebnych funkcji jest przetwarzanie danych w czasie rzeczywistym, do którego wymagany jest akcelerator obliczeń. Dostępne na rynku urządzenia nie posiadają odpowiednich wejść/wyjść np. SAS/SATA lub nie pozwalają na modyfikację połączeń między układami wewnątrz urządzenia tj. nie są elastyczne; ponadto są one bardzo kosztowne. Dlatego zdecydowano się na zaprojektowanie dedykowanego urządzenia.  \\

Celem niniejszej pracy inżynierskiej było zaprojektowanie modułu w formacie AMC pozwalającego na akwizycję i analizę dużej ilości danych pomiarowych w czasie rzeczywistym, na co składa się:
\begin{itemize}
\item projekt schematów elektrycznych
\item projekt obwodów drukowanych
\item symulacje integralności sygnałowej i zasilania
\item napisanie oprogramowania uruchamiającego
\end{itemize}

Akcelerator obliczeniowy będzie rozszerzeniem projektowanego systemu pomiarowego i pozwoli na przetwarzanie i analizę danych  z zastosowaniem zaawansowanych technik cyfrowego przetwarzania sygnałów. Przykładowym zastosowaniem będą eksperymenty JET i WEST, w których obecnie działa poprzednia wersja systemu. Ze względu na to iż system pomiarowy, w którym ma pracować projektowany moduł jest w trakcie realizacji wstrzymano się z produkcją modułu.\\
 
Urządzenie zaprojektowano przy następujących założeniach:


\begin{itemize}
		\item
		zgodność ze standardem AMC
		
		\item
		możliwość rekonfiguracji połączeń między układami
		
		\item
		możliwość komunikacji poprzez interfejs SAS/SATA
		
		\item
		minimalizacja kosztów

\end{itemize}


