% TeX encoding = utf8
% TeX spellcheck = pl_PL

%ROZDZIAL 1 - WPROWADZENIE

Współczesne systemy pomiarowe w eksperymentach fizyki wysokich energii są zaawansowanymi systemami analogowo-cyfrowymi o niestandardowych parametrach. Eksperymenty fizyczne JET \cite{JET} czy WEST \cite{WEST}  podczas pracy generują terabajty danych na sekundę i aby móc obsłużyć taką ilość danych potrzebne są rozwiązania dedykowane, nierzadko przełomowe. Przykładem takiego systemu może być system pomiarowy służący do detekcji promieniowania X z detektora GEM \cite{GEM} wykorzystywany w eksperymencie JET i pozwalający na analizę plazmy w reaktorze termojądrowym. 

 Niniejsza praca stanowi opis projektu urządzenia służącego do przetwarzania danych -\textbf{akceleratora obliczeniowego}, który zostanie wykorzystany w nowym systemie pomiarowym projektowanym przez zespół PERG \textit{Beam Position Monitor Digital Back End} \cite{BPMDBM}.
 
  %W pierwszym rozdziale opisany został system pomiarowy detektora GEM oraz charakterystyka i porównanie dostępnych na rynku urządzeń do przetwarzania danych. Następnie przedstawiona została koncepcja projektu akceleratora obliczeniowego, opis realizacji schematów elektrycznych oraz projekt obwodów drukowanych. Kolejne rozdziały zawierają opis wykonanych symulacji integralności sygnałowej i integralności sieci zasilającej oraz  projekt oprogramowania uruchamiającego. 
  
  W pierwszym rozdziale zawarto opis systemu pomiarowego detektora GEM oraz ogólną charakterystykę akceleratorów obliczeniowych wraz z porównaniem dostępnych na rynku urządzeń tego typu. Następnie przedstawiona jest geneza, cele i założenia projektowe. Rozdziały trzeci, czwarty i piąty zawierają opis koncepcji projektu, realizacji schematów elektrycznych oraz projekt obwodów drukowanych. Dalej, w rozdziałach szóstym i siódmym zaprezentowano wyniki wykonanych symulacji integralności sygnałowej i zasilania oraz projekt oprogramowania uruchamiającego. Ostatni rozdział zawiera wnioski końcowe podsumowujące wykonany projekt. Na końcu znajdują się dwa dodatki, pierwszy opisujący pliki projektowe znajdujące się na dołączonej płycie, a drugi zawiera charakterystykę standardów MTCA i AMC.
  
%##################################################################################
\section{System pomiarowy detektora GEM}
Detektor GEM służy do pomiaru promieniowania w eksperymentach fizycznych. Zjonizowane cząstki o określonej energii w przepływającym przez detektor gazie są powielane przez specjalne płyty z otworami, do których przyłożone jest wysokie napięcie. Wewnątrz odbywa się pomiar energii przepływających elektronów. Pomiar promieniowania jest możliwy dzięki wzmocnieniu jakie można uzyskać w detektorze. Przykładem wykorzystania takiego detektora jest eksperyment JET gdzie detektor GEM jest częścią detekcyjną spektrometru X. 

	\begin{figure}[!here]
	\begin{center}
	\includegraphics[width=10cm]{grafika/detectors.png}
	\end{center}
	\caption{Detektor GEM w JET}
	\label{GEM_PHOTO}
	\end{figure}

Do analizy danych z detektora zaprojektowany został system pomiarowy przez zespół PERG z Instytutu Systemów Elektronicznych Wydziału Elektroniki i Technik Informacyjnych Politechniki Warszawskiej. System pomiarowy składa się z:
\begin{itemize}
\item detektora GEM
\item  analogowych kart pomiarowych Analog Front End (AFE)
\item kart FMC z przetwornikami ADC dołączonymi do płyty-matki  (\textit{Carrier-board})
\item płyty głównej w standardzie Mini ITX z procesorem x86 i systemem operacyjnym Linux 
\item jednostki zasilającej
\end{itemize}

	\begin{figure}[!ht]
	\begin{center}
	\includegraphics[width=14cm]{grafika/GEM_block_diagram.png}
	\end{center}
	\caption{Struktura systemu pomiarowego GEM}
	\label{GEMSTRUCTURE}
	\end{figure}


	
System posiada 256 wejściowych kanałów pomiarowych, które zamienia w jeden wyjściowy będący histogramem energii widma fotonów w czasie ustalonym podczas pomiaru. Przetwarzanie danych w systemie pomiarowym GEM odbywa się w kilku etapach. Początkowo dane zebrane przez analogowe karty pomiarowe (AFE) są zamienione na postać cyfrową za pomocą kart FMC z przetwornikami ADC. Na kartach odbywa się również kontrola sygnałów wyzwalających oraz identyfikacja ładunków wykonana w FPGA. Następnie dane z kart są analizowane w układach FPGA znajdujących się na \textit{Carrier Board}, których zadaniem jest integracja histogramów. Ostatnim etapem jest zebranie wszystkich danych pomiarowych z czterech \textit{Carrier Board} i ich finalna integracja w \textit{Backplane}. Tam też odbywa się zapis pomiaru do pamięci DDR. Cały system jest podłączony do komputera PC gdzie dane mogą być diagnozowane za pomocą oprogramowania Matlab firmy Mathworks \cite{MATLAB}.  

Detekcja promieniowania o danej energii odbywa się poprzez pomiar napięcia z detektora. Aby rozróżnić energię fotonów, które nakładają się na siebie, potrzebna jest dokładniejsza analiza sygnału z zastosowaniem cyfrowego przetwarzania sygnałów. Obecny system pozwala na to w ograniczonym zakresie i wiele pomiarów jest odrzucanych. Dlatego zdecydowano się na zaprojektowanie dedykowanego urządzenia, które będzie służyło do przetwarzania i analizy danych pomiarowych z detektora w czasie rzeczywistym. 

\subsection{Nowa wersja systemu}
Obecny system pomiarowy detektora GEM jest bazą do nowego systemu pomiarowego \textit{Beam Position Monitor Digital Back End}. System BPM DBE składać się będzie z kasety MTCA oraz z kart rozszerzeń AMC oraz FMC. Projekt jest umieszczony na ogólnodostępnym repozytorium \textit{Open Hardware Repository} \cite{BPMDBM}. 


	\begin{figure}[!here]
	\begin{center}
	\includegraphics[width=6cm]{grafika/bpm.png}
	\end{center}
	\caption{Wizualizacja kasety MTCA, nowej wersji systemu pomiarowego detektora GEM}
	\label{GEM_BPM}
	\end{figure} 
  
   
    
Karty FMC wykorzystywane w poprzedniej wersji systemu, mogą być wykorzystane w BPM DBE dzieki zaprojektowanemu modułowi AMC FMC Carrier \cite{AFC}, który jest płytą-matką dla kart FMC. Dzięki temu system posiada taką samą funkcjonalność jak poprzedni. Standard AMC staje się coraz bardziej popularny w zastosowaniach fizyki wysokiej energii \cite{MTCA_4} ze względu na niskie koszty i dużą elastyczność systemu. 

Do systemu potrzebne jest urządzenie służące do przetwarzania danych - \textbf{akcelerator obliczeniowy}, który pozwoli na analizę sygnałów czasie rzeczywistym. 

%##################################################################################
\section{Charakterystyka akceleratorów obliczeniowych}

Akcelerator obliczeniowy jest to system wbudowany (\textit{ang. embedded system}) służący przetwarzaniu danych. Charakteryzuje się dużą mocą obliczeniową przy relatywnie niskim poborze mocy oraz dużą przepustowością interfejsów wejść/wyjść. Do akceleratorów obliczeniowych możemy zaliczyć np. karty graficzne czy dedykowane moduły np. w standardzie AMC, FMC lub PCIe posiadające specjalizowane układy przetwarzające DSP lub FPGA. 

Urządzenia tego typu znajdują zastosowanie np. w nowoczesnych systemach pomiarowych fizyki wysokich energii, gdzie analiza sygnałów musi być wykonywana w czasie rzeczywistym. Innym przykładem mogą być szybkie kamery o wysokiej rozdzielczości.

 
\subsection{Układy przetwarzania} % 
Przetwarzanie w akceleratorach obliczeniowych jest wykonywane przez procesory, posiadające odpowiednią architekturę dzięki której można uzyskać bardzo dużą wydajność obliczeń. Poniższej przedstawiono charakterystykę poszczególnych typów układów, które mogą być wykorzystane do przetwarzania danych. 

\subsubsection{Procesory DSP}
Procesory DSP charakteryzują się architekturą zaprojektowaną specjalnie dla przetwarzania sygnałów (stąd też nazwa \textit{Digital Signal Processor}). Dzięki temu potrafią wykonywać skomplikowane operacje matematyczne równolegle. Specjalizowana architektura minimalizuje pobór mocy. Przykładowo procesor DSP firmy Texas Instruments \cite{COMPANY:TEXAS} TMS320C6678 posiada teoretyczną moc obliczeniową 160 GFLOPs przy ok. 10 W TDP (\textit{ ang. Total Dissipated Power}). 

\begin{itemize}
\item Zalety:
\begin{itemize}
\item bardzo korzystny stosunek mocy obliczeniowej do pobieranej energii
\item obsługa wielu szybkich interfejsów szeregowych
\item niski koszt układu
\item niski pobór mocy
\end{itemize}

\item Wady:
\begin{itemize}
\item rozwój oprogramowania jest skomplikowany i długotrwały
\end{itemize}

\end{itemize}

\subsubsection{Układy programowalne FPGA}
Dużą popularnością w systemach przetwarzania sygnałów cieszą się układy programowalne FPGA (\textit{Field Programmable Gate Array}). Największą zaletą tych układów jest możliwość dowolnego łączenia komórek logicznych wewnątrz układu co pozwala na tworzenie zaawansowanych systemów cyfrowych np. filtrów, bloków kryptograficznych itp..  Dodatkowo nowoczesne FPGA obsługują szybkie interfejsy szeregowe \cite{FPGA:GTP}. Rozwój tych układów w ostatnich latach pozwolił na minimalizację pobieranej mocy i znaczne zwiększenie ilości jednostek logicznych dzięki czemu, układy te znajdują coraz szersze zastosowanie w systemach przetwarzania sygnałów. 

\begin{itemize}
\item Zalety:
\begin{itemize}
\item możliwość realizacji dowolnych systemów cyfrowych
\item obsługa szybkich interfejsów szeregowych
\item średni pobór mocy
\end{itemize}

\item Wady:
\begin{itemize}
\item wysoki koszt układu
\item  rozwój oprogramowania jest skomplikowany i długotrwały
\end{itemize}

\end{itemize}

\subsubsection{Układy GPGPU}
GPGPU (\textit{ang. General Purpose Graphics Processing Unit}) są to jednostki obliczeniowe stosowane w kartach graficznych, komputerów klasy PC. Układy te charakteryzują się bardzo dużą wydajnością ze względu na wieloprocesorową architekturę. Na czele zastosowań przetwarzania stoi firma NVIDIA \cite{NVIDIA} z architekturą CUDA \cite{CUDA}, która posiada bardzo dopracowane środowisko do rozwoju oprogramowania (tzw. SDK \textit{Software Developement Kit}).  

\begin{itemize}
\item Zalety:
\begin{itemize}
\item bardzo duża wydajność
\item dopracowane środowisko programistyczne 
\end{itemize}

\item Wady:
\begin{itemize}
\item bardzo duży pobór mocy
\item komunikacja ograniczona do interfejsu PCIe
\item nieopłacalny rozwój własnych urządzeń
\end{itemize}
\end{itemize}

\subsubsection{Układy ASIC}
ASIC są specjalizowanymi układami zaprojektowanymi od podstaw do zastosowania w specyficznych zastosowaniach np. do kompresji wideo. Wykonanie takiego układu jest bardzo kosztowne opłaca się to tylko w wypadku produkcji idącej w tysiącach sztuk. Z drugiej strony dzięki temu układ jest nieporównywalnie szybszy i oszczędniejszy w zużyciu energii od zwykłych procesorów. Rozwój oprogramowania ogranicza się zwykle do konfiguracji rejestrów wewnętrznych, co bardzo skraca czas realizacji projektu. 


\begin{itemize}
\item Zalety:
\begin{itemize}
\item bardzo duża moc obliczeniowa w dedykowanym zastosowaniu np. kompresja wideo
\item mały pobór energii
\item łatwa konfiguracja
\end{itemize}

\item Wady:
\begin{itemize}
\item kosztowna realizacja układu
\end{itemize}

\end{itemize}

\subsubsection{Procesory x86}
Procesory z architekturą x86 są sercem każdego komputera klasy PC. Na rynku istnieją dwie firmy oferujące układy tego typu: Intel \cite{Intel} oraz AMD \cite{AMD}. Największą zaletą tych procesorów jest uniwersalność, dzięki swojej architekturze i superskalarności posiadają bardzo dużą wydajność w większości zastosowań. Dodatkowo układy te są kompatybilne wstecz względem zestawu instrukcji dzięki czemu rozwój oprogramowania jest  łatwy.

Mimo bardzo korzystnego stosunku pobieranej energii do mocy obliczeniowej, układy te bardzo rzadko wykorzystuje się w systemach wbudowanych ze względu na znaczne wymagania mocowe. Przykładowo procesor firmy \textit{Intel} z serii Core i7 odznacza się maksymalną pobieraną mocą na poziomie 140 W TDP (\textit{ang. Total Dissipated Power}). Dodatkowo procesory te często wymagają dodatkowych układów do obsługi szybkich interfejsów szeregowych.  

\begin{itemize}
\item Zalety:
\begin{itemize}
\item bardzo duża wydajność
\item łatwy rozwój oprogramowania.
\end{itemize}

\item Wady:
\begin{itemize}
\item bardzo duży pobór mocy
\item brak bezpośredniej obsługi szybkich interfejsów szeregowych
\item nieopłacalny rozwój własnych urządzeń
\end{itemize}

\end{itemize}


\subsubsection{Porównanie}
Wszystkie układy dedykowane zastosowaniom przetwarzania mają swoje wady i zalety. Wybór konkretnego typu zależy głównie od zastosowania.
Warto podkreślić, że układy DSP i FPGA dzięki swojej architekturze pozwalają na wykonanie tych samych algorytmów równie szybko (bądź szybciej) jak procesory x86 czy GPGPU jednak jest to okupione czasem realizacji oprogramowania. Z drugiej strony często nie ma możliwości wykorzystania procesorów x86 czy GPGPU w systemach wbudowanych ze względu na bardzo duży pobór mocy. Kolejną kwestią jest też nasycenie rynku, zwykle nie opłaca się projektować dedykowanego urządzenia z układami GPGPU, ponieważ na rynku istnieje tak duża ilość dostępnych produktów, że wykonanie konkurencyjnego urządzenia jest bardzo trudne. 
%
%W tabeli [\ref{tbl:cpu_comp}] zamieszczone jest porównanie wcześniej omówionych układów pod względem wydajności, pobieranej mocy, ceny i łatwości programowania. 
%
%\begin{table}[here]
%\centering
%\scriptsize
%    \begin{tabular}{p{2.5cm}| p{2.5cm} |c|c|c|c}
%	\toprule
%    \textbf{Układ} & \textbf{Wydajność} & \textbf{Pobierana moc} & \textbf{Time To Market}  & \textbf{Rozwój oprogramowania}  & \textbf{Kosz}\\
%    \midrule
%    FPGA 	& 	Wysoka		& 	Średnia		&	Długi		&	Trudny			&	Średni\\
%    DSP		& 	Wysoka   		&	Średnia		& 	Długi		&	Trudny			&	Niski\\
%    ASIC 	& 	Bardzo wysoka	& 	Mała			&	Krótki		&	Łatwy				&	Niski\\
%    GPGPU	& 	Bardzo wysoka	& 	Bardzo duża		&	Średni		&	Relatywnie łatwy		&	Średni\\
%    x86		&	Bardzo wysoka	&	Bardzo duża		& 	Średni		&	Łatwy				& 	Wysoki\\
%    \end{tabular}
%	\caption{Porównanie układów dedykowanych przetwarzaniu}
%	\label{tbl:cpu_comp}
%\end{table}
 

\subsection{Interfejsy komunikacyjne}
Bardzo istotnym aspektem w systemach przetwarzania są interfejsy komunikacyjne. Najczęściej spotykanymi interfejsami w systemach wbudowanych są PCI Express, USB, Ethernet oraz RapidIO. Najważniejszym parametrem interfejsów komunikacyjnych jest ich przepustowość, rodzaj transmisji, kontrola błędów w pakietach itp. Istotna jest też kwestia zastosowania, jedne interfejsy zostały zaprojektowane jako sposób komunikacji między układami wewnątrz urządzenia, inne natomiast pozwalają na przesył danych na duże odległości. Poniżej przedstawiono opis najbardziej popularnych interfejsów w systemach wbudowanych.  

\subsubsection{PCI Express} 
PCIe to szybki interfejs szeregowy będący rozwinięciem standardu PCI. Interfejs zapewnia znacznie wyższą przepustowość, zmniejsza ilość połączeń między układami, zapewnia również korekcję błędów transmisji oraz możliwość podłączania urządzeń w czasie pracy (\textit{tzw. hot-plug}). Dane przesyłane są różnicowo za pomocą standardu LVDS z wykorzystaniem kodowania 8b/10b \cite{8B_10B} (w wersji 1.0). Interfejs pozwala na pracę z wykorzystaniem wielu linii (\textit{ang. links}), co pozwala na uzyskanie większej przepustowości. 

 PCIe jest wykorzystywany jako podstawowy interfejs komunikacyjny np. w komputerach PC - między kartą graficzną a chipsetem, jak również w dedykowanych systemach wbudowanych. Zarówno współczesne układy FPGA jak i DSP pozwalają na komunikacje za pomocą tego interfejsu.  
%
%Obecna wersja standardu PCIe to wersja 3.0, kolejna (4.0) jest w trakcie realizacji. PCI Express 3.0 zmienia standard kodowania na 128b/130b i zapewnia transfer na poziomie 985 MB/s na linię.  
%
%Sieć połączeń w standardzie PCIe jest strukturą drzewiastą, gdzie jeden z układów pełni rolę korzenia (\textit{Root Complex}) a pozostałe są do niego bezpośrednio dołączone (\textit{ang. End Point}). 

\subsubsection{Serial Rapid IO}
SRIO jest interfejsem zaprojektowanym specjalnie dla systemów wbudowanych. Zapewnia równie dużą przepustowość jak PCIe i dodatkowo posiada mniej skomplikowany sposób przesyłania sygnałów kontrolnych, co minimalizuje opóźnienia i jitter. Dzięki temu pozwala na przesyłanie sygnałów, które wymagają synchronizacji czasowej bądź nawet pracy w systemach czasu rzeczywistego. 

Interfejs ten korzysta z standardu LVDS i pozwala na uzyskanie 5 Gbps przepustowości na linię. 

\subsubsection{Ethernet}
Ethernet pozwala na przesył danych wykorzystując protokół np. TCP/IP \cite{TCP}. Przepustowość tego interfejsu wynosi od 100 Mbps do 10 Gbps. Istotną kwestią jest kontrola przesyłu pakietów, która jest wykonywana na poziomie sieci (\textit{network layer}), a nie na warstwie fizycznej (\textit{physical layer}). Interfejs ten jest najczęściej wykorzystywany do przesyłu niekrytycznych danych między urządzeniami, często na większą odległość. 


\subsubsection{Inne}
Do innych interfejsów komunikacyjnych wykorzystywanych w systemach wbudowanych możemy zaliczyć interfejsy związane z pamięciami jak SATA czy SAS. Zapewniają one bardzo dużą przepustowość, ale służą głównie do przesyłu danych z dysków twardych, nie wykorzystuje się ich do komunikacji między układami scalonymi wewnątrz systemu. 

\subsection{Standardy urządzeń przetwarzania}
System wbudowany jest zwykle projektowany względem określonego standardu. Określa on specyfikację elektryczno-mechaniczną jakie musi spełniać urządzenie. Przykładem takich standardów są np. VME, AMC, FMC czy ATCA. 

\subsubsection{ATCA}
Standard ATCA (\textit{Advanced Telecommunication Computing Architecture}) specyfikuje systemy przetwarzania wykorzystywane w telekomunikacji, przemyśle i wojsku. System składa się z kart rozszerzeń ATCA (\textit{blade}) i kasety (\textit{shelf}) ATCA łączącej wszystkie moduły za pomocą tzw. \textit{Backplane}. Specyfikacja obejmuje komunikację między modułami, wymagania mechaniczne dla kart rozszerzeń etc. Organizacja PICMG \cite{PICMG}, odpowiedzialna za przygotowanie standardu, zrzesza ponad 100 organizacji na całym świecie. Największą zaletą tego systemu jest jego modułowość, w kracie można umieścić bardzo zróżnicowane moduły dzięki czemu możliwa jest realizacja systemu dopasowanego do potrzeb. 


 	\begin{figure}[!ht]
	\begin{center}
	\includegraphics[width=4cm]{grafika/blade.jpg}
	\end{center}
	\caption{Moduł ATCA wyposażony w dwa procesory Intel Xeon}
	\end{figure}
	 

\subsubsection{AMC}
Standard AMC (\textit{Advanced Mezzanine Card}) specyfikuje moduły rozszerzeń do systemów MTCA \cite{MICRO_TCA} i ATCA \cite{ATCA}. Powstał on w celu zmniejszenia kosztów w porównaniu do systemów ATCA przy zachowaniu porównywalnej funkcjonalności.  Standard opisano szczegółowo w Dodatku B  [\ref{AMC_APP}]. 

\subsubsection{FMC}
Moduły FMC (\textit{FPGA Mezzanine Card}) znajdują szerokie zastosowanie jako karty rozszerzeń do płyt-matek (\textit{carrier-board}). Pozwalają na rozszerzenie funkcjonalności działającego systemu bez potrzeby projektowania go od początku. Dzięki małym wymiarom i nieskomplikowanej  specyfikacji standardu wykonanie urządzeń jest stosunkowo łatwe. FMC wyróżnia się specjalnym, wielopinowym złączem które pozwala na transmisję z zastosowaniem szybkich interfejsów szeregowych. 

	\begin{figure}[!ht]
	\begin{center}
	\includegraphics[width=4cm]{grafika/fmc1.png}
	\end{center}
	\caption{Karta FMC firmy Creotech Instruments SA}
	\end{figure}
	

\section{Przegląd urządzeń dostępnych na rynku}
Obecnie dostępne jest wiele urządzeń, które można wykorzystać w systemie pomiarowym jako akceleratory numeryczne do przetwarzania danych. Poniżej przedstawiono charakterystyki kilku takich urządzeń dostępnych na rynku.

\paragraph {AMC V7-2c6678 firmy \textit{CommAgility}}
AMC-V7-C6678 \cite{COMMAGILITY} jest kartą przetwarzania danych zawierająca dwa procesory DSP TMS320C6678 i układ FPGA Xilinx XC7VX415T-2 zamkniętą w formacie AMC pojedynczej szerokości i pełnej wysokości. Posiada złącze SFP+ oraz złącze miniSAS. Procesory DSP, układ FPGA, złącza na panelu oraz złącze krawędziowe są połączone ze sobą siecią połączeń interfejsu SRIO. Koszt tego urządzenia to 16 tys. EUR. 

	\begin{figure}[!ht]
	\begin{center}
	\includegraphics[width=5cm]{grafika/amc_commagility.jpg}
	\end{center}
	\caption{Akcelerator obliczeń firmy Commagility}
	\label{COMMAGILITY}
	\end{figure}

\paragraph{FMC6678}
Firma 4DSP specjalizuje się w projektowaniu modułów FMC i posiada w swojej ofercie kartę z jednym procesorem TMS320C6678 i 1 GB pamięci dynamicznej DDR3. Na module znajdziemy również gniazdo Gigabit Ethernet, USB oraz pełne 60-pinowe złącze \textit{Trace} do programowania procesora poprzez dedykowany debugger. Interfejsy komunikacyjne SRIO i PCIe są dołączone do złącza FMC. Koszt modułu FMC to 2900 EUR.

	\begin{figure}[!ht]
	\begin{center}
	\includegraphics[width=5cm]{grafika/FMC667.png}
	\end{center}
	\caption{Akcelerator obliczeń firmy 4DSP w formacie FMC}
	\label{4DSP}
	\end{figure}
	
	


\paragraph {Urządzenia firmy ADVANTECH} 
Firma Advantech jest twórcą modułu ewaluacyjnego do procesora TMS320C6678 i posiada w swojej ofercie szereg urządzeń z tym procesorem. Przykładem jest 8 procesorowa karta PCIe DSPC-8682 (64 rdzenie) albo karta ACTA z 20 procesorami (80 rdzeni).

	\begin{figure}[!ht]
	\begin{center}
	\includegraphics[width=4cm]{grafika/advantech20dsp.jpg}
	\end{center}
	\caption{Akcelerator obliczeń firmy Advantech}
	\label{Advantech20DSP}
	\end{figure}

\paragraph {PDAK2H}
Moduł PDAK2H jest modułem AMC z najnowszymi układami SoC firmy \textit{Texas Instruments} z rodziny Keystone II zawierającymi 8 rdzeni DSP C66x oraz czterordzeniowy procesor ARM Cortex-A15. Twórcą modułu jest firma Prodrive. Do procesora może być dołączone do 26 GB pamięci DDR3-1333. Do złącza AMC doprowadzone jest 12 linii SRIO a na przednim panelu zamontowano złącza Gigabit Ethernet oraz UART.  
	\begin{figure}[!ht]
	\begin{center}
	\includegraphics[width=5cm]{grafika/pdak2h.jpg}
	\end{center}
	\caption{Akcelerator obliczeń firmy Prodrive}
	\label{PRODRIVE}
	\end{figure}

\paragraph{Porównanie}	
W tabeli [\ref{tbl:cots}] zestawiono wymienione wcześniej moduły wraz z porównaniem peryferiów i ceny.
\begin{table}[here]
\centering
\scriptsize
	\caption{Zestawienie komercyjnych systemów przetwarzania}
    \begin{tabular}{p{2.5cm}| p{2.5cm} |c|c|c|c}
	\toprule
    \textbf{Urządzenie} & \textbf{Procesory} & \textbf{Standard} & \textbf{Złącza}  & \textbf{Interfejsy MCH}  & \textbf{Cena}\\
    \midrule
    AMC V7-2C6678 	& 	Xilinx XC7VX415T-2  2xTMS320C6678  	& 	AMC Single width, full size		&	SFP, SAS	&	 	SRIO				&	16 000 USD\\
    FMC6678		& 	TMS320C6678  				& 	FMC single width			&	GbE		&	 	SRIO,PCIe, EMIF16		&	2900  EUR\\
    DSPC-8682 	& 	TMS320C6678, Xilinx XC3S200AN  	& 	PCIe					&	SFP, SAS	&	 	GbE				&	BD\\
    PDAK2H 		& 	66AK2H12				  	& 	AMC Single width, full size		&	GbE, UART	&	 	SRIO, PCIe			&	BD\\
    \end{tabular}

	\label{tbl:cots}
\end{table}

\section{Podsumowanie} 

Ilość rodzajów układów scalonych, interfejsów komunikacyjnych oraz standardów dla systemów wbudowanych daje duże możliwości instytucjom tworzącym takie urządzenia. Dzięki szybkiemu rozwojowi, możliwe jest zaprojektowanie akceleratorów obliczeniowych posiadających bardzo dużą moc obliczeniową oraz przepustowość akwizycji danych dopasowaną do konkretnego zastosowania. Szczególna uwaga należy się urządzeniom w standardzie AMC i FMC, które pozwalają na stosunkowo łatwe rozszerzenie funkcjonalności systemów opartych o standard MTCA. 

\subsection{Wymagania funkcjonalne akceleratora obliczeniowego}
Urządzenie do przetwarzania danych, które ma pracować w systemie \textit{BPM DBE} musi spełniać określone wymagania oraz posiadać funkcjonalności, które pozwolą na analizę danych pomiarowych. Akcelerator obliczeń powinien być kartą zgodną ze standardem AMC i posiadać układy o dużej mocy obliczeniowej w zastosowaniach przetwarzania sygnałów - FPGA i DSP. Zastosowanie dwóch typów układów pozwoli na elastyczne podejście do analizy danych. Ponadto, ze względu na to, iż system generuje bardzo dużą ilość danych (obecnie 1 Gbps na kanał) istotne jest aby akcelerator obliczeniowy posiadał dużą ilość pamięci operacyjnej. 

Zgodnie ze standardem MTCA, karty AMC komunikują się z kasetą za pomocą interfejsów komunikacyjnych np. SRIO i PCIe. Dodatkowo w systemie \textit{BPM DBE} zastosowano złącza miniSAS do komunikacji na większe odległości, dlatego projektowane urządzenie powinno posiadać możliwość komunikacji również za pomocą tego interfejsu.  Kolejnym, istotnym kryterium jest możliwość rekonfiguracji połączeń między układami. Ta funkcjonalność wydłuża czas życia urządzenia (\textit{ang. End Of Life}) ze względu na to, iż pozwala na dopasowanie właściwości do danego zastosowania. 

Na podstawie analizy dostępnych urządzeń (COTS - Commercial of the shelf) zdecydowano się na projekt własnego modułu do przetwarzania obliczeń. Głównym powodem jest minimalizacja kosztów oraz zwiększenie funkcjonalności w porównaniu do urządzeń oferowanych na rynku.