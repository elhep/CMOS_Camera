\section{Wstęp}
Poniższy rozdział zawiera opis projektu obwodów drukowanych akceleratora obliczeniowego. Pierwsza część zawiera opis wymagań i ograniczeń producenta, według których ustalone zostały reguły projektowe. Kolejne części zawierają proces projektowania PCB, na co składa się projekt warstw, rozmieszenia elementów oraz opis zasad prowadzenia ścieżek.

\section{Wymagania producenta}
Produkcję obwodów drukowanych zlecono firmie \textit{Brandner PCB} \cite{BRANDNER} z Estonii. Firma ta pozwala na szybką produkcję zaawansowanych projektów PCB. Strona producenta zawiera zestawienie możliwości produkcyjnych \cite{BRANDNER_PRODUCTION}. 


Względem tych zasad ustalone zostały podstawowe reguły projektowe:

\begin{table}[h]
\centering
	\caption{Podstawowe reguły projektowe PCB}
    \begin{tabular}{p{7cm} | p{2cm} |  p{2.5cm}}
	\toprule
    \textbf{Wymaganie} & \textbf{Min [mm]} & \textbf{Maks. [mm]}\\
    \midrule
    Odstęp między ścieżkami 		& 	0.1		&			\\
    Wymiary przelotki (otwór/średnica)	& 	0.2/0.4	&	0.4/0.8	\\
    Odległość między przelotkami 		& 	0.1		&			\\
    Odległość otworów od polygonów 	& 	0.1		&			\\
    Szerokość ścieżek			&	0.08		&	10		\\
    Odległość ścieżek od krawędzi		&	0.2		&			\\
	\toprule
    \end{tabular}

	\label{tbl:pcb_design_rules}
\end{table}

Szczegółowe zasady projektowe zostały wprowadzone do programu \textit{Altium Designer} w zakładce \textit{Design/Rules}.

\section{Oprogramowanie EDA}
Programem wykorzystanym do zaprojektowania obwodów drukowanych jest \textit{Altium Designer} \cite{ALTIUM}. Program ten jest zintegrowanym środowiskiem pozwalającym na tworzenie schematów, obwodów drukowanych, a nawet posiada takie funkcje jak możliwość programowania układów FPGA.  
\section{Projekt warstw}
Ze względu na znaczną ilość linii zasilających oraz ilość układów scalonych projekt akceleratora obliczeniowego wykonano na 16 warstwowym obwodzie drukowanym. Wykorzystano standardowy stackup producenta oparty o materiał FR4 o stałej dielektrycznej $\epsilon = 4.6$. 


\begin{figure}[!ht]
\centering
\includegraphics[width=14cm]{grafika/brandner.pdf}
\caption{Standardowa konfiguracja warstw dla obwodu 16 warstwowego firmy Brandner}
\end{figure}

Dobór ilości warstw jest szeroko opisany w profesjonalnej literaturze dotyczącej projektowania szybkich systemów cyfrowych \cite{BOOK:HIGH_SPEED} \cite{BOOK:BLACK_MAGIC} jak również w poradnikach tzw. \textit{User Guides} udostępnianych przez producentów układów scalonych \cite{UG_STACKUP1} \cite{UG_STACKUP2}. Zdecydowano się na obwód 16 warstwowy z szeregu powodów:

\begin{itemize}
\item
duża ilość szybkich interfejsów szeregowych - nie można prowadzić takich ścieżek pod/nad rozdzielonymi warstwami  (\textit{split planes})
\item
9 głównych linii zasilających
\item
duże upakowanie elementów
\item
rezygnacja z wykorzystania ślepych i zagrzebanych przelotek
\end{itemize}

Zalecaną praktyką jest, aby każda linia zasilania była prowadzona na oddzielnej warstwie. Z punktu widzenia integralności sygnałowej jest to poprawna praktyka, jednak znacznie zwiększa koszty produkcji urządzenia. Dlatego konieczne było rozdzielenie dwóch warstw zasilających, aby poprowadzić wszystkie napięcia zasilające. Ważnym aspektem jest ścieżka powrotna sygnałów, która również może być podatna na zakłócenia. Zastosowana konfiguracja warstw udostępnia każdej warstwie sygnałowej przynajmniej jedną pełną warstwę referencyjną.  

\begin{figure}[!ht]
\centering
\includegraphics[width=12cm]{grafika/stackup_16.png}
\caption{Konfiguracja warstw akceleratora obliczeniowego}
\end{figure}

\section{Rozmieszczenie elementów}
Krytycznym aspektem projektowania urządzeń elektronicznych jest poprawne rozmieszczenie elementów na obwodzie drukowanym. W akceleratorze obliczeniowym zastosowano ogólnie znane praktyki rozmieszczania elementów jak m. in. :
\begin{itemize}
\item
skupienie układów zasilających w jednym miejscu
\item
zapewnienie jak najkrótszej ścieżki powrotnej dla sygnałów
\item
minimalizacja długości połączeń
\end{itemize}

\begin{figure}[!ht]
\centering
\includegraphics[width=12cm]{grafika/pcb_layout.png}
\caption{Podział części analogowej i cyfrowej akceleratora obliczeniowego}
\end{figure}

Główną trudnością podczas projektowania było odpowiednie rozmieszczenie elementów aby zapewnić optymalne połączenia szybkich interfejsów szeregowych między układami. To one głównie decydowały o rozmieszczeniu elementów, dla przykładu: układ przełącznika SRIO ADN4604 umieszczono na drugiej stronie PCB aby możliwe było poprowadzenie ścieżek interfejsu Hyperlink między procesorami DSP. Natomiast przełącznik PCIe umieszczono blisko złącza krawędziowego AMC, ponieważ umieszczenie go pomiędzy procesorami DSP i FPGA spowodowałoby znaczne utrudnienia w prowadzeniu połączeń interfejsu SRIO wcześniej wspomnianego przełącznika. Kolejną kwestią jest dystrybucja sygnałów zegarowych; układy je generujące powinny znajdować się jak najbliżej układu docelowego co starano się wykonać w projekcie. Końcowy efekt jest zadowalający i wszystkie interfejsy spełniają wymagania producenta. 

\section{Zasady prowadzenia ścieżek szybkich interfejsów}

Poniższa część zawiera zasady, którymi kierowano się przy prowadzeniu połączeń szybkich szeregowych interfejsów takich jak: \textit{PCI Express}, \textit{SRIO}, \textit{Hyperlink} oraz sygnałów zegarowych. Sygnały płynące w obwodach drukowanych mają prędkość mniejszą od prędkości światła w przybliżeniu o połowę, dla typowego laminatu FR4. Zgodnie ze wzorem:
$$ v = \frac{c}{\sqrt{\epsilon_{r}}} = \frac{3 \cdot 10^8}{\sqrt{4.6}}  \approx 14{} {cm}/_{ns}  $$

Dokumentacja producentów układów zawiera zasady prowadzenia poszczególnych interfejsów \cite{DSP_SERDES} \cite{DOCS:HIGH_SPEED_DSP} \cite{DOCS:HIGH_SPEED_LAYOUT}. Dopasowanie par różnicowych i linii opiera się o czas propagacji sygnału w linii transmisyjnej jaką jest ścieżka w obwodzie drukowanym przy dużej prędkości transmisji, dlatego określa się je często w \textit{pikosekundach}. 

\begin{table}[h]
\scriptsize
\centering
	\caption{Reguły prowadzenia ścieżek szybkich interfejsów szeregowych}
    \begin{tabular}{c | p{2cm} |  p{2cm} | p{2cm} | p{2cm} | p{2cm} | p{2cm}}
	\toprule
    \textbf{Interfejs} & \textbf{Dopas. P/N min. [ps]} & \textbf{Dopas. P/N min. [mm]} & \textbf{Dopas. P/N maks. [mm]} & \textbf{Dopas. TXn/RXn [ps]} & \textbf{Dopas. TXn/RXn min. [mm]}  & \textbf{Dopas. TXn/RXn maks. [mm]}\\
    \midrule
    SRIO 	& 	5	&	0.7	&	0.9	&	10	& 	1.38	&	1.8\\
    PCIe 	& 	1	&	0.14	&	0.18	&	5	& 	0.7	&	 0.9\\
    Hyperlink 	& 	1	&	0.14	&	0.18	&	100	& 	13	&	18\\
    SGMII 	& 	5	&	0.7	&	0.9	&	5	& 	0.7	&	 0.9\\
	\toprule
    \end{tabular}

	\label{tbl:serdes_rules}
\end{table}

%Zależności długości linii par różnicowych oraz między samymi parami w dokumentacji producentów są opisane w \textit{mils}. Przygotowana została tabela konwertująca te wartości.
%
%
%\begin{table}[h]
%\centering
%    \begin{tabular}{c | c | c | c | c | c }
%	\toprule
%    \textbf{ps} & \textbf{mils [min]} & \textbf{mils [max]} & \textbf{mm [min]} & \textbf{[mm [max]}   \\
%    \midrule
%    1 		& 	5.464		&	7.092	&	0.9	&	10\\
%    5 		& 	27.32		&	35.46	&	0.18	&	5\\		
%    10 		& 	54.64		&	70.92	&	0.18	&	100\\
%    100 	& 	546.4		&	0.7	&	0.9	&	5\\
%	\toprule
%    \end{tabular}
%	\caption{Konwersja ps - mils - mm}
%	\label{tbl:ps}
%\end{table}


\subsection{Dobór szerokości ścieżek}
Kolejnym kluczowym elementem przy prowadzeniu połączeń szybkich szeregowych interfejsów cyfrowych jest impedancja połączeń. Szybkie interfejsy szeregowe wymagają odpowiedniej wartości terminacji przy odbiorniku (która zwykle jest wbudowana w układ). Najbardziej popularną wartością jest  $100\Omega$ lub np. w przypadku pamięci DDR3 $80 \Omega$. Z tego powodu wymagane jest aby \textit{linia transmisyjna} którą jest ścieżka różnicowa miała odpowiednią impedancję charakterystyczną (różnicową \textit{differential} oraz pojedynczą \textit{single ended}). Do wyznaczenia odpowiedniej szerokości ścieżek posłużono się darmowym programem \textit{Saturn PCB Toolkit} \cite{SATURN}.  

\begin{figure}[!ht]
\centering
\includegraphics[width=12cm]{grafika/saturn.png}
\caption{Program Saturn PCB Toolkit}
\end{figure}



Tabela [\ref{tbl:serdes_width}] przedstawia wyliczone wartości szerokości oraz odległości między parami różnicowymi dla interfejsów o impedancji różnicowej $100 \Omega$. Przygotowane zostały cztery tabele, dwie dla impedancji $100 \Omega$ oraz dwie dla impedancji $80 \Omega$, pierwsza przedstawia standardowe wartości szerokości oraz odległości a druga minimalne. Minimalne wartości są szczególnie przydatne, gdy konieczne jest poprowadzenie pary różnicowej np. pod układem w obudowie BGA.



\begin{sidewaystable}[h]
\centering
    \scriptsize
  %  \begin{tabular}{p{2cm}  | p{2.5cm} |  p{2.8cm} | p{2.5cm} | p{2.5cm} | p{2.5cm} | p{2.5cm}  |  p{2.8cm}  | p{2.5cm}}
  %  
	\caption{Szerokości ścieżek szybkich interfejsów szeregowych dla impedancji $100 \Omega$}
 \begin{tabular}{c  | c |  c | c |c | c | c  | c  | c}
\hline
    \textbf{Warstwa} & \textbf{Rodzaj} & \textbf{Grubość miedzi [$\mu m$]} & \textbf{Wys. H1 [mm]} & \textbf{Wys. H2 [mm]} & \textbf{Szer. ścieżki [mm]}  & \textbf{Odległość [mm]}   & \textbf{Impedancja różn. [$\Omega$]}   & \textbf{Impedancje poj. [$\Omega$]}\\

    \hline
    \hline
    TOP 	& 	Edge cpl. ext 		&	18	&	0.11	&	0	& 	0.15	&	0.3	&	98.562	&	51.06	\\
    GND 	& 	Plane		 		&	18	&		&	0	& 		&		&			&		\\	
    L1 		& 	Edge Cpld Int Asym 	&	18	&	0.1	&	0.11	& 	0.08	&	0.16	&	98.536	&	51.8	\\
    POWER 	& 	Plane		 		&	18	&		&	0	& 		&		&			&		\\
    L2		& 	Edge Cpld Int Asym 	&	18	&	0.1	&	0.11	& 	0.08	&	0.16	&	98.536	&	51.8	\\
    POWER 	& 	Plane		 		&	18	&		&	0	& 		&		&			&		\\
    L3		& 	Edge Cpld Int Asym 	&	18	&	0.1	&	0.11	& 	0.08	&	0.16	&	98.536	&	51.8	\\
    POWER 	& 	Plane		 		&	18	&		&	0	& 		&		&			&		\\
    L4		& 	Edge Cpld Int Asym 	&	18	&	0.1	&	0.11	& 	0.08	&	0.16	&	98.536	&	51.8	\\
    POWER 	& 	Plane		 		&	18	&		&	0	& 		&		&			&		\\
    L5 		& 	Edge Cpld Int Asym 	&	18	&	0.1	&	0.11	& 	0.08	&	0.16	&	98.536	&	51.8	\\
    POWER 	& 	Plane		 		&	18	&		&	0	& 		&		&			&		\\
    L6 		& 	Edge Cpld Int Asym 	&	18	&	0.1	&	0.11	& 	0.08	&	0.16	&	98.536	&	51.8	\\
    GND 	& 	Plane		 		&	18	&		&	0	& 		&		&			&		\\
    BTM 	& 	Edge cpl. ext 		&	18	&	0.11	&	0	& 	0.15	&	0.3	&	98.562	&	51.06	\\

    
	\toprule
    \end{tabular}

	\label{tbl:serdes_width}
%\end{sidewaystable}
%
%\begin{sidewaystable}[h]

    \scriptsize
%    \begin{tabular}{p{2cm}  | p{2.5cm} |  p{2.8cm} | p{2.5cm} | p{2.5cm} | p{2.5cm} | p{2.5cm}  |  p{2.8cm}  | p{2.5cm}}
	\caption{Szerokości ścieżek szybkich interfejsów szeregowych dla impedancji $80 \Omega$}
 \begin{tabular}{c  | c |  c | c |c | c | c  | c  | c}
\hline
    \textbf{Warstwa} & \textbf{Rodzaj} & \textbf{Grubość miedzi [$\mu m$]} & \textbf{Wys. H1 [mm]} & \textbf{Wys. H2 [mm]} & \textbf{Szer. ścieżki [mm]}  & \textbf{Odległość [mm]}   & \textbf{Impedancja różn. [$\Omega$]}   & \textbf{Impedancje poj. [$\Omega$]}\\

    \hline
    \hline
    TOP 	& 	Edge cpl. ext 		&	18	&	0.11	&	0	& 	0.22	&	0.33	&	78.202	&	40.184	\\
    GND 	& 	Plane		 		&	18	&		&	0	& 		&		&			&			\\	
    L1 		& 	Edge Cpld Int Asym 	&	18	&	0.1	&	0.11	& 	0.13	&	0.26	&	80.53		&	40.826	\\
    POWER 	& 	Plane		 		&	18	&		&	0	& 		&		&			&			\\
    L2		& 	Edge Cpld Int Asym 	&	18	&	0.1	&	0.11	& 	0.13	&	0.26	&	80.53		&	40.826	\\
    POWER 	& 	Plane		 		&	18	&		&	0	& 		&		&			&			\\
    L3		& 	Edge Cpld Int Asym 	&	18	&	0.1	&	0.11	& 	0.13	&	0.26	&	80.53		&	40.826	\\
    POWER 	& 	Plane		 		&	18	&		&	0	& 		&		&			&			\\
    L4		& 	Edge Cpld Int Asym 	&	18	&	0.1	&	0.11	& 	0.13	&	0.26	&	80.53		&	40.826	\\
    POWER 	& 	Plane		 		&	18	&		&	0	& 		&		&			&			\\
    L5 		& 	Edge Cpld Int Asym 	&	18	&	0.1	&	0.11	& 	0.13	&	0.26	&	80.53		&	40.826	\\
    POWER 	& 	Plane		 		&	18	&		&	0	& 		&		&			&			\\
    L6 		& 	Edge Cpld Int Asym 	&	18	&	0.1	&	0.11	& 	0.13	&	0.26	&	80.53		&	40.826	\\
    GND 	& 	Plane		 		&	18	&		&	0	& 		&		&			&			\\
    BTM 	& 	Edge cpl. ext 		&	18	&	0.11	&	0	& 	0.22	&	0.33	&	78.202	&	40.184	\\
    
	\toprule
    \end{tabular}

\end{sidewaystable}

Pozostałe tabele z wyliczeniami szerokości ścieżek zostały dołączone w na płycie CD [\ref{CDROM}], w pliku \textbf{AMC\_DSP\_High\_Speed\_Interfaces.xls}. 

%\section{Symulacje}

\section{Podsumowanie}
Powyższy rozdział opisuje najważniejsze reguły prowadzenia ścieżek na akceleratorze. Zastosowanie tych reguł pozwala zakładać, iż interfejsy pomiędzy układami będą pracować poprawnie. Do pełnej analizy sygnałowej wymagane są jednak symulacje elektromagnetyczne.
