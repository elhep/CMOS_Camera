% TeX encoding = utf8
% TeX spellcheck = pl_PL

%Uruchomienie i testowanie

\section{Wstęp}
Poniższy rozdział przedstawia proces uruchomienia i testów elektrycznych zaprojektowanego urządzenia. Zawarto w nim opis oprogramowania uruchamiającego napisanego na mikrokontroler LPC1764 oraz przebieg testów uruchomieniowych poszczególnych układów.

\section{IPMI}

\section{Uruchomienie zasilania}

Układ LPC1769 ma za zadanie uruchomić poszczególne linie zasilania poprzez wysterowanie linii \textbf{enable} przetwornic i sprawdzenie linii \textbf{Power OK}. Procesory DSP jak i układ FPGA wymagają określonej sekwencji uruchamiania linii zasilających. Aby uruchomić poprawnie zarówno DSP jak i FPGA sekwencja zasilania przebiega według scenariusza \textit{Core before IO} \cite[page=129]{TMS}. 

\begin{center}
% Define block styles
\tikzstyle{decision} = [diamond, draw, fill=blue!20, 
    text width=4.5em, text badly centered, node distance=3cm, inner sep=0pt]
\tikzstyle{block} = [rectangle, draw, fill=blue!20, 
    text width=5em, text centered, rounded corners, minimum height=3em]
\tikzstyle{line} = [draw, -latex']
\tikzstyle{cloud} = [draw, ellipse,fill=red!20, node distance=3cm,
    minimum height=2em]
    
\begin{tikzpicture}[node distance = 2cm, auto]
    % Place nodes
    \node [block] (cvdd0_en) {Enable DSP0 CVDD};
    \node [cloud, left of=cvdd0_en] (section_init) {Init Power};
    
    \node [block,below of=cvdd0_en,] (cvdd1_en) {Enable DSP1 CVDD};
    
    \node [block,below of=cvdd1_en,] (p1v0_en) {Enable P1V0};    
    
     \node [block,below of=p1v0_en,] (p1v2_en) {Enable P1V2}; 
         
     \node [block,below of=p1v2_en,] (p1v8_en) {Enable P1V8};          
    
     \node [block,below of=p1v8_en,] (p3v3_en) {Enable P3V3};         
       
     \node [block,below of=p3v3_en,] (p1v5_en) {Enable P1V5}; 
     \node [decision, below of=p1v5_en,node distance=3cm] (p1v5_ok) {is power good?};
      
      \node [block, right of=p1v5_ok,node distance=4cm,fill=red!20] (error) {Error};  
      
      \node [block,fill=green!20,below of=p1v5_ok,node distance=3cm] (init_comp) {Init complete}; 
       
       \node [block, right of=error,node distance=3cm,fill=red!20] (reset) {Reset};  
    % Draw edges
    \path [line] (section_init) -- (cvdd0_en);
    
    \path [line] (cvdd0_en) -- (cvdd1_en);
    
    \path [line] (cvdd1_en) -- (p1v0_en);
    
    \path [line] (p1v0_en) -- (p1v2_en);
    
    \path [line] (p1v2_en) -- (p1v8_en);	
    
    \path [line] (p1v8_en) -- (p3v3_en);	
    
    \path [line] (p3v3_en) -- (p1v5_en);	
    
    \path [line] (p1v5_en) -- (p1v5_ok);
    
    \path [line] (p1v5_ok) -- node [near start] {no} (error);
    
    \path [line] (p1v5_ok) -- node [near start]{yes}(init_comp);
    
    \path [line] (error) -- (reset);	
         
         
\end{tikzpicture}

\end{center}

Napisane oprogramowanie wykonuje powyższą funkcjonalność. Sygnały Power Good linii 1.0V, 1.2V, 1.5V, 1.8V i 3.3V są zwarte ze sobą. 


\section{Konfiguracja układów}

\subsection{UCD9222}
\subsection{CDCM6208}
\subsection{AD9522}
\subsection{ADN4604}
\subsection{CDCUN1208}

\section{Oprogramowanie FPGA}
W tej części zawarto opis oprogramowania na układ XC7A200T, które służy do poprawnego uruchomienia procesorów DSP.
\subsection{Założenia}
\subsection{Opis oprogramowania}

\section{Uruchomienie DSP}

\section{Testy elektryczne}

\section{Podsumowanie}





