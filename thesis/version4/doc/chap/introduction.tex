

Camera design technology exists since the discovery of bla bla bla
The following thesis is divided into 6 chapters. The first chapter describes thescientific camera design, the theory and
current state of technology and alsothe reasons for doing this work as well as the basic requirements. Secondchapter
presents the concept phase of the project, where the informationregarding possible ways of achieving the requirements is
presented. After theconcept chapter, the realisation is described in chapter 4.

\section{Camera design theory}

    This section introduces the topic of scientific camera design. Reasons to do this work are presented as well as 
    the today's state of knowledge regarding camera design.  

    \subsection{Camera parameters}

      \subsubsection{Frames per second}
      $ fps = \frac{n}{T} $ 
    \subsection{Mechanics}
      \subsubsection{Chasis}
      \subsubsection{Shutter}
      \subsubsection{Optics}
         \paragraph{Lens parameters}
         \paragraph{Mounts}

    \subsection{Electronics}
       \subsubsection{Main control units}
         \paragraph{FPGA}
         \paragraph{SoC}
         \paragraph{Application processor}
         \paragraph{ASICs}
         \paragraph{Microcontroller}

     \subsubsection{Video data storage}
         \paragraph{SSD}
         \paragraph{HDD}
         \paragraph{SD Card}
         \paragraph{Flash}

    \subsection{Firmware}
      Firmware in a camera is responsible for numerous functions of the device. Most cameras incorporate application
      processors, SoC or more advanced microcontrollers which run program usually written in C. Those who doesn't are usually
      specialised cameras which use ASIC to acquire an image and control the functions of the device. Frequently, firmware
      is highly complicated and incorporate digital system (HDL), low level drivers, high level operating systems like
      embedded Linux, Android or Windows CE.  

       \subsubsection{Embedded Linux camera specific firmware}
         Linux operating system provides many tools and programs which can be used in a camera.
        
         Embedded camera systems which runs Linux operating system can benefit from a number of libraries available for
         this Operating System. 

         The following list provides some of the most popular programs. 
       
         \begin{itemize}
           \item video4linux
             %TODO add description
           \item gstreamer 
             %TODO add description
           \item ohc/evcusb 
             %TODO add description
           \item Qt
             %TODO add description
           \item Wayland 
             %TODO add description
           \item OpenCV 
             %TODO add description
          \end{itemize}


  %  \subsection{Software}
  %    Usually with the camera as a system for acquiring a video there is some way to control it using a PC or any other
  %    device like a smartphone for example. 

    \subsection{Sensors}
       \subsubsection{CMOS}
          %TODO add description
       \subsubsection{CCD}
          %TODO add description
       \subsubsection{Other}
     
       \subsection{Interface}
          \subsubsection{Wire based interfaces}
            \paragraph{USB}
                %TODO add description
            \paragraph{Camera Link}
                %TODO add description
            \paragraph{ThunderBolt}
                %TODO add description
            \paragraph{FireWire}
                %TODO add description
            \paragraph{Ethernet}
                %TODO add description
          \subsubsection{Wireless}
            \paragraph{Bluetooth}
                %TODO add description
            \paragraph{WiFi}
                %TODO add description
          \subsubsection{Optical}

  
 
  \section{Scientific camera design theory}
  
  \subsection{Features}
            \paragraph{Extremely low noise}
                %TODO add description
            \paragraph{Binning}
                %TODO add description
            \paragraph{Throughput}
                %TODO add description
            \paragraph{Ethernet}
                %TODO add description

    \subsubsection{Existing devices on the market}
      \paragraph{Elphel camera}
          %TODO add description
      \paragraph{Creotech K20 Camera}
          %TODO add description
      \paragraph{IoT camera}
          %TODO add description
      \paragraph{Other}
          %TODO add description
  
  \subsection{Summary}

\section{Reasons to do this project}
          %TODO add description
 Universal open, camera design architecture 
One cannot deny the fact that, novelty is the  in today's hightech electronics market the most important aspect of the
product is time-to-market. Sometimes even a worse product can win on the market just due to the fact that it was
released eariler.  In specialised markets where COTS (Custom - Off - The - Shelf) components are being used by
customers, the possibility to quickly adjust the parameters of a specific system given suited to the customers need is a
key to success.  Another

Scientific applications for camera systems are, undoubtedly broad.

\section{Requirements}
          %TODO add description
The following list provides basic parameters for the designed camera framework.    
high processing performance - for support of high resolutions
ease of adding a support for a different sensor
Secure and versatile OS
RTOS capability
Future proof
high speed communication - to send high amounts of data live
multichannel operation - astronomical as well as medical applications require it
As described in previous sections, mechanics, casing, envirionment related design choices are different for
each camera project. This is why this work doesn't consist of design of optics and other mechanical aspects of
camera design. This project goal is to address the intersection of all camera problems provide a framework that
will allow for faster development.

\section{Summary}


