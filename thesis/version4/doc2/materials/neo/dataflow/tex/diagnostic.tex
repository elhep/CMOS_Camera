\subsection{Overview}
Camera provides various diagnostic possibilities both for health monitoring and calibration purposes. In current section diagnostic routines and physical quantities measured are described. Most of the diagnostic is oriented towards remote control through ethernet. Apart from that NEOSTEL camera is equipped with two UART channels which provide direct access to console of operating system.

\subsection{Submodules self-tests}

\subsubsection{CCD cooling}

\begin{description}
\item \textbf{Diagnostics} \hfill \\
Parameters monitored: temperature on cold and hot side of Peltier plate and current. In case of current or temperature exceeding normal operation limits, error is asserted and module is shut down. CCD cooling subsystem target temperature can be fully controlled within operational limits and module can by disabled on user request. 

\item \textbf{Self-test} \hfill \\
Self-test checks cooling rate and current in specified operating conditions. If any of monitored parameters exceeds pre-set limits, error is generated.
\end{description}

\subsubsection{Shutter}

\begin{description}

\item \textbf{Self-test} \hfill \\
Shutter provides following sensor array:
\begin{itemize}
\item accelerometer
\item inductor coil current sensor
\item blade position sensor
\end{itemize}

Test procedure (for each blade; initial position of shutter is closed):

\begin{enumerate}
\item open shutter
\item during opening measure coil current, vibration, blade position, opening time
\item close shutter
\item during closing measure coil current, vibration, blade position, closing time
\item verify gathered measurements against preset limits
\item if limits are exceeded, then set error
\end{enumerate}

Diagnostic self-test can be invoked on demand by command between Zynq and shutter control module. After self-test shutter module responds with error code. Error code is 1 byte long. Error value 0 means that self-test was passed. Error value higher than 0 means that self-test failed.

\item \textbf{Shutter diagnostics during normal operation} \hfill \\
During normal operation on every shutter event (opening/closing) same as self-test diagnostic measurements are taken and monitored.

\item \textbf{Emergency hardware shutdown} \hfill \\
In case of overheating of coil driver, driver shuts down itself and asserts signal to the control FPGA. Additionally there is possibility of external control of shutter power supply, so in case of emergency shutter power can be completely cut off.

\end{description}

\subsubsection{ADC Interface \& VDMA subsystem}
It is possible to turn AD7960 ADCs used in project into diagnostic mode in which known, pseudo-random test pattern is generated and output for readout. ADC interface test relies on test pattern verification.
Test data is first read as normal video frame, and then compared to known pattern. If data and pattern match then test is passed, otherwise test is failed. ADC generates 18 - bit samples but during processing 2 LSBs are discarded. Thus 16 MSBs are tested. This test verifies both ADC readout correctness and DMA subsystem. In case of failure of any element of data flow pipeline, samples will become corrupted and test pattern verification will not pass.

\subsection{RTOS}
There won't be any special keepalive nor heartbeat mechanism. The information about presence and proper operation of RTOS will be based on time elapsed since last communication (eg. diagnostic readout). Since Linux OS is master node for all of the IPC transmission there will be RTOS reset mechanism implemented in Linux.  In case of lack of answer in timely fashion, RTOS firmware is reloaded, and camera hardware is reinitialized.

\subsection{Linux}
\subsubsection{Booting procedure self-test and logging}
When system starts, first self-test procedure is being run. The results of self-test are stored on non-volatile memory (SD card) so that they are available in case of system start failure.
\todo[inline, caption={Linux boot diag}]{Przemyśleć dobrze diagnostykę, jakie mamy możliwości interakcji z komponentami kamery?}

\subsubsection{Watchdog}
ARM core running Linux OS is connected to hardware watchdog. In case of system halt for a longer period of time (heartbeat not being generated by Linux), a hardware reset is asserted causing the camera to restart.

\subsection{CCD}
In diagnostic mode shutter control is decoupled from CCD readout which enables various CCD diagnostic procedures. For example it is possible to to read the CCD dark field while keeping the shutter closed.

\subsection{Memory}
 \todo[inline, caption={Test pamięci}]{Na czym polega test pamięci?}
 
\subsection{Camera environmental parameters}
There are various sensors attached to different camera PCBs for monitoring and diagnostic purposes (apart from ones dedicated for specialized camera subsystems such as cooling and shutter). Physical quantities monitored:
\begin{itemize}
\item temperature (main board, communication board, CCD board, chiller output)
\item humidity (main board, communication board, CCD board)
\item voltages (communication board)
\item acceleration (mainboard)
\item coolant flow rate
\end{itemize}

Measurements from all of this sensors are periodically read by RTOS, converted to float numeric format and written to shared buffer. Monitoring process on Linux then accesses this data and takes appropriate action in case of any parameters exceeding set limits.
Apart from software monitoring, there is also hardware temperature monitor attached to mainboard which shuts down power for entire camera in case of exceeding temperature limit. There is external voltage sustaining capacitor connected to Zynq SOC, which enables writing diagnostic log in case of such event before system shuts down completely.

\subsection{Diagnostic logging}
There are two independent, timestamped logs of events anticipated for the camera – both written to the SD card. One from the Linux OS and the other from the RTOS.
 \todo[inline, caption={Diagnostyka - logowanie}]{Diagnostyka na zewnętrzny serwer syslog? Jakiś custom? Przez EPICS wszystko? Alarmy?}
 
\section{Camera calibration}

\subsection{Introduction}
There are predicted two stages of camera calibration: first in the lab before shipping and second while the camera is mounted on the telescope with optical system. Image correction utilizing calibration information and frames will be done on higher level software running on acquisition client workstation. Camera shutter is fully programmable and can be controlled manually as well, what enables various calibration procedures. All calibrating parameters will be provided with detailed description of methods used to obtain them. Because calibration is a high level process based on basic camera functionality, detailed calibration procedures will be delivered at later stage along with high level camera operation software.

\subsection{Laboratory calibration}

\subsubsection{Analog readout channels calibration}
There's no analog correction procedure anticipated for analog readout channels in order to not introduce additional noise. Amplifiers in the read channels are designed to scale full well CCD output signal to the full scale of analog-digital converters. Any correction will have to be done later, on digital data.

\subsubsection{Dark, bias, light frames \& flat fielding}
Camera has ability to collect all kind of calibration frames: dark, bias and flat-field frames. Master frames of all kinds will be measured and generated in laboratory conditions in normal camera operating temperature in basic operation mode and provided for each camera. 

\subsubsection{Gain}
%source: http://www.photometrics.com/resources/learningzone/gain.php
%http://www.mirametrics.com/tech_note_ccdgain.htm
Gain of each of the cameras will be measured in the lab before shipping. \\
Simple gain measurement method:
\begin{itemize}
\item Collect bias image
\item Collect two flat images
\item Calculate difference: diff = flat1 - flat0
\item Calculate the standard deviation of the central 100x100 pixels in difference image
\item Calculate variance from standard deviation by squaring and dividing by 2
\item Calculate a bias-corrected image by subtracting the bias from one of the flat images: corr = flat0 - bias
\item Calculate the mean illumination level by calculating the mean of the central 100x100 ROI of the corr image
\item gain=mean/variance
\end{itemize}

If more rigorous method is required one of the following procedures can be used:
\begin{itemize}
\item Mortara and Fowler (\emph{SPIE Vol. 290 Solid State Imagers for Astronomy} (1981) pp. 28-33)
\item Janesick et al. (\emph{Optical Engineering Vol. 26} (10) (1987) pp. 972-980)
\end{itemize}

\subsubsection{Linearity}
%source: http://www.photometrics.com/resources/learningzone/linearity.php
Linearity of each of the cameras will be measured in lab before shipping. 
Linearity will be measured according to the following procedure:
\begin{itemize}
\item Camera will be placed in a room with stable light source (i.e. flat panel display)
\item Test pictures will be taken with even increments in exposure from zero until CCD saturates
\item For each exposure mean image illumination will be calculated
\item Linear least-squares regression will be conducted on illumination-exposure data
\item The deviation of each point from the calculated line produces measure of nonlinearity of the system
\end{itemize}

\subsubsection{Cosmetics}
Selected CCD sensors can have following imperfections:
\begin{itemize}
\item \textbf{white spots} \hfill \\
White spot defect is qualified as such when dark generation rate is >5 e/pixel/s at 173K.
\item \textbf{black spots} \hfill \\
Black spot is a pixel with photo-response less than 50\% of the local mean.
\item \textbf{column defects} \hfill \\
Column is counted as defect if it contains at least 100 white or dark single pixel defects.
\item \textbf{traps} \hfill \\
A trap causes charge to be temporarily held in a charge well. It is counted as defect if the quantity of trapped charge is greater than 200 e.
\end{itemize}

Value of defect pixels will be interpolated from neighboring pixels - there will be provided several algorithms to do this - simplest one can be for e.g. arithmetic mean of all pixels values in 8-connected neighborhood. \\
Pixel defect maps for each camera will be created in the lab by manually inspecting test images in various operating conditions.

\subsubsection{Point Spread Function}
Experimental determination of PSF for CCD sensor itself is not anticipated. CCD manufacturer's data will be provided.

\subsubsection{Charge Transfer Efficiency}
CTE measurement or calculation is not anticipated. CCD manufacturer data will be provided.

\subsection{Correction software}
Software running on acquisition client workstation will be responsible for all image corrections. It will have central database of calibration frames, defect maps and other calibration parameters for all cameras in the ensemble. Apart from image correction, software will also perform source centroiding and diameter calculation.

\subsection{In site calibration with mounted optics}
Flat fielding and PSF will have to be repeated on site in order to accommodate effects introduced by optics. In site calibration methods employing optics will need to be discussed further in order to clarify requirements and methods.
