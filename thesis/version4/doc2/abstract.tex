% $Log: abstract.tex,v $
% Revision 1.0  11.2015 % 
% 
%
%% The text of your abstract and nothing else (other than comments) goes here.
%% It will be single-spaced and the rest of the text that is supposed to go on
%% the abstract page will be generated by the abstractpage environment.  This
%% file should be \input (not \include 'd) from cover.tex.

%Original text:
%In this thesis, I designed and implemented a compiler which performs
%optimizations that reduce the number of low-level floating point operations
%necessary for a specific task; this involves the optimization of chains of
%floating point operations as well as the implementation of a ``fixed'' point
%data type that allows some floating point operations to simulated with integer
%arithmetic.  The source language of the compiler is a subset of C, and the
%destination language is assembly language for a micro-floating point CPU.  An
%instruction-level simulator of the CPU was written to allow testing of the
%code.  A series of test pieces of codes was compiled, both with and without
%optimization, to determine how effective these optimizations were.


In this Master Thesis a Scientific Camera framework design is presented. As far as an embedded system design is
concerned, scientific cameras present a great engineering effort in order to succesfully design and implement this kind
of device. This is why this framework was created, so that an engineer wanting to quickly test his or her design can
benefit from it. Proposed framework is build using Xilinx Zynq SoC and thus allows for creating a camera that has 
a multigigabit data acquisition capability as well as multigigabit transmission using SATA or 10 GbE interfaces. 
What is more, a designer can benefit from using a heterogenous operating system where on a multicore processor 
one core is running a real-time operating system whereas on the second core an embedded Linux operating system is 
being run. This provides a great deal of possibilites for numerous applications. Another feature of the framework is 
the multichannel support where multiple cameras can be synchronised using either a dedicated MLVDS interface or 
Ethernet based Precision-Time-Protocol. Mentioned features given as a tested and ready to use subsusystems of a Xilinx 
Vivado project allows for quicker and less error prone scientific camera design. Project is targeted specifically 
towards scientific camera systems due to the fact that this market is very broad and every project has different
requirements. What is common in all scientific camera projects is the sensor data acquisition, synchronisation
capability as well as data transmission. This proves that such a framework can greatly speed up the development 
of a scientific camera. 
