% Piotr Zdunek Master Thesis
% Faculty of Electronics and Computer Science
% Warsaw University of Technology
% Institute of Electronics Systems
% Prepared on the basis of MIT Master thesis template
% Link : 

% -*- Mode:TeX -*-

%% IMPORTANT: The official thesis specifications are available at:
%%            http://libraries.mit.edu/archives/thesis-specs/
%%
%%            Please verify your thesis' formatting and copyright
%%            assignment before submission.  If you notice any
%%            discrepancies between these templates and the 
%%            MIT Libraries' specs, please let us know
%%            by e-mailing thesis@mit.edu

%% The documentclass options along with the pagestyle can be used to generate
%% a technical report, a draft copy, or a regular thesis.  You may need to
%% re-specify the pagestyle after you \include  cover.tex.  For more
%% information, see the first few lines of mitthesis.cls. 

%\documentclass[12pt,vi,twoside]{mitthesis}
%%
%%  If you want your thesis copyright to you instead of MIT, use the
%%  ``vi'' option, as above.
%%
%\documentclass[12pt,twoside,leftblank]{mitthesis}
%%
%% If you want blank pages before new chapters to be labelled ``This
%% Page Intentionally Left Blank'', use the ``leftblank'' option, as
%% above. 
\documentclass[a4paper,12pt,twoside]{mitthesis}
%\usepackage[sorting=none]{biblatex}
\usepackage{lgrind}
%% These have been added at the request of the MIT Libraries, because
%% some PDF conversions mess up the ligatures.  -LB, 1/22/2014
\usepackage{cmap}
%\usepackage{siunit}
\usepackage[utf8]{inputenc}
\usepackage[T1]{fontenc}
\usepackage{lmodern}
\DeclareUnicodeCharacter{00A0}{ }
%\usepackage[utf8]{inputenc}
%\pagestyle{plain}
\usepackage{setspace}% http://ctan.org/pkg/setspace
% Piotr Zdunek packages
\usepackage{graphicx}
\usepackage{caption}
\usepackage{subcaption}
\usepackage{courier}
\usepackage{color}
\usepackage{pgf}
\usepackage{tikz}
\usepackage{pdfpages}
\usetikzlibrary{arrows,automata}
%colors for the listings
\definecolor{sh_comment}{rgb}{0.12, 0.38, 0.18 } %adjusted, in Eclipse: {0.25, 0.42, 0.30 } = #3F6A4D
\definecolor{sh_keyword}{rgb}{0.37, 0.08, 0.25}  % #5F1441
\definecolor{sh_string}{rgb}{0.06, 0.10, 0.98} % #101AF9

\usepackage{listings}
\lstset {
    basicstyle=\ttfamily,
    numberstyle=\tiny,numbers=left,
    breaklines=true
}

%\lstset {
%    frame=shadowbox,
%    rulesepcolor=\color{black},
%    showspaces=false,showtabs=false,tabsize=2,
%    numberstyle=\tiny,numbers=left,
%    basicstyle= \footnotesize\ttfamily,
%    stringstyle=\color{sh_string},
%    keywordstyle = \color{sh_keyword}\bfseries,
%    commentstyle=\color{sh_comment}\itshape,
%    captionpos=b,
%    xleftmargin=0.7cm, xrightmargin=0.5cm,
%    lineskip=-0.3em,
%    breaklines=true
%}
%
%compact enumerations
\usepackage{enumitem}
\setitemize{noitemsep,topsep=0pt,parsep=0pt,partopsep=0pt}

\usepackage{float}
\usepackage{tabularx}
\def\tabularxcolumn#1{m{#1}}
\usepackage{footnote}

%no borders around hyperlinks
\usepackage{hyperref}
\hypersetup{%
    pdfborder = {0 0 0},
    colorlinks,
    citecolor=black,
    filecolor=black,
    linkcolor=black,
    urlcolor=black
}
\usepackage{xcolor}
\lstset{
    frame=single,
    basicstyle=\scriptsize,
    showstringspaces=false,
    commentstyle=\color{red},
    keywordstyle=\color{blue}
}

\usepackage{acronym}
\usepackage[nonumberlist,nopostdot]{glossaries}
\renewcommand{\glossarypreamble}{\scriptsize}
\newcommand{\dictentry}[2]{%
    \newglossaryentry{#1}{name=#1,description={#2}}%
    \glslink{#1}{}%
    \glsgroupskip
}
\makeglossaries

%% This bit allows you to either specify only the files which you wish to
%% process, or `all' to process all files which you \include.
%% Krishna Sethuraman (1990).

%This produces a bug (pzdunek)
%\typein [\files]{all}
%{Enter file names to process, (chap1,chap2 ...), or `all' to
%process all files:}
%\def\all{all}
%\ifx\files\all \typeout{Including all files.} \else \typeout{Including only %\files.} \includeonly{\files} \fi

%Panowie, nie jest tak zle jak to wygląda- poniżej zalaczam mail o Tarapaty.
%Może Piotrze skupisz się na samej wielokanalowosci i synchronizacji?
%Wymagaloby to przepisania samej koncepcji i w realizacji dopisania kawalka
%skupiającego się na wybranym temacie.
%

%We wstępie wygladaloby to tak:
%Wychodzisz z tematu kamer, dalej go zawężasz do kamer naukowych (scientific
%grade) następnie dalej zawężasz temt do kamer obrazujących w zakresie X.
%I tutaj sa w zasadzie 2 mozliwosci - kamery ze scyntylatorem, gdzie nie ma
%możliwości dykryminacji energii fotonu oraz bazujące na detektorach GEM i
%krzemowych.

%O GEMach pisales w pracy inz, wiec możesz si sam zacytować, skupiasz się na
%detektorach krzemowych. Wywalasz oczywiste obrazki mowiace o np. ISO.
%Na rynku w zasadzie sa 2 czy 3 w tym Medpix. Dalej piszesz ze bierzesz
%udział w projekcie budowy takiej kamery z partnerem naukowym gdzie jesteś
%odpowiedzialny za opracowanie firmware i HDL. I tutaj pojawiają się
%konkretne wymagania dla czujnika X. Podkreslasz ze opracowana kamera będzie
%miała romzmiar pozwalający skladac z kawalkow. Piszesz ze czujnik X jest w
%opracowywaniu wiec tworzysz model firmware na  czujniku z podobnym
%interfejsem -CMV4000 i tutaj opisujesz co robiles. Dalej, w następnym
%rozdziale  piszesz już ogólniej o integracji i testach z czujnikiem X.
%Potem piszesz o sposobach w jaki można przechwycić sygnal z czujnika i dalej
%go wysylac w swiat, czyli to co masz.

%Potem pojawia się koncepcja w której opisujesz konkretna realizacje ze
%szczgolowym podzialem na bloki. Bez podawania schmatow ideowych.
%- układ deserializera
%- układ sekwencera
%- układ video DMA z buforami
%- układ generacji napiec i biasow
%- system Linuxowy i jego rola
%- interfejsy komunikacyjne. O SATA tylko wspominasz ze uruchomiles dla celów
%diagnostycznych.
%- oprogramowanie sterujace
%
%Jaki masz wkład naukowy?
%- opracowanie metody weryfikacji sprzętu bez dostępnego czujnika poprzez
%uzycie czujnika o podobnym interfejsie. Dzieki blokowej architekturze
%(frameworku) da się potem latwo podmienić czujnik.
%- opracowanie sposobu synchronizacji wielu kamer od strony sprzętowej i
%programowej
%- deserializacja sygnalu z interfejsow szeregowych czujnika ze sledzeniem
%fazy sygnalu. Przeciez w CMV to robiles.
%- weryfikacja oprogramowania i sprzętu metoda drobnych kroczków, czyli to co
%AGH na nas wymuszal. Testowanie wszystkich blokow a na samym końcu czujnika.
%
%Dalej przy testach oczywiście pokazujesz sygnal z czujnika CMV pisząc ze
%czujnik X nie był jeszcze gotowy do integracji bo w zasadzie nie był gdyż
%AGH nie pozwolil nam tego robic.
%Opisujesz metody testowania pamięci, Ethernetu, serializerow i
%deserializerow. Mase czasu nad tym spedziles. Podlaczenie czujnika to potem
%formalnosc


\begin{document}

% -*-latex-*-
% 
% For questions, comments, concerns or complaints:
% thesis@mit.edu
% 
%
% $Log: cover.tex,v $
% Revision 1.8  2008/05/13 15:02:15  jdreed
% Degree month is June, not May.  Added note about prevdegrees.
% Arthur Smith's title updated
%
% Revision 1.7  2001/02/08 18:53:16  boojum
% changed some \newpages to \cleardoublepages
%
% Revision 1.6  1999/10/21 14:49:31  boojum
% changed comment referring to documentstyle
%
% Revision 1.5  1999/10/21 14:39:04  boojum
% *** empty log message ***
%
% Revision 1.4  1997/04/18  17:54:10  othomas
% added page numbers on abstract and cover, and made 1 abstract
% page the default rather than 2.  (anne hunter tells me this
% is the new institute standard.)
%
% Revision 1.4  1997/04/18  17:54:10  othomas
% added page numbers on abstract and cover, and made 1 abstract
% page the default rather than 2.  (anne hunter tells me this
% is the new institute standard.)
%
% Revision 1.3  93/05/17  17:06:29  starflt
% Added acknowledgements section (suggested by tompalka)
% 
% Revision 1.2  92/04/22  13:13:13  epeisach
% Fixes for 1991 course 6 requirements
% Phrase "and to grant others the right to do so" has been added to 
% permission clause
% Second copy of abstract is not counted as separate pages so numbering works
% out
% 
% Revision 1.1  92/04/22  13:08:20  epeisach

   \begin{figure}
     \centering
       \includegraphics[scale=0.65]{img/titlepage/conver_logo.png}
   \end{figure}


% NOTE:
% These templates make an effort to conform to the MIT Thesis specifications,
% however the specifications can change.  We recommend that you verify the
% layout of your title page with your thesis advisor and/or the MIT 
% Libraries before printing your final copy.
\title{Master Thesis}
\title{High speed multichannel camera framework with 10 GbE for scientific applications}
%\title{Scientific camera design framework}

\author{Piotr Zdunek}
% If you wish to list your previous degrees on the cover page, use the 
% previous degrees command:
%       \prevdegrees{A.A., Harvard University (1985)}
% You can use the \\ command to list multiple previous degrees
%       \prevdegrees{B.S., University of California (1978) \\
%                    S.M., Massachusetts Institute of Technology (1981)}
\department{Faculty of Electronics and Information Technology}

% If the thesis is for two degrees simultaneously, list them both
% separated by \and like this:
% \degree{Doctor of Philosophy \and Master of Science}
\degree{Master of Science}

% As of the 2007-08 academic year, valid degree months are September, 
% February, or June.  The default is June.
\degreemonth{Feburary}
\degreeyear{2016}
\thesisdate{20.01.2016}

%% By default, the thesis will be copyrighted to MIT.  If you need to copyright
%% the thesis to yourself, just specify the `vi' documentclass option.  If for
%% some reason you want to exactly specify the copyright notice text, you can
%% use the \copyrightnoticetext command.  
\copyrightnoticetext{\copyright Warsaw University of Technology}

% If there is more than one supervisor, use the \supervisor command
% once for each.
\supervisor{Grzegorz Kasprowicz}{Doctor of Philosophy}

% Make the titlepage based on the above information.  If you need
% something special and can't use the standard form, you can specify
% the exact text of the titlepage yourself.  Put it in a titlepage
% environment and leave blank lines where you want vertical space.
% The spaces will be adjusted to fill the entire page.  The dotted
% lines for the signatures are made with the \signature command.
\maketitle

% The abstractpage environment sets up everything on the page except
% the text itself.  The title and other header material are put at the
% top of the page, and the supervisors are listed at the bottom.  A
% new page is begun both before and after.  Of course, an abstract may
% be more than one page itself.  If you need more control over the
% format of the page, you can use the abstract environment, which puts
% the word "Abstract" at the beginning and single spaces its text.

%% You can either \input (*not* \include) your abstract file, or you can put
%% the text of the abstract directly between the \begin{abstractpage} and
%% \end{abstractpage} commands.

% First copy: start a new page, and save the page number.
\cleardoublepage
% Uncomment the next line if you do NOT want a page number on your
% abstract and acknowledgments pages.
\pagestyle{empty}
\setcounter{savepage}{\thepage}
\begin{abstractpage}
% $Log: abstract.tex,v $
% Revision 1.0  11.2015 % 
% 
%
%% The text of your abstract and nothing else (other than comments) goes here.
%% It will be single-spaced and the rest of the text that is supposed to go on
%% the abstract page will be generated by the abstractpage environment.  This
%% file should be \input (not \include 'd) from cover.tex.

%Original text:
%In this thesis, I designed and implemented a compiler which performs
%optimizations that reduce the number of low-level floating point operations
%necessary for a specific task; this involves the optimization of chains of
%floating point operations as well as the implementation of a ``fixed'' point
%data type that allows some floating point operations to simulated with integer
%arithmetic.  The source language of the compiler is a subset of C, and the
%destination language is assembly language for a micro-floating point CPU.  An
%instruction-level simulator of the CPU was written to allow testing of the
%code.  A series of test pieces of codes was compiled, both with and without
%optimization, to determine how effective these optimizations were.


In this Master Thesis a Scientific Camera framework design is presented. As far as an embedded system design is
concerned, scientific cameras present a great engineering effort in order to succesfully design and implement this kind
of device. This is why this framework was created, so that an engineer wanting to quickly test his or her design can
benefit from it. Proposed framework is build using Xilinx Zynq SoC and thus allows for creating a camera that has 
a multigigabit data acquisition capability as well as multigigabit transmission using SATA or 10 GbE interfaces. 
What is more, a designer can benefit from using a heterogenous operating system where on a multicore processor 
one core is running a real-time operating system whereas on the second core an embedded Linux operating system is 
being run. This provides a great deal of possibilites for numerous applications. Another feature of the framework is 
the multichannel support where multiple cameras can be synchronised using either a dedicated MLVDS interface or 
Ethernet based Precision-Time-Protocol. Mentioned features given as a tested and ready to use subsusystems of a Xilinx 
Vivado project allows for quicker and less error prone scientific camera design. Project is targeted specifically 
towards scientific camera systems due to the fact that this market is very broad and every project has different
requirements. What is common in all scientific camera projects is the sensor data acquisition, synchronisation
capability as well as data transmission. This proves that such a framework can greatly speed up the development 
of a scientific camera. 

\end{abstractpage}
%
% Additional copy: start a new page, and reset the page number.  This way,
% the second copy of the abstract is not counted as separate pages.
% Uncomment the next 6 lines if you need two copies of the abstract
% page.
% \setcounter{page}{\thesavepage}
% \begin{abstractpage}
% % $Log: abstract.tex,v $
% Revision 1.0  11.2015 % 
% 
%
%% The text of your abstract and nothing else (other than comments) goes here.
%% It will be single-spaced and the rest of the text that is supposed to go on
%% the abstract page will be generated by the abstractpage environment.  This
%% file should be \input (not \include 'd) from cover.tex.

%Original text:
%In this thesis, I designed and implemented a compiler which performs
%optimizations that reduce the number of low-level floating point operations
%necessary for a specific task; this involves the optimization of chains of
%floating point operations as well as the implementation of a ``fixed'' point
%data type that allows some floating point operations to simulated with integer
%arithmetic.  The source language of the compiler is a subset of C, and the
%destination language is assembly language for a micro-floating point CPU.  An
%instruction-level simulator of the CPU was written to allow testing of the
%code.  A series of test pieces of codes was compiled, both with and without
%optimization, to determine how effective these optimizations were.


In this Master Thesis a Scientific Camera framework design is presented. As far as an embedded system design is
concerned, scientific cameras present a great engineering effort in order to succesfully design and implement this kind
of device. This is why this framework was created, so that an engineer wanting to quickly test his or her design can
benefit from it. Proposed framework is build using Xilinx Zynq SoC and thus allows for creating a camera that has 
a multigigabit data acquisition capability as well as multigigabit transmission using SATA or 10 GbE interfaces. 
What is more, a designer can benefit from using a heterogenous operating system where on a multicore processor 
one core is running a real-time operating system whereas on the second core an embedded Linux operating system is 
being run. This provides a great deal of possibilites for numerous applications. Another feature of the framework is 
the multichannel support where multiple cameras can be synchronised using either a dedicated MLVDS interface or 
Ethernet based Precision-Time-Protocol. Mentioned features given as a tested and ready to use subsusystems of a Xilinx 
Vivado project allows for quicker and less error prone scientific camera design. Project is targeted specifically 
towards scientific camera systems due to the fact that this market is very broad and every project has different
requirements. What is common in all scientific camera projects is the sensor data acquisition, synchronisation
capability as well as data transmission. This proves that such a framework can greatly speed up the development 
of a scientific camera. 

% \end{abstractpage}

%\cleardoublepage


%\section*{Acknowledgments}
%
%This is the acknowledgements section.  You should replace this with your
%own acknowledgements.

%\section{Glossary}
%\dictentry{BGA}{Ball Grid Array}
\dictentry{CMOS}{Complementary Metal-Oxide Semiconductor} 
\dictentry{PCIe}{Peripherial Component Interconnect Express}
\dictentry{GTP}{Gigabit Transceiver Port}
\dictentry{AC}{Alternating Current}
\dictentry{DC}{Direct Current}
\dictentry{FPGA}{Field Programmable Gate Array}
\dictentry{IC}{Integrated Circuit}
\dictentry{JTAG}{Joint Test Action Group}
\dictentry{DDR}{Double Data Rate}
\dictentry{PCB}{Printed Circuit Board}
\dictentry{PERG}{Photonics and Web Engineering Group}
\dictentry{SFP+} {Small form-factor pluggable transceiver}
\dictentry{Gbps} {Gigabits per second}
\dictentry{GFLOPs} {Giga Floating Point Operations Per Second}
\dictentry{SPI} {Serial Peripherial Interconnect}
\dictentry{I2C} {Inter Integrated Circuit}
\dictentry{ppm} {points per milion}
\dictentry{LVDS} {Low Voltage Differential Signaling}
\dictentry{GB} {Gigabyte}
\dictentry{EDA} {Electronic Design Automation}
\dictentry{OHWR} {Open Hardware Repository}
\dictentry{TTM} {Time To Market}
\dictentry{SoC} {System-On-Chip}
\dictentry{PC}	{Personal Computer}
\dictentry{OS}	{Operating System}
\dictentry{GPU}	{Graphics Processing Unit}
\dictentry{GPGPU}{General Purpose Graphics Processing Unit}
\dictentry{PS}{Processing System}
\dictentry{PL}{Programmable Logic}
\dictentry{SoC}{System-on-a-chip}
\dictentry{LPC}{Low Pin Count}
\dictentry{HPC}{High Pin Count}
\dictentry{MB}{Megabyte}
\dictentry{Mbps}{Mega bits per scond}
\dictentry{SD}{Secure Digital}
\dictentry{APU}{Application Processing Unit}
\dictentry{MP}{Megapixel}
\dictentry{px}{px}
\dictentry{PTP}{Precision-Time-Protocol}
\dictentry{SSD}{Solid State Drive}
\dictentry{CCD}{Charged Coupled Device}
\dictentry{CMOS}{Complementary Metal-Oxide-Semiconductor}
\dictentry{TCF}{Target Communication Framework}
\dictentry{GDB}{GNU Debugger}
\dictentry{MPU}{Main Processing Unit}

\glsnogroupskiptrue

\printglossary[style=index]
%\printglossaries



%%%%%%%%%%%%%%%%%%%%%%%%%%%%%%%%%%%%%%%%%%%%%%%%%%%%%%%%%%%%%%%%%%%%%%
% -*-latex-*-

%
%\large{\textbf{Tytuł:} System realizacji kamer do zastosowań naukowych.}\\

\thispagestyle{empty}
\setcounter{savepage}{\thepage}

\begin{center}

%\textbf{Title: Camera design framework for scientific applications}
\large{\textbf{Tytuł:} Akwizycja danych z czujników optycznych dla kamer do zastosowań naukowych}

\vspace{0.5cm}
\textbf{Streszczenie}

\end{center}
%
%
%W ramach niniejszej pracy magisterskiej wykonane zostało oprogramowanie na system wbudowany pozwalające na realizację 
%kamer do zastosowań naukowych. Rozwój kamer jest skomplikowanym i długotrwałym procesem. Projekt powstał, aby wspomóc
%tworzenie prototypu takiej kamery. Dzięki niemu możliwe jest szybkie sprawdzenie koncepcji swojej aplikacji 
%wykorzystując gotowe podsystemy.
%
%Aplikacja wykorzystuje Xilinx Zynq SoC dzięki czemu umożliwia realizację urządzenia posiadającego możliwość
%akwizycji znacznej ilości danych oraz możliwość transmisji danych z wielogigabitową przepustowością. 
%
%Dodatkowo system jest wyposażony w heterogeniczny system operacyjny, który pracuje na dwurdzeniowym układzie
%Cortex A9. Na jednym rdzeniu uruchomiony jest system czasu rzeczywistego FreeRTOS, a na drugim system operacyjny
%wysokiego poziomu Linux. Taki tandem pozwala na elastyczną realizację wielu aplikacji. Dodatkowo dzięki wykorzystaniu
%Ethernet system pozwala na pracę wielokanałową z wykorzystaniem standardu synchronizacji czasu
%Precision-Time-Protocol. 
%
%Wspomniane funkcje kamery zrealizowane w formie elastycznego systemu do prototypowania kamer pozwala na szybką realizację 
%prototypu kamery, gdzie podstawowe funkcjonalności są już zaimplementowane. Pozwala to na zmniejszenie ilości błędów oraz 
%sprawdzenie poprawności koncepcji. Mając na uwadze powyższe, projekt zrealizowany w ramach 
%pracy magisterskiej może mieć szerokie zastosowanie zarówno w przemyśle jak i w nauce. 
%
%W pierwszym rozdziale opisane zostały podstawowe informacje dotyczące kamer i uwzględnieniem szczególnych parametrów
%kamer do zastosowań naukowych. Rozdział 2 przedstawia genezę oraz wymagania projektu. Następnie, w rozdziale 3
%zaprezentowana została koncepcja realizacji aplikacji i w rozdziale 4 sama realizacja wraz z wynikami testów. Finalnie,
%rozdział 5 podsumowuje zrealizowany projekt. 
%
\cleardoublepage

% $Log: abstract.tex,v $
% Revision 1.0  11.2015 % 
% 
%
%% The text of your abstract and nothing else (other than comments) goes here.
%% It will be single-spaced and the rest of the text that is supposed to go on
%% the abstract page will be generated by the abstractpage environment.  This
%% file should be \input (not \include 'd) from cover.tex.

%Original text:
%In this thesis, I designed and implemented a compiler which performs
%optimizations that reduce the number of low-level floating point operations
%necessary for a specific task; this involves the optimization of chains of
%floating point operations as well as the implementation of a ``fixed'' point
%data type that allows some floating point operations to simulated with integer
%arithmetic.  The source language of the compiler is a subset of C, and the
%destination language is assembly language for a micro-floating point CPU.  An
%instruction-level simulator of the CPU was written to allow testing of the
%code.  A series of test pieces of codes was compiled, both with and without
%optimization, to determine how effective these optimizations were.


In this Master Thesis a Scientific Camera framework design is presented. As far as an embedded system design is
concerned, scientific cameras present a great engineering effort in order to succesfully design and implement this kind
of device. This is why this framework was created, so that an engineer wanting to quickly test his or her design can
benefit from it. Proposed framework is build using Xilinx Zynq SoC and thus allows for creating a camera that has 
a multigigabit data acquisition capability as well as multigigabit transmission using SATA or 10 GbE interfaces. 
What is more, a designer can benefit from using a heterogenous operating system where on a multicore processor 
one core is running a real-time operating system whereas on the second core an embedded Linux operating system is 
being run. This provides a great deal of possibilites for numerous applications. Another feature of the framework is 
the multichannel support where multiple cameras can be synchronised using either a dedicated MLVDS interface or 
Ethernet based Precision-Time-Protocol. Mentioned features given as a tested and ready to use subsusystems of a Xilinx 
Vivado project allows for quicker and less error prone scientific camera design. Project is targeted specifically 
towards scientific camera systems due to the fact that this market is very broad and every project has different
requirements. What is common in all scientific camera projects is the sensor data acquisition, synchronisation
capability as well as data transmission. This proves that such a framework can greatly speed up the development 
of a scientific camera. 

%
  % -*- Mode:TeX -*-
%% This file simply contains the commands that actually generate the table of
%% contents and lists of figures and tables.  You can omit any or all of
%% these files by simply taking out the appropriate command.  For more
%% information on these files, see appendix C.3.3 of the LaTeX manual. 
\tableofcontents
%\newpage
%\listoffigures
%\newpage
%\listoftables
%\dictentry{BGA}{Ball Grid Array}
\dictentry{CMOS}{Complementary Metal-Oxide Semiconductor} 
\dictentry{PCIe}{Peripherial Component Interconnect Express}
\dictentry{GTP}{Gigabit Transceiver Port}
\dictentry{AC}{Alternating Current}
\dictentry{DC}{Direct Current}
\dictentry{FPGA}{Field Programmable Gate Array}
\dictentry{IC}{Integrated Circuit}
\dictentry{JTAG}{Joint Test Action Group}
\dictentry{DDR}{Double Data Rate}
\dictentry{PCB}{Printed Circuit Board}
\dictentry{PERG}{Photonics and Web Engineering Group}
\dictentry{SFP+} {Small form-factor pluggable transceiver}
\dictentry{Gbps} {Gigabits per second}
\dictentry{GFLOPs} {Giga Floating Point Operations Per Second}
\dictentry{SPI} {Serial Peripherial Interconnect}
\dictentry{I2C} {Inter Integrated Circuit}
\dictentry{ppm} {points per milion}
\dictentry{LVDS} {Low Voltage Differential Signaling}
\dictentry{GB} {Gigabyte}
\dictentry{EDA} {Electronic Design Automation}
\dictentry{OHWR} {Open Hardware Repository}
\dictentry{TTM} {Time To Market}
\dictentry{SoC} {System-On-Chip}
\dictentry{PC}	{Personal Computer}
\dictentry{OS}	{Operating System}
\dictentry{GPU}	{Graphics Processing Unit}
\dictentry{GPGPU}{General Purpose Graphics Processing Unit}
\dictentry{PS}{Processing System}
\dictentry{PL}{Programmable Logic}
\dictentry{SoC}{System-on-a-chip}
\dictentry{LPC}{Low Pin Count}
\dictentry{HPC}{High Pin Count}
\dictentry{MB}{Megabyte}
\dictentry{Mbps}{Mega bits per scond}
\dictentry{SD}{Secure Digital}
\dictentry{APU}{Application Processing Unit}
\dictentry{MP}{Megapixel}
\dictentry{px}{px}
\dictentry{PTP}{Precision-Time-Protocol}
\dictentry{SSD}{Solid State Drive}
\dictentry{CCD}{Charged Coupled Device}
\dictentry{CMOS}{Complementary Metal-Oxide-Semiconductor}
\dictentry{TCF}{Target Communication Framework}
\dictentry{GDB}{GNU Debugger}
\dictentry{MPU}{Main Processing Unit}

\glsnogroupskiptrue

\printglossary[style=index]
%\printglossaries






\chapter{Introduction}

\lipsum[3-56]
\section{Project genesis}
\section{Motivation and Objectives}
\subsection{Literature review}
\subsection{Market review} 
\section{Requirements}
%Compact, low weight, high speed, two different sensors, rugged
\section{Thesis statement} 
%Make an embedded camera and evaluate the use of high-speed interfaces for hyper-spectral aerial applications. 

\chapter{Genesis}

In this master thesis an implementation of Precision-Time-Protocol Timestamping Unit in FPGA fabric for scientific
camera systems is presented. The project was completed at Photonics and Web Engineering Group at the Institute of 
Electronics Systems which has a significant contribution to X-ray measurement research (TODO publikacje).  Having a scientific cooperation with
another Polish university, there was a need to develop hardware and firmware for novel extremely high-speed,
multichannel, X-ray silicon based camera. This project is undergoing a patent application, and for this reason the 
detailed description of the project cannot be included in this thesis. 

Specifically, a time synchronisation system providing an accurate UTC time was required in order to correctly control
the exposure time between the systems' channels.  This master thesis focuses on that aspect of the project.   


\section{Problem statement}
Providing an accurate timestamping for modern scientific grade camera system is a \textbf{complicated engineering
problem}. The designed hardware for the camera system used Xilinx Zynq SoC\cite{XIL:ZYNQ} which has built
in timestamping capability in the Media Access Controller (MAC). Nevertheless, the timestamping register is not available for
to be read by the operating system and programmable logic \cite[16.4.2]{XIL:ZYNQ_TRM} and the provided functionality of timestamping from
Xilinx is limited and provides low accuracy \cite[16.2.7]{XIL:ZYNQ_TRM} and significant jitter \cite{XIL:PTP_TESTS}. 
Xilinx User Guide Number 585 - Technical Rerence Manual explicitly mentions the fact that the Timestamping Unit can be 
implemented in hardware (programmable logic) in order to achieve better accuracy. This has not been done before and 
this thesis provides the solution to the mentioned problem. 

\section{Solution}
The solution for the problem is to design a Timestamping Unit (TSU) in digital system in FPGA fabric for the Zynq SoC
and use the MAC's built in PTP filtering capability to use this IP Core as a replacement for the internal built in TSU.
What is more, an Ethernet driver modification is required to exchange the TSU and an external oscillator has to be added
to the system in order to precisely run the counters in the TSU. 

\section{Statement of Originality}

This solution provides a way to perform PTP based time synchronisation using Zynq SoC. There are
other methods which provide time synchronisation of different precision such as:
\begin{itemize}
    \item GPS
    \item NTP - precision of up to
    \item PTP (by standard) - sub-milisecond precision 
    \item White Rabbit - sub-nanosecond precision 
\end{itemize}

Nevertheless, the solution provided in this master thesis is \textbf{original}. Standard PTP in the
Zynq SoC does not function properly and in order to be able to use PTP on Zynq with high precision and low jitter,
TSU needs to be implemented in digital fabric.  


\chapter{Requirements}

\begin{itemize}
    \item provide timestamping capability with accuracy in ns range, better than built-in solution provided by Xilinx
    \item timestamping register value should be available by operating system and programmable logic


\end{itemize}

%
\chapter{My work}
\label{chapter4}

%Due to the fact that the design of the camera framework was a multidisciplinary project, some parts of it were designed by
%students and associates of The Institute of Electronics Systems, Photonics and Web Engineering Group and Division of
%Television, Institute Radioelectronics and Multimedia Technology at Warsaw University of 
%Technology. 
%
%People who greatly helped me during the development are those mentioned: Grzegorz Kasprowicz, Damian Krystkiewicz,
%Maciej Trochimiuk, Andrzej Abramowski, Bartłomiej Juszczyk, Adrian Byszuk oraz Krzysztof Sielewicz. 
%
%My work in this project was to perform the following tasks:
%\begin{itemize}
%    \item Petalinux operating system configuration 
%    \item modified Serial ATA IP Core in two SSD drive configuration
%    \item 1000Base-X support with PCS/PMA 
%    \item deserialisation of data from silicon counting sensor  
%    \item baremetal software for digital system control 
%    \item PTP synchronisation between multiple ZC706 Development Boards
%    \item AMP operating system tests
%    \item system tests
%\end{itemize}


\chapter{Concept of design}
In this chapter a concept of the design of the camera is presented. First of all, main camera requirements are shown and juxtaposed with possible solutions. Afterwards the specification of the design is described in detail.

\section{Main requirements}

\begin{itemize}
\item Framerate at 100 fps at 2048 x 2048 resolution
\item High speed interface 
\item Processing capability
\item Possible IMU integration
\end{itemize}

\section{Specification}

\begin{itemize}
\item Framerate at 180 fps at 2048 x 2048 resolution
for CMV4000 and 100 fps for CIS1910F
\item 6.25 Gbps interfaces: SDI, CoaXPress, Aurora, PCie
\item FPGA fabric for processing ability
\item RS485 for communication with IMU and master controller

\end{itemize}

\chapter{Camera development}




\chapter{Summary}
In this chapter the summary of the master thesis is presented. Firstly the thesis objectives are contrasted with the final results. Then the main aspects of the work are shown with the description of possible future work. At the end the final summary is shown. 

\section{Thesis objectives and results}
\section{Future work}
\section{Final summary}
%% This defines the bibliography file (main.bib) and the bibliography style.
%% If you want to create a bibliography file by hand, change the contents of
%% this file to a `thebibliography' environment.  For more information 
%% see section 4.3 of the LaTeX manual.
%\begin{singlespace}
\bibliography{./chap/bib}
\bibliographystyle{ieeetr}
%\end{singlespace}

\dictentry{BGA}{Ball Grid Array}
\dictentry{CMOS}{Complementary Metal-Oxide Semiconductor} 
\dictentry{PCIe}{Peripherial Component Interconnect Express}
\dictentry{GTP}{Gigabit Transceiver Port}
\dictentry{AC}{Alternating Current}
\dictentry{DC}{Direct Current}
\dictentry{FPGA}{Field Programmable Gate Array}
\dictentry{IC}{Integrated Circuit}
\dictentry{JTAG}{Joint Test Action Group}
\dictentry{DDR}{Double Data Rate}
\dictentry{PCB}{Printed Circuit Board}
\dictentry{PERG}{Photonics and Web Engineering Group}
\dictentry{SFP+} {Small form-factor pluggable transceiver}
\dictentry{Gbps} {Gigabits per second}
\dictentry{GFLOPs} {Giga Floating Point Operations Per Second}
\dictentry{SPI} {Serial Peripherial Interconnect}
\dictentry{I2C} {Inter Integrated Circuit}
\dictentry{ppm} {points per milion}
\dictentry{LVDS} {Low Voltage Differential Signaling}
\dictentry{GB} {Gigabyte}
\dictentry{EDA} {Electronic Design Automation}
\dictentry{OHWR} {Open Hardware Repository}
\dictentry{TTM} {Time To Market}
\dictentry{SoC} {System-On-Chip}
\dictentry{PC}	{Personal Computer}
\dictentry{OS}	{Operating System}
\dictentry{GPU}	{Graphics Processing Unit}
\dictentry{GPGPU}{General Purpose Graphics Processing Unit}
\dictentry{PS}{Processing System}
\dictentry{PL}{Programmable Logic}
\dictentry{SoC}{System-on-a-chip}
\dictentry{LPC}{Low Pin Count}
\dictentry{HPC}{High Pin Count}
\dictentry{MB}{Megabyte}
\dictentry{Mbps}{Mega bits per scond}
\dictentry{SD}{Secure Digital}
\dictentry{APU}{Application Processing Unit}
\dictentry{MP}{Megapixel}
\dictentry{px}{px}
\dictentry{PTP}{Precision-Time-Protocol}
\dictentry{SSD}{Solid State Drive}
\dictentry{CCD}{Charged Coupled Device}
\dictentry{CMOS}{Complementary Metal-Oxide-Semiconductor}
\dictentry{TCF}{Target Communication Framework}
\dictentry{GDB}{GNU Debugger}
\dictentry{MPU}{Main Processing Unit}

\glsnogroupskiptrue

\printglossary[style=index]
%\printglossaries




\listoffigures
\newpage
\listoftables
\newpage

\appendix
% TeX encoding = utf8
% TeX spellcheck = pl_PL

\chapter{CD-ROM}

\label{CDROM}

\begin{itemize}

    \item test
\end{itemize}







%\chapter{Zynq Embedded System Design Guide}\label{apndx:ZynqProjectTips}

This appendix is a general list of design guides for any Embedded System consiting of Xilinx Zynq SoC.

\begin{enumerate}
  \item General design guidelines
  \item Software design guidelines
  \item Digital system design guidelines
  \item Hardware design guidelines 

    \begin{enumerate}
      \item Add UART to PS so that it is possible to see system messages at boot
      \item Add I2C to PS so that it is possible to configure a peripheral before Linux boot
    \end{enumerate}

\end{enumerate}


\clearpage
\newpage

%\chapter{Code listings}

\section{Petalinux Applications}

\subsection{Autologin application}

\label{APP:AUTOLOGIN}

\begin{lstlisting}[language=bash]
ifndef PETALINUX
$(error "Error: PETALINUX environment variable not set. 
Change to the root of your PetaLinux install, and source the settings.sh file")
endif

include apps.common.mk

APP = autologin

# Add any other object files to this list below
APP_OBJS = autologin.o

all: build install

build: $(APP)

$(APP): $(APP_OBJS)
$(CC) $(LDFLAGS) -o $@ $(APP_OBJS) $(LDLIBS)

clean:
-rm -f $(APP) *.elf *.gdb *.o

.PHONY: install image

install: $(APP)
#	$(TARGETINST) -d $(APP) /bin/$(APP)
$(TARGETINST) -d -p 0755 autologin /etc/init.d/autologin
$(TARGETINST) -s /etc/init.d/autologin /etc/rc5.d/S99autologin

          %.o: %.c
$(CC) -c $(CFLAGS) -o $@ $<

help:
@echo ""
@echo "Quick reference for various supported build targets for $(INSTANCE)."
@echo "----------------------------------------------------"
@echo "  clean              clean out build objects"
@echo "  all                build $(INSTANCE) and install to rootfs host copy"
@echo "  build              build subsystem"
@echo "  install            install built objects to rootfs host copy"

\end{lstlisting}

\begin{lstlisting}[language=C]

/*
*
* Autologin application. 
* Ethernet doesn't want to start when there is no user logged in.
*
*/
#include <unistd.h>
#include <stdio.h>

int main(int argc, char *argv[])
{

  execlp("login","login","-f","root",0);

}


\end{lstlisting}

\subsection{1 GbE PCS/PMA driver setup}
\label{APP:PCS_PMA}

\begin{lstlisting}[language=bash]
#!/bin/sh

# Startup script for setting ethernet using a kernel module for PCS/PMA
# Author: Piotr Zdunek
# Date: 04.01.2016

echo "Setting up Ethernet.."
echo "Loading Ethernet kernel module..."
EMACPS_DIR=`find /lib -name xilinx_emacps_emio.ko`

insmod $EMACPS_DIR

# do not setup ethernet before kernel module is properly loaded
wait

echo "Setting up profile.."
touch /home/root/.profile
echo "echo CAMERA v1.0" >> /home/root/.profile
echo "ifconfig eth0  up" >> /home/root/.profile

camera-server --ControllerPort=500 --DataPort=700 \ 
--MaintenancePort=600 --LogFile=log.txt &

\end{lstlisting}

\subsection{Device Tree}
\label{LST:DEVICE_TREE}

\begin{lstlisting}[language=xml,caption{Device tree camera.dtsi}, label{LST:DEVICE_TREE}]]
{
    amba_pl: amba_pl {
        #address-cells = <1>;
        #size-cells = <1>;
        compatible = "simple-bus";
        ranges ;

        IO_axi_quad_spi_0: axi_quad_spi@41e00000 {
            compatible = "xlnx,xps-spi-2.00.a","xlnx,xps-spi-2.00.a";
            interrupt-parent = <&intc>;
            interrupts = <0 29 1>;
            status = "okay";
            reg = <0x41e00000 0x10000>;
            num-cs = <0x5>;
        };

 ctrl_reg: ctrl_reg3@43c00000 {
        compatible = "generic-uio";
        reg = <0x43c00000 0x10000>;
        xlnx,s00-axi-addr-width = <0x6>;
        xlnx,s00-axi-data-width = <0x20>;
    };
};
};
\end{lstlisting}

\label{LST:REG}
\begin{lstlisting}
\label{LST:CTRL_REG}
 ctrl_reg: ctrl_reg3@43c00000 {
        compatible = "generic-uio";
        reg = <0x43c00000 0x10000>;
        xlnx,s00-axi-addr-width = <0x6>;
        xlnx,s00-axi-data-width = <0x20>;
    };

\end{lstlisting}



%check the device tree entry for DDR3 MIG
\begin{lstlisting}

/dts-v1/;
/include/ "zynq-7000.dtsi"
/include/ "camera.dtsi"

/ {
    model = "Camera v1.0";
    compatible = "xlnx,zynq-7000";

    aliases {
        ethernet0 = &gem1;
        i2c0 = &i2c0;
        serial0 = &uart1;
        spi0 = &qspi;
    };

    memory {
        device_type = "memory";
        reg = <0x0 0x40000000>;
    };

    chosen {
        bootargs = "console=ttyPS0,115200 root=/dev/ram rw earlyprintk";
        linux,stdout-path = "/amba/serial@e0001000";
    };
};

&gem1 {
    compatible = "xlnx,ps7-ethernet-emio-1.00.a";
    status = "okay";
    phy-mode = "rgmii-id";
    phy-handle = <&phy1>;
    gt-reset-gpios = <&gpio0 54 1 >;

    phy1: phy@6 {
        reg = <6>;
    };
};

&i2c0 {
    status = "okay";
    clock-frequency = <400000>;

        i2c@0 {
            #address-cells = <1>;
            #size-cells = <0>;
            reg = <2>;
            eeprom@54 {
                compatible = "at,24c08";
                reg = <0x54>;
            };
        };

        i2c@1 {
            #address-cells = <1>;
            #size-cells = <0>;
            reg = <0>;
            si570: clock-generator@5d {
                #clock-cells = <0>;
                compatible = "silabs,si570";
                temperature-stability = <50>;
                reg = <0x5d>;
                factory-fout = <200000000>;
                clock-frequency = <100000000>;
            };
        };

        i2c@2 {
            #address-cells = <1>;
            #size-cells = <0>;
            reg = <0>;
            fan_ctrl@5e {
                compatible = "mi,max6639";
                reg = <0x5e>;
            };
        };
};

&sdhci0 {
    status = "okay";
};

&uart1 {
    status = "okay";
}
\end{lstlisting}



\end{document}

