\chapter{Code listings}

\section{Petalinux Applications}

\subsection{Autologin application}

\label{APP:AUTOLOGIN}

\begin{lstlisting}[language=bash]
ifndef PETALINUX
$(error "Error: PETALINUX environment variable not set. 
Change to the root of your PetaLinux install, and source the settings.sh file")
endif

include apps.common.mk

APP = autologin

# Add any other object files to this list below
APP_OBJS = autologin.o

all: build install

build: $(APP)

$(APP): $(APP_OBJS)
$(CC) $(LDFLAGS) -o $@ $(APP_OBJS) $(LDLIBS)

clean:
-rm -f $(APP) *.elf *.gdb *.o

.PHONY: install image

install: $(APP)
#	$(TARGETINST) -d $(APP) /bin/$(APP)
$(TARGETINST) -d -p 0755 autologin /etc/init.d/autologin
$(TARGETINST) -s /etc/init.d/autologin /etc/rc5.d/S99autologin

          %.o: %.c
$(CC) -c $(CFLAGS) -o $@ $<

help:
@echo ""
@echo "Quick reference for various supported build targets for $(INSTANCE)."
@echo "----------------------------------------------------"
@echo "  clean              clean out build objects"
@echo "  all                build $(INSTANCE) and install to rootfs host copy"
@echo "  build              build subsystem"
@echo "  install            install built objects to rootfs host copy"

\end{lstlisting}

\begin{lstlisting}[language=C]

/*
*
* Autologin application. 
* Ethernet doesn't want to start when there is no user logged in.
*
*/
#include <unistd.h>
#include <stdio.h>

int main(int argc, char *argv[])
{

  execlp("login","login","-f","root",0);

}


\end{lstlisting}

\subsection{1 GbE PCS/PMA driver setup}
\label{APP:PCS_PMA}

\begin{lstlisting}[language=bash]
#!/bin/sh

# Startup script for setting ethernet using a kernel module for PCS/PMA
# Author: Piotr Zdunek
# Date: 04.01.2016

echo "Setting up Ethernet.."
echo "Loading Ethernet kernel module..."
EMACPS_DIR=`find /lib -name xilinx_emacps_emio.ko`

insmod $EMACPS_DIR

# do not setup ethernet before kernel module is properly loaded
wait

echo "Setting up profile.."
touch /home/root/.profile
echo "echo CAMERA v1.0" >> /home/root/.profile
echo "ifconfig eth0  up" >> /home/root/.profile

camera-server --ControllerPort=500 --DataPort=700 \ 
--MaintenancePort=600 --LogFile=log.txt &

\end{lstlisting}

\subsection{Device Tree}
\label{LST:DEVICE_TREE}

\begin{lstlisting}[language=xml,caption{Device tree camera.dtsi}, label{LST:DEVICE_TREE}]]
{
    amba_pl: amba_pl {
        #address-cells = <1>;
        #size-cells = <1>;
        compatible = "simple-bus";
        ranges ;

        IO_axi_quad_spi_0: axi_quad_spi@41e00000 {
            compatible = "xlnx,xps-spi-2.00.a","xlnx,xps-spi-2.00.a";
            interrupt-parent = <&intc>;
            interrupts = <0 29 1>;
            status = "okay";
            reg = <0x41e00000 0x10000>;
            num-cs = <0x5>;
        };

 ctrl_reg: ctrl_reg3@43c00000 {
        compatible = "generic-uio";
        reg = <0x43c00000 0x10000>;
        xlnx,s00-axi-addr-width = <0x6>;
        xlnx,s00-axi-data-width = <0x20>;
    };
};
};
\end{lstlisting}

\label{LST:REG}
\begin{lstlisting}
\label{LST:CTRL_REG}
 ctrl_reg: ctrl_reg3@43c00000 {
        compatible = "generic-uio";
        reg = <0x43c00000 0x10000>;
        xlnx,s00-axi-addr-width = <0x6>;
        xlnx,s00-axi-data-width = <0x20>;
    };

\end{lstlisting}



%check the device tree entry for DDR3 MIG
\begin{lstlisting}

/dts-v1/;
/include/ "zynq-7000.dtsi"
/include/ "camera.dtsi"

/ {
    model = "Camera v1.0";
    compatible = "xlnx,zynq-7000";

    aliases {
        ethernet0 = &gem1;
        i2c0 = &i2c0;
        serial0 = &uart1;
        spi0 = &qspi;
    };

    memory {
        device_type = "memory";
        reg = <0x0 0x40000000>;
    };

    chosen {
        bootargs = "console=ttyPS0,115200 root=/dev/ram rw earlyprintk";
        linux,stdout-path = "/amba/serial@e0001000";
    };
};

&gem1 {
    compatible = "xlnx,ps7-ethernet-emio-1.00.a";
    status = "okay";
    phy-mode = "rgmii-id";
    phy-handle = <&phy1>;
    gt-reset-gpios = <&gpio0 54 1 >;

    phy1: phy@6 {
        reg = <6>;
    };
};

&i2c0 {
    status = "okay";
    clock-frequency = <400000>;

        i2c@0 {
            #address-cells = <1>;
            #size-cells = <0>;
            reg = <2>;
            eeprom@54 {
                compatible = "at,24c08";
                reg = <0x54>;
            };
        };

        i2c@1 {
            #address-cells = <1>;
            #size-cells = <0>;
            reg = <0>;
            si570: clock-generator@5d {
                #clock-cells = <0>;
                compatible = "silabs,si570";
                temperature-stability = <50>;
                reg = <0x5d>;
                factory-fout = <200000000>;
                clock-frequency = <100000000>;
            };
        };

        i2c@2 {
            #address-cells = <1>;
            #size-cells = <0>;
            reg = <0>;
            fan_ctrl@5e {
                compatible = "mi,max6639";
                reg = <0x5e>;
            };
        };
};

&sdhci0 {
    status = "okay";
};

&uart1 {
    status = "okay";
}
\end{lstlisting}


