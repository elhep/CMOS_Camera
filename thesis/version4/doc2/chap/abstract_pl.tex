%\large{\textbf{Tytuł:} System realizacji kamer do zastosowań naukowych.}\\

\thispagestyle{empty}
\setcounter{savepage}{\thepage}

\begin{center}

%\textbf{Title: Camera design framework for scientific applications}
\large{\textbf{Tytuł:} Akwizycja danych z czujników optycznych dla kamer do zastosowań naukowych}

\vspace{0.5cm}
\textbf{Streszczenie}

\end{center}
%
%
%W ramach niniejszej pracy magisterskiej wykonane zostało oprogramowanie na system wbudowany pozwalające na realizację 
%kamer do zastosowań naukowych. Rozwój kamer jest skomplikowanym i długotrwałym procesem. Projekt powstał, aby wspomóc
%tworzenie prototypu takiej kamery. Dzięki niemu możliwe jest szybkie sprawdzenie koncepcji swojej aplikacji 
%wykorzystując gotowe podsystemy.
%
%Aplikacja wykorzystuje Xilinx Zynq SoC dzięki czemu umożliwia realizację urządzenia posiadającego możliwość
%akwizycji znacznej ilości danych oraz możliwość transmisji danych z wielogigabitową przepustowością. 
%
%Dodatkowo system jest wyposażony w heterogeniczny system operacyjny, który pracuje na dwurdzeniowym układzie
%Cortex A9. Na jednym rdzeniu uruchomiony jest system czasu rzeczywistego FreeRTOS, a na drugim system operacyjny
%wysokiego poziomu Linux. Taki tandem pozwala na elastyczną realizację wielu aplikacji. Dodatkowo dzięki wykorzystaniu
%Ethernet system pozwala na pracę wielokanałową z wykorzystaniem standardu synchronizacji czasu
%Precision-Time-Protocol. 
%
%Wspomniane funkcje kamery zrealizowane w formie elastycznego systemu do prototypowania kamer pozwala na szybką realizację 
%prototypu kamery, gdzie podstawowe funkcjonalności są już zaimplementowane. Pozwala to na zmniejszenie ilości błędów oraz 
%sprawdzenie poprawności koncepcji. Mając na uwadze powyższe, projekt zrealizowany w ramach 
%pracy magisterskiej może mieć szerokie zastosowanie zarówno w przemyśle jak i w nauce. 
%
%W pierwszym rozdziale opisane zostały podstawowe informacje dotyczące kamer i uwzględnieniem szczególnych parametrów
%kamer do zastosowań naukowych. Rozdział 2 przedstawia genezę oraz wymagania projektu. Następnie, w rozdziale 3
%zaprezentowana została koncepcja realizacji aplikacji i w rozdziale 4 sama realizacja wraz z wynikami testów. Finalnie,
%rozdział 5 podsumowuje zrealizowany projekt. 
%
\cleardoublepage
