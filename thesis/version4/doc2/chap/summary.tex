
\chapter{Summary}
%
%The camera framework, presented in this master thesis, has been successfully implemented. The designed system allowes
%for easy and straightforward adoption of a COTS sensor.
%The realised project files are provided with the thesis as is and are free to use and develop. Created framework is a 
%valid basis for any camera system development. 
%
%\section{Camera framework project main results}
%As a summary of the project it is beneficial to list all the main outcomes of the project which were evaluated during
%development. 
%
%\begin{itemize}
%
%    \item \textbf{Modern SoC are a viable solution for embedded camera systems}
%The use of Zynq SoC as the Main Processing Unit uncovered the versatility of this type of architecture. 
%Having an FPGA fabric as well as dual core application processor in one package allows the designer to implement
%various different applications, depending on the needs. 
%
%One has to bear in mind that the SoC is a complicated Integrated Circuit and making your application work may
%require significant development time and expertise. 
%
%\item \textbf{The FPGA fabric allows for using different types of sensors with the same hardware}
%The use of an FPGA in camera development allowed for designing two significantly different readout architectures for
%two types of sensors having just one piece of hardware (MPU). This is truly an outstanding result, due to the
%fact that normally one would require a different IC with a proper interface for a specific sensor. 
%
%The use of an FPGA in a camera system design is a viable solution and allows for having different types of
%sensors connected to the same hardware. 
%
%\end{itemize}
%
%\section{Future development}
%
%Due to the use of the state-of-the-art SoC Xilinx Zynq, the camera framework allows for upgrading some of its functions.
%One key upgrade can be the implementation of 10 GbE instead of 1 GbE, and as an addition upgrading to Serial ATA Gen 3 interface.
%Zynq SoC is capable of transmitting data using Multi Gigabit Transceivers with a throughput of up to 12.5 Gbps. 
%Additionally, a command encryption mechanism should be added to the throughput upgrade so that the commands are not
%sent as a plain text. 
%
%Furthermore, an incomparable time stamping accuracy can be achieved with the adoption of the White Rabbit protocol 
%which would allow for sub nanosecond time synchronisation. 
%
%As stated before some of the specified functions were troublesome to implement, but one of the possible upgrades of the 
%framework can be a use of dedicated RTOS which would allow for true deterministic soft-real time operation. 
%
%
