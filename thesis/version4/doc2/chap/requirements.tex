
\chapter{Requirements}\label{ch2:req}
% I tutaj pojawiają się
%konkretne wymagania dla czujnika X. Podkreslasz ze opracowana kamera będzie
%miała romzmiar pozwalający skladac z kawalkow. Piszesz ze czujnik X jest w
%opracowywaniu wiec tworzysz model firmware na  czujniku z podobnym
%interfejsem -CMV4000 i tutaj opisujesz co robiles. Dalej, w następnym
%rozdziale  piszesz już ogólniej o integracji i testach z czujnikiem X.
%Potem piszesz o sposobach w jaki można przechwycić sygnal z czujnika i dalej
%go wysylac w swiat, czyli to co masz.

%\section{X-band silicon sensor}
%%description of the sensor
%

\begin{enumerate}
    \item high throughput multiple serial data acuqisition - 16 differential pairs with up to 5 Gbps per line
    \item online and offline sensor data transfer and buffering
    \item multichannel operation, possiblity to trigger multiple cameras with high precision (~1 ns)
\end{enumerate}

%This chapter presents the requirements for the camera framework. 

%This chapter presents the genesis and requirements for the designed camera framework. Firstly, the genesis of the
%project is shown, then general requirements are stated which present the main goals to be achieved by the project. 
%Finally, the requirement analysis is performed and assumptions are presented. 
%
%\section{Project genesis}
%\label{ch2:genesis}
%
%The multitude and variety of sensors, interfaces, as well as requirements regarding specific camera projects make the design
%immensely difficult. Given the parameters for a specific application camera, the firmware, data acquisition and data
%transmission has to be designed specifically for each design. This poses a troublesome aspect of camera design that
%needs to be addressed.
%
%The purpose of my thesis is to address the issue of the scientific camera design. One cannot deny the fact that 
%\textbf{there is no framework for camera designs}. All designs are custom, and what is needed is a system which
%would allow for a faster time-to-market, less error prone development and high throughput connectivity. Also, an easy way
%of adding support for different sensors is essential. 
%
%\textbf{The thesis addresses this problem} by providing a scalable framework which can be used as a basis for high speed
%camera designs. 
%
%%Moreover, in Chapter 2 where \textit{Concept of design}  of the project is stated the main decisions are made.
%%These decisions have an influence on the possible margins of the requirements parameters values. In the end of concept 
%%phase the requirements will be revisited and a \emph{measurable} parameters will be provided for the project. 
%
% 
%
%
%\section{Requirements analysis}
%
%The main requirement of the camera framework is to make the development process of the prototype of a camera system
%easier. In order to do that, already tested subsystems can be used or analysed so that the development can be more
%straightforward.  
%
%It is essential in any engineering design to define the requirements, so that they are \emph{measurable}. 
%It might be difficult to define the exact values for each requirement
%at an early stage of the project, nevertheless some assumptions have to be made. What is more, this also relates to the
%design methodology - one cannot design a system right the first time, as any engineering design process is an iterative
%one where achievable parameter values are adjusted at each iteration. 
%
%The values stated here are gathered based on existing camera systems and available hardware. It is worth noting that none
%of the available open source cameras~\ref{ch1:existing} have all of those features and are in a  mature enough state to
%use them at all.  
%
%
%The requirements for the camera framework subsystems are listed below:
%
%\begin{enumerate}
%    \item \textbf{High processing performance} - data acquisition from the sensor
%        \begin{itemize}
%            \item  input datapath throughput equal to 600 MB/s which corresponds to 4 MP (2048 x 2048 px) at frame rate 100 fps, 12 bits per pixel   
%        \end{itemize}
%
%    \item \textbf{Ease of adding support for a different sensor} - the framework has to be designed this way so that the
%        designer can easily add support for a different type of sensor
%
%    \item \textbf{High speed communication} - so that it is possible to transfer the acquired data with high throughput
%        \begin{itemize}
%            \item 3 Gbps Serial ATA 
%            \item 1 Gbps Ethernet 
%            \item \textit{possible future upgrade} 10 Gbps Ethernet 
%        \end{itemize}
%
%    \item \textbf{Multichannel operation} - in order to support multiple cameras at the same time
%        \begin{itemize}
%            \item PTP Ethernet synchronisation - for synchronisation - using 1 Gbps Ethernet
%            \item \textit{possible future upgrade} MLVDS bus synchronisation for IP Cores controlling the sensor - short range
%            \item \textit{possible future upgrade} White Rabbit synchronisation - for subnanosecond (up to 10 km) synchronisation 
%        \end{itemize}
%
%    \item \textbf{Real time capability} - in order to perform real time tasks
%    \item \textbf{Versatile OS} - which supports high level languages and has got a lot of support
%        \begin{itemize}
%            \item GNU/Linux 
%            \item possibility to use a certified OS
%        \end{itemize}
%\end{enumerate}
%
%\section{Assumptions}
%As already described, the mechanics and enclosure choices are different for each camera project. 
%That is why this project does not consist of of optics design or other mechanical aspects of camera design. The
%goal of this project is to address the intersection of all camera problems from an embedded system design point of view.
%
%
%

%\subsection{Requirement analysis}\label{ch2:req_analysis}
%
% It is essential in any engineering design 
%to define the requirements, so that they are \emph{measurable}. It might be difficult to define the exact values for each requirement
%at an early stage of the project, nevertheless some assumptions have to be made. What is more, this also relates to the
%design methodology - one cannot design a system right the first time, as any engineering design process is an iterative
%one where achievable parameter values are adjusted at each iteration. 
%%~\ref{fig:pres_bottom_up_design_proc}. 
%
%%More information regarding the approach to project design is presented in section~\ref{ch2:methodology}.
%
%This section presents the requirements in their final form. The designed system will be confronted with the values
%specified here in the chapter~\ref{ch6:tests}. 
%
%The values stated here are gathered based on existing camera systems and available hardware. It is worth noting that none
%of the available open source cameras~\ref{ch1:existing} have all of those features and are in a  mature enough state to
%use them at all.  

%      \textbf{Requirements:}
%          \begin{enumerate}
%              \item \textbf{High data acquisition} - input datapath throughput equal to 600 MB/s\\
%                    which conforms to 4 MP (2048 x 2048 px) at frame rate 100 fps, 12 bits per pixel   
%                \item \textbf{Ease of adding a support for a different sensor} - use of an FPGA integrated circuit for
%                    sensor data acquisition 
%
%                \item \textbf{High speed communication} 
%                 \begin{enumerate}
%                   \item 3 Gbps Serial ATA 
%                   \item 1 Gbps Ethernet 
%                   \item \textit{possible future upgrade} 10 Gbps Ethernet 
%               \end{enumerate}
%             \item \textbf{Multichannel operation} 
%                 \begin{enumerate}
%                  \item PTP Ethernet synchronisation - for synchronisation - using 1 Gbps Ethernet
%                  \item \textit{possible future upgrade} MLVDS bus synchronisation for IP Cores controlling the sensor - short range
%                  \item \textit{possible future upgrade} White Rabbit synchronisation - for subnanosecond (up to 10 km) synchronisation 
%                \end{enumerate}
%            \item \textbf{Real time capability} - soft real time system - 1 ms interrupt and process switching time   
%            \item \textbf{Versatile OS}
%                \begin{enumerate}
%                  \item GNU/Linux 
%                  \item possibility to use a certified OS
%                \end{enumerate}
%
%          \end{enumerate}
% 
%
%\section{Acknowledgements}

%Due to the fact that the design of the camera framework was a multidisciplinary project, some parts of it were designed by
%students and associates of The Institute of Electronics Systems, Photonics and Web Engineering Group and Division of
%Television, Institute Radioelectronics and Multimedia Technology at Warsaw University of 
%Technology. 
%
%%People who greatly helped me during the development are those mentioned: Grzegorz Kasprowicz, Damian Krystkiewicz,
%%Maciej Trochimiuk, Andrzej Abramowski, Bartłomiej Juszczyk, Adrian Byszuk oraz Krzysztof Sielewicz. 
%
