
\chapter{Genesis}

Photonics and Web Engineering Group at the Institute of Electronics Systems had an immensely significant contribution to
X-ray based measurement scientific research throughout many years (references). %TODO: references
Having a scientific cooperation with another Polish university, there was a need to develop hardware and firmware for novel extremely high-speed,
multichannel, X-ray silicon based camera. My work in this project was to develop firmware and to design digital system.
This project is undergoing a patent application, and for this reason the detailed description of the project cannot be
included in this thesis. 

\section{X-band silicon sensor description}
Description


\section{Thesis problem}
The main thesis problem is the X-band counting silicon sensor data acquisition. 
This problem is not solvable by existing solutions and a custom one has to be designed. The main aspects of the X-band 
counting silicon sensor acquisition are:
\begin{itemize}
    \item high throughput interface 8x differential lines with maximum throughput of 5 Gbps per line
    \item independent of data deterministic control of the sensor with a configuration shift register
\end{itemize}

Another aspect of the thesis are the system level requirements, such as: multichannel operation and high speed interface
for sensor data transfer. 

%Dalej piszesz ze bierzesz
%udział w projekcie budowy takiej kamery z partnerem naukowym gdzie jesteś
%odpowiedzialny za opracowanie firmware i HDL. 

%The multitude and variety of sensors, interfaces, as well as requirements regarding specific camera projects make the design
%immensely difficult. Given the parameters for a specific application camera, the firmware, data acquisition and data
%transmission has to be designed specifically for each design. This poses a troublesome aspect of camera design that
%needs to be addressed.

\section{Solution}
The solution of the thesis problem is to design the readout from the sensor in FPGA fabric along with firmware to
facilitate other requirements. 

%
%The purpose of my thesis is to address the issue of the scientific camera design. One cannot deny the fact that 
%\textbf{there is no framework for camera designs}. All designs are custom, and what is needed is a system which
%would allow for a faster time-to-market, less error prone development and high throughput connectivity. Also, an easy way
%of adding support for different sensors is essential. 
%
%\textbf{The thesis addresses this problem} by providing a scalable framework which can be used as a basis for high speed
%camera designs. 


\section{Goals}
To provide a method for acquiring the data from the X-band counting silicon sensor.

\section{Limitations}
This project does not consist of hardware, mechanical or optics design.  
