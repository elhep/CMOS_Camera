% $Log: abstract.tex,v $
% Revision 1.0  11.2015 % 
% 
%
%% The text of your abstract and nothing else (other than comments) goes here.
%% It will be single-spaced and the rest of the text that is supposed to go on
%% the abstract page will be generated by the abstractpage environment.  This
%% file should be \input (not \include 'd) from cover.tex.

%Original text:
%In this thesis, I designed and implemented a compiler which performs
%optimizations that reduce the number of low-level floating point operations
%necessary for a specific task; this involves the optimization of chains of
%floating point operations as well as the implementation of a ``fixed'' point
%data type that allows some floating point operations to simulated with integer
%arithmetic.  The source language of the compiler is a subset of C, and the
%destination language is assembly language for a micro-floating point CPU.  An
%instruction-level simulator of the CPU was written to allow testing of the
%code.  A series of test pieces of codes was compiled, both with and without
%optimization, to determine how effective these optimizations were.

%\large{\textbf{Title:} Camera design framework for scientific applications.}\\

\thispagestyle{empty}
\setcounter{savepage}{\thepage}

\begin{center}


    \large{\textbf{Title: Sensor data acquisition for scientific grade cameras}}

\vspace{0.5cm}

\textbf{Abstract}

\end{center}

In this master thesis an FPGA implementation of sensor data acquisition for scientific grade camera is presented.
Novel X-band silicon sensors have a unique high throughput interface for data and control which needs to be controlled
 deterministically. The design, implementation and testing of digital system along with firmware pose a great
engineering challenge which was analysed and solved in this thesis.   

In the first chapter the basics of digital cameras, scientific grade cameras and X-band scientific cameras are
presented. Then, then in the second chapter a genesis of the master thesis problem is shown as well as goals and
limitations. In the third chapter, requirements for the sensor data acquisition are shown. 4th chapter presents concept
of design where digital system and firmware design is specified. In the 5th chapter the realisation and tests of
the sensor data acquisition is presented. Final chapter shows the summary of the thesis.  


%In this Master Thesis, a firmware for embedded  camera system is presented.
%Scientific cameras present a great engineering effort in order to successfully design and implement this kind of device. 
%Especially in scientific application where a reasearcher wants to test his or her hypothesis and needs to develop almost
%the whole camera. In order to speed up the development of any camera system, a set of ready-to-use subsystems were
%implemented.  
%
%The proposed solution is a type of a firmware framework and was built using Xilinx Zynq SoC, which is a combination of FPGA and
%dual-core Cortex A9. This allows for creating a versatile camera with multi gigabit data acquisition capability and high
%video processing performance with the use of MGT Transceivers and Programmable Logic.
%
%
%What is more, a designer can benefit from using a heterogenous operating system in which, on a multicore processor, 
%one core is running a real-time operating system, while on the second core, an embedded Linux operating system is 
%being run. This provides a great deal of possibilities for numerous applications. Another feature of the framework is 
%the multichannel support where multiple cameras can be synchronised using Ethernet interface with the use of the Precision-Time-Protocol.
%The mentioned features given as tested and ready to use subsystems of a Xilinx Vivado project allow for quicker 
%and less error-prone scientific camera design. The project is targeted specifically towards scientific camera systems due 
%to the fact that this market is broad and every project has different requirements. What is common in all 
%scientific camera projects is the sensor data acquisition, synchronisation capability, and data transmission. 
%
%No such framework (that is free to use) exists on the market, which proves that this project can greatly speed up the
%development of any scientific camera. 
%
%In the first chapter, basic information about cameras is presented with emphasis on the specifics of scientific
%cameras. Then, Chapter 2 presents the genesis and requirements of the project. Chapter 3 provides a concept of
%realisation of design and, in the end,  Chapter 4 describes the realisation as well as test results. 
%In the end, Chapter 6 gives the final summary of the project.  


\cleardoublepage
