\chapter{Genesis}

In this master thesis an implementation of Precision-Time-Protocol Timestamping Unit in FPGA fabric for scientific
camera systems is presented. The project was completed at Photonics and Web Engineering Group at the Institute of 
Electronics Systems which has a significant contribution to X-ray measurement research (TODO publikacje).  Having a scientific cooperation with
another Polish university, there was a need to develop hardware and firmware for novel extremely high-speed,
multichannel, X-ray silicon based camera. This project is undergoing a patent application, and for this reason the 
detailed description of the project cannot be included in this thesis. 

Specifically, a time synchronisation system providing an accurate UTC time was required in order to correctly control
the exposure time between the systems' channels.  This master thesis focuses on that aspect of the project.   


\section{Problem statement}
Providing an accurate timestamping for modern scientific grade camera system is a \textbf{complicated engineering
problem}. The designed hardware for the camera system used Xilinx Zynq SoC\cite{XIL:ZYNQ} which has built
in timestamping capability in the Media Access Controller (MAC). Nevertheless, the timestamping register is not available for
to be read by the operating system and programmable logic \cite[16.4.2]{XIL:ZYNQ_TRM} and the provided functionality of timestamping from
Xilinx is limited and provides low accuracy \cite[16.2.7]{XIL:ZYNQ_TRM} and significant jitter \cite{XIL:PTP_TESTS}. 
Xilinx User Guide Number 585 - Technical Rerence Manual explicitly mentions the fact that the Timestamping Unit can be 
implemented in hardware (programmable logic) in order to achieve better accuracy. This has not been done before and 
this thesis provides the solution to the mentioned problem. 

\section{Solution}
The solution for the problem is to design a Timestamping Unit (TSU) in digital system in FPGA fabric for the Zynq SoC
and use the MAC's built in PTP filtering capability to use this IP Core as a replacement for the internal built in TSU.
What is more, an Ethernet driver modification is required to exchange the TSU and an external oscillator has to be added
to the system in order to precisely run the counters in the TSU. 

\section{Statement of Originality}

This solution provides a way to perform PTP based time synchronisation using Zynq SoC. There are
other methods which provide time synchronisation of different precision such as:
\begin{itemize}
    \item GPS
    \item NTP - precision of up to
    \item PTP (by standard) - sub-milisecond precision 
    \item White Rabbit - sub-nanosecond precision 
\end{itemize}

Nevertheless, the solution provided in this master thesis is \textbf{original}. Standard PTP in the
Zynq SoC does not function properly and in order to be able to use PTP on Zynq with high precision and low jitter,
TSU needs to be implemented in digital fabric.  

