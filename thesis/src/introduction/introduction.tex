\pagestyle{plain}
\chapter{Introduction}
In this Master Thesis a design of an embedded camera and an evaluation of high speed interfaces is presented. The project is a part of a research project carried out by Center of space research in Warsaw. In this chapter a genesis of the project is presented as well as camera's main requirements. Additionally a literature review is shown together with market review. At the end of the chapter a thesis statement is presented.

%\dictentry{test,test2}
%\gls{formula}

\lipsum[3-56]
\section{Project genesis}

In this master thesis a design of a compact embedded camera as well as an evaluation of high speed interfaces for aerial multi-spectral imaging application is presented. 
The project was completed at Photonics and Web Engineering Group at the Institute of 
Electronics Systems which has a significant contribution in scientific camera development. (TODO publikacje).  Having a scientific cooperation with
Institute of Space Research, there was a need to develop hardware, firmware and evaluate the use of different high-speed interfaces for novel hyperspectral camera.

Specifically, a camera is needed inside of hyperspectral imaging camera for high speed data sensor acquisition, filtering and transmission. Additionally the camera, needs to support different sensor types which have different parameters in order to evaluate their usefulness in the hyper-spectral application. On top of that, the system is planned to be used in aerial vehicles (such as planes or drones) it needs to exhibit optimal SWAP factor and have interfaces who are highly reliable in this conditions. 

\subsection{Motivation and Objectives}
\section{Requirements}
\subsection{Literature review}
\subsection{Market review} 

%Compact, low weight, high speed, two different sensors, rugged
\section{Thesis statement} 
%Make an embedded camera and evaluate the use of high-speed interfaces for hyper-spectral aerial applications. 
